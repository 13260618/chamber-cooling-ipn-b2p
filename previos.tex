
%\pagenumbering{roman}

	\pagenumbering{roman}

\tableofcontents
\listoftables
\listoffigures 
 
\section*{Índice de Ecuaciones}
 
 \addcontentsline{toc}{chapter}{Índice de Ecuaciones}  % Agregar la sección al índice general
 
 % Listar todas las ecuaciones en este índice, incluyendo el número de página con \pageref
 \begin{enumerate}
 	\item Ecuación \ref{eq:Calor-de-enfriamiento}: Calor de enfriamiento \dotfill \pageref{eq:Calor-de-enfriamiento}
 	\item Ecuación \ref{eq:energia_especifica}: Energía específica de un cuerpo \dotfill \pageref{eq:energia_especifica}
 	\item Ecuación \ref{eq:carga_iluminacion_descarga}: Carga térmica por iluminación \dotfill \pageref{eq:carga_iluminacion_descarga}
 \end{enumerate}


\newpage


\section*{Planteamiento del problema}
\addcontentsline{toc}{chapter}{Planteamiento del problema}


La \textit{diabetes mellitus}\footnote{Enfermedad metabólica producida por deficiencias en la cantidad o en la utilización de la insulina, lo que produce un exceso de glucosa en la sangre \cite{RAE01} } representa sin lugar a dudas uno de los retos más importantes como causa principal de muerte entre mexicanos, esta enfermedad ha aumentado el número de defunciones más de treinta veces durante los últimos 50 años. De tan solo 1500 muertes reportadas en 1955 para el año 2000 ya había incrementado a más de 47 mil muertes. Actualmente INEGI en su comunicado de prensa del 26 de enero de 2024, reportó que la diabetes fue catalogada como la segunda causa de muerte principal en mexicanos en el período enero-junio del año inmediato anterior esto con un total de 55,885 defunciones. \cite{hernandez2013, inegi2024}.

Ahora bien, la diabetes es una enfermedad metabólica crónica y actualmente no cuenta con cura. Generalmente la diabetes se presenta cuando el cuerpo no produce las cantidades necesarias de insulina\footnote{Hormona segregada por los islotes de Langerhans en el páncreas, que regula la cantidad de glucosa existente en la sangre \cite{RAE23}.} es por esto que la producción de insulina se ha vuelto indispensable en la industria farmacéutica.

Tanto en la producción y su posterior distribución, la insulina requiere de condiciones térmicas específicas para conservar su eficacia y estabilidad. La temperatura es la condición vital en este fármaco, para así asegurar altos estándares de calidad. La temperatura de refrigeración ideal de la insulina es de 2 a 8 grados Celsius. Es decir, la temperatura de conservación es muy específica puesto que la insulina es de los medicamentos más sensibles a temperaturas elevadas además tampoco es recomendable llegar al punto de congelación.

La Ciudad de México lidera el consumo de insulina a nivel nacional. A través de su plataforma digital Data México la Secretaría de Economía (SE) reportó que la entidad capitalina del país mexicano acumuló un total de \$4.39 millones de dólares en compras internacionales de insulina \cite{datamex}.

En este sentido, la cadena de frío (refrigeración), es una de las actividades, si no la más importante, que los centros de almacenamiento y distribución, en este caso hospitales y/o clínicas públicas y privadas  deben realizar para garantizar la seguridad, calidad y eficacia de la insulina, con el fin de proteger a la población enferma.

Un aspecto a tener en cuenta es el aumento progresivo de la temperatura media ambiental en los últimos años,  producto de la alta contaminación y degradación del  medio ambiente, lo cual puede provocar roturas en las cadenas de frío de la insulina, lo que conlleva a disminuir su eficacia al momento de la aplicación. 

La clínica 40 perteneciente al Instituto Mexicano del Seguro Social (IMSS), es una Unidad de Medicina Familiar (UMF) ubicada en la alcaldía Azcapotzalco, la cual es muy importante para la Ciudad de México además de la misma alcaldía a la que pertenece. A dicha UMF asisten tanto habitantes de Azcapotzalco así como de barrios y colonias aledañas a la aplicación de insulina como parte de la terapia que estos llevan para controlar la diabetes.

Como se ha mencionado el centro de nuestro país es un punto clave y primordial para el almacenamiento y la conservación de la insulina, con la finalidad de controlar y tratar la diabetes en ciudadanos que sufren esta enfermedad, tanto en la zona metropolitana como las entidades federativas colindantes.

Considerando la alta demanda y la limitada capacidad de almacenamiento de insulina en la UMF 40 de la alcaldía Azcapotzalco, Ciudad de México, se propone el cálculo y diseño de una cámara frigorífica para almacenar este medicamento en condiciones térmicas óptimas. Este diseño permitirá garantizar el acceso a la insulina en condiciones ideales de temperatura\footnote{En 2021 se realizó un estudio por el Programa de Investigación en Cambio Climático (PINCC) el año 2021 se registró como el sexto año más caluroso a nivel global y el cuarto para México del que se tenga registro.} así como de seguridad, asegurando mantener la eficacia con la que ha sido diseñada. 


\addcontentsline{toc}{chapter}{Objetivo general}
\section*{Objetivo general}
Diseñar y calcular una cámara de refrigeración para la conservación de insulina ubicada en la Ciudad de México. 

\section*{Objetivos específicos}
\addcontentsline{toc}{section}{Objetivos específicos}

\begin{itemize}
	\item	Calcular la potencia, capacidad y carga térmica del sistema. 
	\item	Selección de material aislante térmico bajo las especificaciones obtenidas de la cámara.
	\item	Seleccionar los elementos térmicos para el funcionamiento del sistema frigorífico.
	\item	Diseñar el sistema eléctrico de la cámara de refrigeración. 
	\item	Determinar la capacidad de almacenamiento en función del espacio disponible en la clínica 40 de Azcapotzalco.
	\item	Generar una cámara de dimensiones óptimas comparadas a las del mercado.
	
\end{itemize}

\newpage
\section*{Delimitación}
 \addcontentsline{toc}{chapter}{Delimitación}
 
El presente proyecto se centra en el diseño térmico de una cámara frigorífica en específico para la conservación de la hormona de la insulina para la clínica 40 situada en la colonia Santa Barbara perteneciente a la alcaldía Azcapotzalco.

En este contexto, el proyecto abordará: cálculos del diseño, selección de elementos térmicos, capacidad térmica y disposición de los componentes de enfriamiento, sin dejar de lado la selección de materiales adecuados para sistemas de refrigeración tales como equipo de iluminación, puntos de acceso a la cámara, formas de almacenamiento etcétera. 

Cabe recalcar que la base geográfica ya definida, ayudará a determinar en forma precisa las condiciones climáticas locales, formas de traslado hacia y fuera de la cámara desde un punto de ubicación cercana a la clínica 40, con la simple finalidad de garantizar seguridad junto a la eficacia del fármaco. 





\newpage
\section*{Justificación}

\addcontentsline{toc}{chapter}{Justificación}

El proyecto en cuestión es de gran importancia para apoyar al sistema farmacéutico público del país mexicano a la conservación de la eficacia de la insulina, en específico  servirá a ciudadanos de la Ciudad de México o puntualmente a vecinos de la Unidad Médica Familiar 40 ubicada en la alcaldía Azcapotzalco.

Dado que los pacientes de \textit{diabetes mellitus} en México han ido incrementando de forma abismal y desafortunadamente la cifra de muertes también aumenta año tras año, se espera tener un impacto alto en la mejora de la calidad de vida de los pacientes con acceso a la insulina almacenada en la cámara de refrigeración a trabajar, con la finalidad de aportar un esbozo de esfuerzo a la reducción de las cifras ya mencionadas.

Para la estación de verano que es la época más caliente para la Ciudad de 
México así como para el resto del país, es cuando se espera obtener los mejores resultados en la preservación de la eficacia de la insulina, debido a su sensibilidad por las temperaturas elevadas. Y en su contraparte se espera que para la época de invierno también esta eficacia se mantenga estable por el sistema de enfriamiento controlado, por lo que está de más mencionar que estos resultados se espera que sean muy similares a lo largo del año. 



\section*{Justificación social}
\addcontentsline{toc}{section}{Justificación social}

El presente proyecto beneficiará a la clínica 40 que a su vez ayudará a otros centros médicos de la zona a disminuir su aforo en aplicación de insulina obteniendo así una distribución de la población adecuada al personal con el que cada centro cuenta. Así se garantiza un mejor nivel de vida en las personas que sufren diabetes, además de servir de apoyo para mitigar en gran medida la pérdida de eficiencia de la hormona de la insulina, pero que a su vez ayude a una distribución de calidad y oportuna del medicamento con la finalidad de disminuir el número de defunciones que presenta nuestro país al fin de un año.

 
\section*{Justificación práctica}
Gracias al uso de un equipo diseñado en condiciones específicas bajo las normas y leyes de diseño de equipo frigorífico vigentes, es posible diseñar una cámara capaz de conservar la insulina contribuyendo así también a la mejora de diseño de equipo médico dirigido al campo de la medicina.  

\newpage
\section*{Beneficios esperados}
\addcontentsline{toc}{section}{Beneficios esperados}
\begin{itemize}
	\item	Preservación de la eficiencia de la insulina: se garantiza un sistema térmico estable en un punto fijo (valor) de temperatura adecuado/determinado para la insulina. 
	\item 	Mejora en la calidad de vida de enfermos de diabetes: al usar la hormona de la insulina en condiciones óptimas se espera que el paciente mejore su salud.
	\item Mayor seguridad de medicación: la preservación de la insulina avala seguridad en el fármaco siendo así de ayuda a los beneficiados.
	
\end{itemize}



 
