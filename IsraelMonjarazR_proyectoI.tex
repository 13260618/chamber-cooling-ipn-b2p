\documentclass[11pt,a4paper]{report}
\usepackage[utf8]{inputenc}
\usepackage[spanish, es-noquoting]{babel} %... para poner . en equations
\decimalpoint




\usepackage[utf8]{inputenc}
\usepackage[spanish,es-noquoting]{babel} %es-noquo... y \decimal paa poner . en equations
\decimalpoint

\usepackage{amssymb}
\usepackage{amsmath}

\usepackage{tabularx}
\usepackage{amsfonts}
\usepackage{array}
\usepackage{graphicx}

\usepackage{multicol}
\usepackage{multirow}

\usepackage{makeidx} % para las tablas de contenidos índices etc


\usepackage{lscape}
\usepackage{float}
\usepackage{array}

\usepackage{lmodern} 
\usepackage{fancyhdr}

\fancypagestyle{plain}{%
	\fancyhf{} % Limpia todos los encabezados y pies de página anteriores
	\renewcommand{\headrulewidth}{0pt} % Sin línea de encabezado
	\renewcommand{\footrulewidth}{0pt} % Sin línea de pie de página
	\fancyfoot[c]{\thepage} % Números de página en la posición central
}



%  usando arial en el trabajo
\usepackage[T1]{fontenc}
\usepackage{helvet}

\renewcommand*\familydefault{\sfdefault}
\usepackage{setspace} % interlineados
\usepackage{parskip}  % sangrias 
\setlength{\parindent}{0pt} %sangria párrafos
\onehalfspacing % interlineado total

\usepackage{sectsty}
\sectionfont{\fontsize{12}{15}\selectfont} % Tamaño 12 y espaciado 15
\subsectionfont{\fontsize{12}{15}\selectfont} % Tamaño 12 y espaciado 15

\usepackage{longtable}
\usepackage{booktabs}
\usepackage{ragged2e} 
\usepackage{pdfpages}
\usepackage[hidelinks]{hyperref} % hidelinks para quitar los bordes de los enlaces


%\usepackage{xcolor}
 \usepackage{listings}
\usepackage{enumitem}
\usepackage{tikz}
\usetikzlibrary{decorations.markings}

\usetikzlibrary{shapes,arrows}
\usetikzlibrary{shapes.geometric, arrows}

\usetikzlibrary{mindmap, trees}
\usetikzlibrary{fadings}
\usetikzlibrary{patterns} %pared compuesta 

\usetikzlibrary{shapes.geometric, arrows.meta, positioning}
\tikzstyle{block} = [rectangle, draw, fill=blue!20, text width=2.5cm, text centered, minimum height=1.5cm, rounded corners]
\tikzstyle{arrow} = [draw, -{Stealth[scale=1.5]}, thick]
\usepackage{pgfplots} %graphs // temperature graph chap5

\usepackage{fontawesome} % Para íconos
%\usepackage{fontawesome5}
%\usetikzlibrary{positioning}



%\usepackage{natbib}

\usepackage{titlesec}

\titleformat{\chapter}[display]
{\normalfont\huge\bfseries}{\chaptertitlename\ \thechapter}{11}{\Huge}
% Establecer el espaciado antes y después de los capítulos
\titlespacing*{\chapter}{0pt}{-40pt}{15pt} % Ajusta el último valor para cambiar el espacio después del título

\usepackage{caption} 


\usepackage{tabularx}
\usepackage{array}
\usepackage{multirow}



\usepackage{etoolbox} % Para modificar la numeración de página
\usepackage{tocbibind}  %numeracion dif


%tabla productos
\usepackage[normalem]{ulem}



 % Configura el punto como separador decimal
\usepackage{siunitx}
\sisetup{output-decimal-marker = {.}}


\usepackage{multirow}
%\usepackage[table,xcdraw]{xcolor}







\usepackage[top=2.5cm, bottom=2.5cm, right=2.5cm, left=3cm]{geometry}

\usepackage{apacite}

\newcommand{\rsp}{\vspace{-0.5cm}}
\newcommand{\rspitems}{\vspace{-0.25cm}}

\newcommand{\degree}{^\circ}

% \setcounter{page}{40}
\usepackage{chngcntr} % Para cambiar los contadores
\usepackage{pifont} % checks
\begin{document} 
	
	%	\setlength{\cftdotsep}{1.5pt}  % Ajustar la separación de los puntos
	% \renewcommand{\cfteqnleader}{\dotfill}  % Agregar puntos suspensivos
	
	

	\sloppy
	
	\setcounter{tocdepth}{4} %4 indices
	
	%\setstretch{1.5}
	
	\numberwithin{equation}{chapter} % Numeración de ecuaciones vinculada al capítulo
	\numberwithin{figure}{chapter} % Numeración de figuras vinculada al capítulo
	\numberwithin{table}{chapter} % Numeración de tablas vinculada al capítulo
	
	
	
	% diagramas con tikz
	
	
	\tikzstyle{block} = [rectangle, draw, fill=blue!20, 
	text width=5em, text centered, rounded corners, minimum height=4em]
	\tikzstyle{line} = [draw, -latex']
	
	
	
	
	\includepdf[pages=-]{portada}

	
%\pagenumbering{roman}

	\pagenumbering{roman}

\tableofcontents
\listoftables
\listoffigures 
 
\section*{Índice de Ecuaciones}
 
 \addcontentsline{toc}{chapter}{Índice de Ecuaciones}  % Agregar la sección al índice general
 
 % Listar todas las ecuaciones en este índice, incluyendo el número de página con \pageref
 \begin{enumerate}
 	\item Ecuación \ref{eq:Calor-de-enfriamiento}: Calor de enfriamiento \dotfill \pageref{eq:Calor-de-enfriamiento}
 	\item Ecuación \ref{eq:energia_especifica}: Energía específica de un cuerpo \dotfill \pageref{eq:energia_especifica}
 	\item Ecuación \ref{eq:carga_iluminacion_descarga}: Carga térmica por iluminación \dotfill \pageref{eq:carga_iluminacion_descarga}
 \end{enumerate}


\newpage


\section*{Planteamiento del problema}
\addcontentsline{toc}{chapter}{Planteamiento del problema}


La \textit{diabetes mellitus}\footnote{Enfermedad metabólica producida por deficiencias en la cantidad o en la utilización de la insulina, lo que produce un exceso de glucosa en la sangre \cite{RAE01} } representa sin lugar a dudas uno de los retos más importantes como causa principal de muerte entre mexicanos, esta enfermedad ha aumentado el número de defunciones más de treinta veces durante los últimos 50 años. De tan solo 1500 muertes reportadas en 1955 para el año 2000 ya había incrementado a más de 47 mil muertes. Actualmente INEGI en su comunicado de prensa del 26 de enero de 2024, reportó que la diabetes fue catalogada como la segunda causa de muerte principal en mexicanos en el período enero-junio del año inmediato anterior esto con un total de 55,885 defunciones. \cite{hernandez2013, inegi2024}.

Ahora bien, la diabetes es una enfermedad metabólica crónica y actualmente no cuenta con cura. Generalmente la diabetes se presenta cuando el cuerpo no produce las cantidades necesarias de insulina\footnote{Hormona segregada por los islotes de Langerhans en el páncreas, que regula la cantidad de glucosa existente en la sangre \cite{RAE23}.} es por esto que la producción de insulina se ha vuelto indispensable en la industria farmacéutica.

Tanto en la producción y su posterior distribución, la insulina requiere de condiciones térmicas específicas para conservar su eficacia y estabilidad. La temperatura es la condición vital en este fármaco, para así asegurar altos estándares de calidad. La temperatura de refrigeración ideal de la insulina es de 2 a 8 grados Celsius. Es decir, la temperatura de conservación es muy específica puesto que la insulina es de los medicamentos más sensibles a temperaturas elevadas además tampoco es recomendable llegar al punto de congelación.

La Ciudad de México lidera el consumo de insulina a nivel nacional. A través de su plataforma digital Data México la Secretaría de Economía (SE) reportó que la entidad capitalina del país mexicano acumuló un total de \$4.39 millones de dólares en compras internacionales de insulina \cite{datamex}.

En este sentido, la cadena de frío (refrigeración), es una de las actividades, si no la más importante, que los centros de almacenamiento y distribución, en este caso hospitales y/o clínicas públicas y privadas  deben realizar para garantizar la seguridad, calidad y eficacia de la insulina, con el fin de proteger a la población enferma.

Un aspecto a tener en cuenta es el aumento progresivo de la temperatura media ambiental en los últimos años,  producto de la alta contaminación y degradación del  medio ambiente, lo cual puede provocar roturas en las cadenas de frío de la insulina, lo que conlleva a disminuir su eficacia al momento de la aplicación. 

La clínica 40 perteneciente al Instituto Mexicano del Seguro Social (IMSS), es una Unidad de Medicina Familiar (UMF) ubicada en la alcaldía Azcapotzalco, la cual es muy importante para la Ciudad de México además de la misma alcaldía a la que pertenece. A dicha UMF asisten tanto habitantes de Azcapotzalco así como de barrios y colonias aledañas a la aplicación de insulina como parte de la terapia que estos llevan para controlar la diabetes.

Como se ha mencionado el centro de nuestro país es un punto clave y primordial para el almacenamiento y la conservación de la insulina, con la finalidad de controlar y tratar la diabetes en ciudadanos que sufren esta enfermedad, tanto en la zona metropolitana como las entidades federativas colindantes.

Considerando la alta demanda y la limitada capacidad de almacenamiento de insulina en la UMF 40 de la alcaldía Azcapotzalco, Ciudad de México, se propone el cálculo y diseño de una cámara frigorífica para almacenar este medicamento en condiciones térmicas óptimas. Este diseño permitirá garantizar el acceso a la insulina en condiciones ideales de temperatura\footnote{En 2021 se realizó un estudio por el Programa de Investigación en Cambio Climático (PINCC) el año 2021 se registró como el sexto año más caluroso a nivel global y el cuarto para México del que se tenga registro.} así como de seguridad, asegurando mantener la eficacia con la que ha sido diseñada. 


\addcontentsline{toc}{chapter}{Objetivo general}
\section*{Objetivo general}
Diseñar y calcular una cámara de refrigeración para la conservación de insulina ubicada en la Ciudad de México. 

\section*{Objetivos específicos}
\addcontentsline{toc}{section}{Objetivos específicos}

\begin{itemize}
	\item	Calcular la potencia, capacidad y carga térmica del sistema. 
	\item	Selección de material aislante térmico bajo las especificaciones obtenidas de la cámara.
	\item	Seleccionar los elementos térmicos para el funcionamiento del sistema frigorífico.
	\item	Diseñar el sistema eléctrico de la cámara de refrigeración. 
	\item	Determinar la capacidad de almacenamiento en función del espacio disponible en la clínica 40 de Azcapotzalco.
	\item	Generar una cámara de dimensiones óptimas comparadas a las del mercado.
	
\end{itemize}

\newpage
\section*{Delimitación}
 \addcontentsline{toc}{chapter}{Delimitación}
 
El presente proyecto se centra en el diseño térmico de una cámara frigorífica en específico para la conservación de la hormona de la insulina para la clínica 40 situada en la colonia Santa Barbara perteneciente a la alcaldía Azcapotzalco.

En este contexto, el proyecto abordará: cálculos del diseño, selección de elementos térmicos, capacidad térmica y disposición de los componentes de enfriamiento, sin dejar de lado la selección de materiales adecuados para sistemas de refrigeración tales como equipo de iluminación, puntos de acceso a la cámara, formas de almacenamiento etcétera. 

Cabe recalcar que la base geográfica ya definida, ayudará a determinar en forma precisa las condiciones climáticas locales, formas de traslado hacia y fuera de la cámara desde un punto de ubicación cercana a la clínica 40, con la simple finalidad de garantizar seguridad junto a la eficacia del fármaco. 





\newpage
\section*{Justificación}

\addcontentsline{toc}{chapter}{Justificación}

El proyecto en cuestión es de gran importancia para apoyar al sistema farmacéutico público del país mexicano a la conservación de la eficacia de la insulina, en específico  servirá a ciudadanos de la Ciudad de México o puntualmente a vecinos de la Unidad Médica Familiar 40 ubicada en la alcaldía Azcapotzalco.

Dado que los pacientes de \textit{diabetes mellitus} en México han ido incrementando de forma abismal y desafortunadamente la cifra de muertes también aumenta año tras año, se espera tener un impacto alto en la mejora de la calidad de vida de los pacientes con acceso a la insulina almacenada en la cámara de refrigeración a trabajar, con la finalidad de aportar un esbozo de esfuerzo a la reducción de las cifras ya mencionadas.

Para la estación de verano que es la época más caliente para la Ciudad de 
México así como para el resto del país, es cuando se espera obtener los mejores resultados en la preservación de la eficacia de la insulina, debido a su sensibilidad por las temperaturas elevadas. Y en su contraparte se espera que para la época de invierno también esta eficacia se mantenga estable por el sistema de enfriamiento controlado, por lo que está de más mencionar que estos resultados se espera que sean muy similares a lo largo del año. 



\section*{Justificación social}
\addcontentsline{toc}{section}{Justificación social}

El presente proyecto beneficiará a la clínica 40 que a su vez ayudará a otros centros médicos de la zona a disminuir su aforo en aplicación de insulina obteniendo así una distribución de la población adecuada al personal con el que cada centro cuenta. Así se garantiza un mejor nivel de vida en las personas que sufren diabetes, además de servir de apoyo para mitigar en gran medida la pérdida de eficiencia de la hormona de la insulina, pero que a su vez ayude a una distribución de calidad y oportuna del medicamento con la finalidad de disminuir el número de defunciones que presenta nuestro país al fin de un año.

 
\section*{Justificación práctica}
Gracias al uso de un equipo diseñado en condiciones específicas bajo las normas y leyes de diseño de equipo frigorífico vigentes, es posible diseñar una cámara capaz de conservar la insulina contribuyendo así también a la mejora de diseño de equipo médico dirigido al campo de la medicina.  

\newpage
\section*{Beneficios esperados}
\addcontentsline{toc}{section}{Beneficios esperados}
\begin{itemize}
	\item	Preservación de la eficiencia de la insulina: se garantiza un sistema térmico estable en un punto fijo (valor) de temperatura adecuado/determinado para la insulina. 
	\item 	Mejora en la calidad de vida de enfermos de diabetes: al usar la hormona de la insulina en condiciones óptimas se espera que el paciente mejore su salud.
	\item Mayor seguridad de medicación: la preservación de la insulina avala seguridad en el fármaco siendo así de ayuda a los beneficiados.
	
\end{itemize}



 

	 
 \newpage
%\clearpage
\pagenumbering{arabic}
%\setcounter{page}{13}
 
 
%\addcontentsline{toc}{chapter}{ \hfill 12}
 
 \addtocontents{toc}{\protect\contentsline{chapter}{CAPÍTULO I. Generalidades del diseño térmico y cámaras de refrigeración de insulina \hfill 12}{}{}}
 


\begin{titlepage}
 
	
	
	\centering
	\begin{tikzpicture}%opacity=0.5
		\node[inner sep=0pt, ] (image) at (0,0) {\includegraphics[width=\textwidth]{figures/bg-cap1}};
		\fill [white,path fading=south] (-5,-4) rectangle (5,4);
		\node[black,font=\Huge\bfseries] at (0,3) {Capítulo I};
		\node[black,font=\Huge\bfseries] at (0,2) {Antecedentes históricos};
				
		\node[black,font=\LARGE\bfseries] at (0,0.8) {Generalidades del diseño térmico y};
		\node[black,font=\LARGE\bfseries] at (0,-0.4) {cámaras de refrigeración de insulina};
	\end{tikzpicture}
\end{titlepage}

 

\newpage 

\section*{Introducción}


\addcontentsline{toc}{section}{{Introducción}}\rsp
\setcounter{chapter}{1}
 \setcounter{page}{13}


La refrigeración no es más que el proceso en el que se elimina el calor contenido en un espacio cerrado para excluirlo a una temperatura más alta, es decir técnicamente se está trasladando calor de una temperatura relativamente baja a una más alta \cite{Hundy1984}.

En países industrializados la refrigeración tradicionalmente se ha empleado para preservar alimentos a temperaturas bajas, lo que impide la proliferación de bacterias, levaduras y moho que pueden causar su deterioro. Esto posibilita la congelación de muchos productos perecederos, lo que permite su conservación durante largos periodos, incluso meses o años, sin una considerable pérdida de valor nutricional, sabor o alteración en su apariencia \cite{britannica2023}.

De acuerdo con \citeauthor{khurana2019}, \citeyear{khurana2019}, la insulina es una hormona endógena que se produce de manera natural por el páncreas. La función principal radica en facilitar la absorción y utilización de la glucosa por parte de las células del cuerpo. Esta glucosa sirve como fuente de energía para las diversas funciones celulares en el cuerpo humano. Las personas con \textit{diabetes mellitus} (DM) experimentan una disminución en la capacidad para absorber y utilizar la glucosa en la sangre, lo que resulta en un aumento de los niveles de glucosa en la misma. En el caso de la diabetes tipo 1, el páncreas no produce suficiente insulina, lo que requiere tratamiento con terapia de insulina. Por otro lado, en la diabetes tipo 2, los pacientes producen insulina, pero las células de su cuerpo no responden adecuadamente a ella, lo que se conoce como resistencia a la insulina. Aunque la insulina se utiliza comúnmente en el tratamiento de la diabetes tipo 1, también puede ser recetada para pacientes con diabetes tipo 2 para superar la resistencia a la insulina. 

\newpage

\section{Las cámaras de refrigeración a través de la historia}

La práctica de la refrigeración, a menudo asociada con tecnología moderna, tiene sus raíces en tiempos antiguos, donde la preservación de alimentos era crucial. Desde la antigüedad, los seres humanos han recurrido a métodos naturales, como almacenar alimentos en cuevas frescas o utilizando hielo de montañas para mantenerlos frescos, vea la figura \ref{fig:pozo-nieve} de un ejemplo situado la Mancomunidad Turística de Sierra Espuña. Este enfoque permitía tener reservas alimenticias disponibles en tiempos de necesidad. Además, desde el siglo XVI, se ha empleado la técnica de mezclar hielo con sal para lograr temperaturas por debajo de su punto de fusión, destacando la continua búsqueda de soluciones para la conservación de alimentos en condiciones adversas. (UPV, 2020)

\begin{figure}[H]
	\centering
	\includegraphics[width=0.6\linewidth]{figures/pozo-nieve}
	\caption{Esquema de las partes principales de un pozo de nieve}
		Fuente: \cite{ecoproyecta2024}
	\label{fig:pozo-nieve}
\end{figure}

\subsection{Época antigua}

Durante siglos, se ha tenido conocimiento de que la evaporación del agua genera una sensación de frescor. En épocas antiguas, aunque no se buscaba comprender el fenómeno, se observaba que cualquier parte del cuerpo mojada se enfriaba al secarse al aire. Ya en el siglo II, en Egipto se empleaba la evaporación para enfriar jarras de agua, mientras que en la antigua India se utilizaba para la producción de hielo (Neuberger, 1930).


\subsection{Grecia y el imperio romano}
En la antigua Grecia e Imperio Romano, se desarrollaron ingeniosos métodos de refrigeración que se basaban en el aprovechamiento de la nieve y el hielo. Griegos y romanos contaban con una comprensión rudimentaria de cómo preservar alimentos en climas cálidos, utilizando nieve de montañas y áreas frías. Uno de estos métodos implicaba cavar hoyos en la tierra y forrarlos con aislantes naturales como paja y ramas, donde se almacenaba cuidadosamente nieve o hielo recolectados en invierno, manteniendo una temperatura baja y conservando alimentos perecederos como carnes y pescados durante las estaciones más calurosas del año.

Esta práctica se difundió por el Mediterráneo y otras regiones donde las altas temperaturas planteaban desafíos para la conservación de alimentos, convirtiéndose en una técnica esencial para comunidades rurales y empleada ampliamente hasta el siglo XX  \cite{jr2013}.

\subsection{Época media}
Durante la Edad Media, surgieron técnicas innovadoras de refrigeración en la India del siglo IV y en la Península Ibérica bajo el gobierno musulmán, donde se utilizaban procesos químicos como la disolución de nitrato sódico y nitrato de potasio en agua para bajar la temperatura, representando un avance significativo en la tecnología de refrigeración. En el siglo XVI, Blas Villafranca, un médico español en Roma, experimentó con la refrigeración de líquidos como agua y vino mediante mezclas refrigerantes. Sin embargo, el descubrimiento más impactante ocurrió en 1607, cuando se descubrió que una combinación de agua y sal tenía la capacidad de congelar el agua, revolucionando la producción de hielo artificial y transformando la conservación de alimentos y la refrigeración en una época donde las altas temperaturas presentaban desafíos continuos.

Estos avances en la refrigeración artificial durante la Edad Media marcaron un hito en la evolución de la tecnología de conservación de alimentos y bebidas en condiciones climáticas adversas \cite{bernad2023}.

\subsection{Época contemporánea}

Según estudios reportados de Goosman en el año 1924, los primeros intentos de crear refrigeración mecánica se basaron en los efectos refrigerantes de la evaporación del agua. En 1755, William Cullen, un médico escocés, logró obtener temperaturas lo suficientemente bajas como para producir hielo. Esto lo logró al reducir la presión del agua en un recipiente sellado mediante una bomba de aire. Bajo una presión muy pequeña, el agua se evaporaba o hervía a una temperatura baja, extrayendo calor del resto del agua y provocando la formación de hielo. Desde Cullen, numerosos ingenieros y científicos han creado una variedad de inventos para comprender los principios básicos de la refrigeración mecánica \cite{dincer2010}.

Según indican \cite{critchell1912}, en el año 1834, Jacob Perkins, un ciudadano estadounidense viviendo en Inglaterra, diseñó y obtuvo la patente de una máquina que empleaba compresión de vapor, que incluía un compresor, un condensador, un evaporador y un grifo situado entre el condensador y el evaporador. A continuación en la figura \ref{fig:aparato-jacob} se muestra el aparato diseñado por Perkins en donde se aprecia que el refrigerante (éter u otro fluido volátil) hierve en el evaporador B tomando calor del agua circundante en el recipiente A. La bomba C extrae el vapor y lo comprime a una presión más alta a la que puede condensarse a líquidos en los tubos D, cediendo calor al agua en el recipiente E. El líquido condensado fluye a través de la válvula cargada de peso H, que mantiene la diferencia de presión entre el condensador y el evaporador. La pequeña bomba situada encima de H se utiliza para cargar el aparato con refrigerante.

\begin{figure}[H]
	\centering
	\includegraphics[width=0.6\linewidth]{figures/aparato-jacob}
	\caption{Aparato de Jacob Perkins en la especificación de su patente de 1834}
	Fuente: \cite{vidyaputra}.
	\label{fig:aparato-jacob}
\end{figure}

 
Durante las tres décadas posteriores a 1850, la creciente demanda impulsó una ola de innovación y progreso. Científicos destacados como Faraday y Thilorier introdujeron nuevas sustancias, como el amoníaco y el dióxido de carbono, que demostraron ser más eficaces que el agua y el éter, incluso pudiendo ser licuadas. Este avance fue respaldado por un sólido fundamento teórico proporcionado por Rumford y Davy, quienes habían esclarecido la naturaleza del calor, y por científicos como Kelvin, Joule y Rankine, quienes ampliaron el trabajo iniciado por Sadi Carnot en la formulación de la ciencia de la termodinámica, sentando así las bases para la refrigeración mecánica.

\section{El origen de la refrigeración en China y su desarrollo en occidente.}
La historia de la refrigeración se remonta a la antigua China, donde se descubrió por primera vez el potencial de la salmuera para enfriar alimentos. En el siglo XIV, Marco Polo trajo consigo desde China a Occidente el conocimiento sobre la fabricación de sorbetes de leche, que se cree se basaba en el principio de la evaporación de la salmuera. Los chinos, que utilizaban la salmuera para preservar alimentos, probablemente observaron sus propiedades refrigerantes \cite{curiosfera2023}. 

En torno al año 1660, el italiano Zimara propuso utilizar una combinación de nieve y salitre como agente refrigerante. Posteriormente, de forma empírica, se observó que la rápida evaporación de la salmuera caliente provocaba la absorción de calor. Este proceso, influenciado por las ideas de Zimara y por el uso de alcarrazas turcas (recipientes de barro poroso que mantenían el agua fresca mediante la evaporación), sentó las bases en el siglo XVII para los primeros intentos de refrigeración controlada en Occidente. Así, la historia de la refrigeración comenzó en China y gradualmente se difundió por todo el mundo, influyendo en el desarrollo de técnicas y tecnologías de conservación de alimentos inicialmente, hoy en día estas aplicaciones son extensas (Atecyr, 2020)

\section{Historia de la insulina y su preservación}
El 12 de diciembre de 1921 Banting y Best descubrieron la insulina, que nació como una posible esperanza de cura. Al año siguiente, Leonard Thompson, un niño de 14 años con diabetes severa, fue el primer paciente al que se le aplicó una inyección de extracto pancreático vacuno (Vecchio et.al, 2018).

De acuerdo a Buse JB, Davies MJ, Frier BM, et al 100 years on: the impact of the discovery of insulin  (2021), hasta comienzos del siglo pasado la diabetes representaba un trastorno devastador, especialmente cuando se diagnosticaba en la infancia, ya que solía resultar mortal. Por lo tanto, el exitoso descubrimiento y extracción de insulina a partir del páncreas en 1921 marcó un avance maravilloso que transformó radicalmente la vida de quienes padecían esta enfermedad. En la figura \ref{fig:nino1diabetes} se aprecia a un niño el cual era un paciente de diabetes tipo 1, antes de ser medicado de insulina y obtener los beneficios de la hormona dejando de lado la rigurosa ingesta de calorías en los centros médicos de aquellos tiempos.

\begin{figure}[H]
	\centering
	\includegraphics[width=0.6\linewidth]{figures/niño1diabetes}
	\caption{Fotos de antes y después de un niño con diabetes tipo 1.}
	Fuente: \cite{novo}
	\label{fig:nino1diabetes}
\end{figure}

\subsection{Medidas preventivas de la insulina}
La insulina, al ser una proteína, puede sufrir cambios y perder su eficacia. Es especialmente vulnerable a la precipitación causada por alteraciones en el pH y la temperatura. Estos cambios físico químicos pueden provocar su degradación tanto física (cambios en su estado físico sin alterar su estructura covalente) como química (alteraciones en su estructura covalente). Por ello, es crucial almacenar los viales de insulina sin abrir en un ambiente controlado entre 2 y 8°C, protegidos de la luz \cite{khurana2019}.

Los productos de insulina, ya sea en frascos o cartuchos proporcionados por los fabricantes, pueden mantenerse sin refrigeración a temperaturas entre 15 °C y 30 °C durante un máximo de 28 días y seguir siendo efectivos, tanto si están abiertos como si no. Sin embargo, cualquier insulina que haya sido modificada para diluirse o que se haya extraído del envase original del fabricante debe desecharse en un plazo de dos semanas. Cuanto más prolongada sea la exposición a estas temperaturas, menor erá su eficacia, lo que puede conducir a una pérdida de control de la glucosa en la sangre con el tiempo. 

\subsection{Preservación de la insulina}

Las primeras formas de almacenamiento y preservación de la insulina para mantener estable su capacidad y eficacia fueron a través de jeringas. Las primeras jeringas de vidrio fabricadas por Becton Dickinson eran pesadas y frágiles, con agujas largas y gruesas que requerían esterilización mediante ebullición antes de ser reutilizadas. En 1954, se introdujo la primera jeringa desechable de vidrio, seguida por una versión de plástico en 1955. Estas jeringas, estaban equipadas con agujas integradas o desechables hacían que las inyecciones fueran menos dolorosas, pero aún presentaban problemas de dosificación. La jeringa de insulina U100, hecha de un plástico especial con unidades marcadas, mejoró la precisión y permitió una reutilización segura. Avances posteriores incluyeron agujas de calibre más corto para minimizar el dolor, aunque persistieron problemas de precisión en la dosificación \cite{buse2021}.

\subsection{Refrigeración de la insulina. }

Como indica en su artículo Smith, J. (2020), la modernización de la refrigeración de la insulina ha sido un proceso gradual que ha ido evolucionando en paralelo con los avances tecnológicos y científicos. Inicialmente, la conservación de la insulina dependía de métodos básicos como el almacenamiento en lugares frescos o en recipientes con hielo. Sin embargo, a medida que avanzaba el tiempo, se produjeron mejoras significativas en este campo. Con el desarrollo de la tecnología en la década de 1920, se introdujeron los primeros refrigeradores domésticos, lo que permitió un almacenamiento más seguro en los hogares\footnote{A principios del siglo XIX, los inventores estadounidenses Oliver Evans, Jacob Perkins y John Gorrie desarrollaron las primeras versiones del frigorífico moderno}. 

El autor también describe que estos primeros modelos eran bastante simples y proporcionaban un ambiente fresco, pero carecían de un control preciso de la temperatura. A lo largo del siglo XX, con el avance tecnológico, se diseñaron refrigeradores más sofisticados con controles de temperatura precisos, específicamente adaptados para almacenar medicamentos sensibles como la insulina. En las últimas décadas, se han desarrollado refrigeradores portátiles y dispositivos de almacenamiento compactos, lo que ha proporcionado a los pacientes con diabetes una forma conveniente y segura de transportar su insulina. Estos dispositivos suelen estar equipados con tecnología avanzada de control de temperatura y funciones de monitoreo para garantizar la integridad del medicamento.

\subsection{Avances tecnológicos en la refrigeración de la insulina}

Durante la última década, ha habido avances significativos en la tecnología utilizada para tratar la diabetes. Hoy en día disponemos de dispositivos de infusión continua de insulina subcutánea o bombas de insulina que intentan imitar la secreción fisiológica del páncreas, así como de monitorización continua de la glucosa que proporciona información de la glucosa intersticial las 24 horas del día. Los modelos más modernos combinan ambas tecnologías en un solo dispositivo e incluso son capaces de detener la infusión de insulina en caso de hipoglucemia (Apablaza et al., 2016). 

De acuerdo a una modificación de la imagen dada por MedlinePlus, 2024, en Estados Unidos a través de su sitio web oficial, la imagen \ref{fig:bomba-insul} que se muestra a continuación en la cual se observa cómo la bomba suministra insulina de manera continua. En su interior, dispone de un depósito que contiene un análogo de insulina rápida (insulina ultrarrápida), el cual se conecta a un catéter encargado de transferirla al tejido subcutáneo mediante una cánula.


\begin{figure}[H]
	\centering
	\includegraphics[width=0.7\linewidth]{figures/bomba-insul}
	\caption{Bomba de insulina y set de infusión (A), (B) posición de la Cánula en el tejido Subcutáneo}
	Fuente:\cite{apablaza-2016}
	\label{fig:bomba-insul}
\end{figure}


 
 
	


\clearpage
%\counterwithin{figure}{chapter}

%\useunder{\uline}{\ul}{}


%addcontentsline{toc}{chapter}{CAPÍTULO II. Contexto del proyecto \hfill 20}{}

%\addtocontents{toc}{\protect\contentsline{section}{Introducción \hfill 34}{}}
\addtocontents{toc}{\protect\contentsline{chapter}{CAPÍTULO II. Contexto del proyecto \hfill 20}{}{}}




\begin{titlepage}
	
	
	\centering
	\begin{tikzpicture}%opacity=0.5
		\node[inner sep=0pt, ] (image) at (0,0) {\includegraphics[width=\textwidth]{figures/bg-cap2}};
		\fill [white,path fading=south] (-6,-4) rectangle (6,4);
		\node[black,font=\Huge\bfseries] at (0,3) {Capítulo II};
		\node[black,font=\Huge\bfseries] at (0,2) {Contexto del proyecto};		
		\node[black,font=\LARGE\bfseries] at (0,0.8) {\textit{Estudio del estado de}};
		\node[black,font=\LARGE\bfseries] at (0,-0.2) {\textit{la técnica del proyecto}};
	\end{tikzpicture}
\end{titlepage}




\newpage 

\section*{Introducción}
\addcontentsline{toc}{section}{{Introducción}}\rsp
\setcounter{chapter}{2}
\setcounter{section}{0}
\setcounter{figure}{0}

\setcounter{page}{21}

Es momento de abordar los diferentes contextos en los que nuestro proyecto se ve involucrado, dado que, como se ha manifestado en el capítulo anterior la insulina es un medicamento vital para millones de personas en todo el mundo que viven con diabetes.

Y en específico para habitantes en la zona centro del país, garantizar el almacenamiento seguro y eficaz de la insulina es fundamental para su salud y calidad de vida. Por lo tanto, el diseño de una cámara de refrigeración para la insulina tiene un impacto directo en la salud y bienestar de la población.

Se ha considerado que los avances tecnológicos en refrigeración, como sistemas de control de temperatura precisos y monitoreo remoto, pueden mejorar la eficiencia y confiabilidad de la cámara de refrigeración. Además, consideraremos la integración de tecnologías de seguimiento pueden proporcionar datos importantes sobre el almacenamiento y uso de la insulina.

Así mismo se recapacitado seriamente en el cumplir con las regulaciones y normativas relacionadas con el almacenamiento de medicamentos ya que es fundamental para garantizar la seguridad y eficacia de la insulina. Las cámaras de refrigeración deben cumplir con estándares específicos de la industria farmacéutica y de salud, así como con requisitos de seguridad y calidad establecidos por organismos reguladores.

Últimamente, comprender las características específicas de la insulina, como su sensibilidad a la temperatura y la luz, es esencial para el diseño de la cámara de refrigeración. Las condiciones de almacenamiento inadecuadas pueden comprometer la estabilidad y eficacia de la insulina, lo que podría tener graves consecuencias para los pacientes.






\newpage

\section{Contexto tecnológico}

Desde 2013, la industria del almacenamiento en cadenas de frío\footnote{La cadena de frío es un término aplicado a la manipulación y distribución de alimentos donde el producto se mantiene en condiciones de temperatura adecuadas desde la cosecha, pasando por el proceso de enfriamiento o congelación hasta el punto de venta \cite{Hundy1984}.}   ha crecido a un ritmo impresionante. El papel que juegan en las cadenas de suministro que se centran en el control de la temperatura continúa creciendo significativamente debido a los avances en los almacenes refrigerados y el almacenamiento en frío (Rithehite, s/f). 

\subsection{Cámaras frigoríficas comerciales}

La mayoría de las personas asocian probablemente el término "refrigeración" con un refrigerador doméstico fiable y fácil de usar que se compra como una sola unidad que incluye un refrigerador y un congelador. La situación es muy diferente en el sector industrial y comercial. Los refrigeradores y congeladores no solo suelen ser unidades totalmente separadas, sino que los componentes de un frigorífico (compresor, condensador, dispositivo de expansión y evaporador) no se compran necesariamente al mismo tiempo ni al mismo proveedor \cite{carson2013}. 

Entre 1850 y 1880 se desarrollaron aparatos de refrigeración, que se clasificaron en función del material (refrigerante). Las máquinas de aire comprimido o de aire frío son aparatos que utilizan aire como refrigerante y son importantes en la historia de la refrigeración. El Dr. estadounidense John Gorrie creó un dispositivo comercial práctico de aire frío, que patentó en Inglaterra en 1950 y en Estados Unidos en 1951 \cite{doi1952}.

En un sistema de refrigeración, la función principal es eliminar el calor de $"$otro medio de baja temperatura$"$ (la fuente de calor) ) y transferir este calor a otra situación de mayor temperatura (disipador de calor). El sistema refrigerado de la Figura \ref{fig:ciclo-termo} es un sistema termodinámico. ¿Qué hace este sistema? La carga de enfriamiento (efecto de enfriamiento) es el calor transferido desde la fuente. Por el contrario, la temperatura a la que el disipador recibe calor es alta. El trabajo comúnmente denotado por $W$ es responsable de ambos efectos y el sistema debe operar de acuerdo con la primera ley de la termodinámica\footnote{De acuerdo con \citeauthor{cengel1999}, \citeyear{cengel1999} la primera ley obedece al principio de la conservación de la energía, es decir, “el calor jamás fluye espontáneamente de un objeto frío a un objeto caliente”}  para mantener su funcionalidad.


\begin{figure}[H]
	\centering
	\includegraphics[width=0.7\linewidth]{figures/ciclo-termo}
	\caption{Ciclo termodinámico representando un refrigerador}
	\label{fig:ciclo-termo}
\end{figure}

\subsection{Tipos de cámaras frigoríficas usados en farmacéutica}



Basándose en el módulo III $"$Cadena de frío", del Curso de gerencia para el manejo efectivo del Programa Ampliado de Inmunización PAI, desarrollado por la Organización Panamericana de la Salud (OMPS) la cual es la Oficina Regional de Organización Mundial de la Salud en el año 2006. 

Se tiene que, para almacenar y conservar el medicamento del PAI se utilizan tres tipos de refrigeradores, 

\begin{enumerate}
	\item	Refrigerador por compresión eléctrica\\
	Se utiliza ampliamente para almacenar vacunas en establecimientos de salud con electricidad permanente (ver Figura \ref{fig:compresionelec})
	\begin{figure}[H]
		\centering
		\includegraphics[width=0.6\linewidth]{figures/compresionelec}
		\caption{Refrigerador de laboratorio por compresión eléctrica.}
		Fuente: \cite{refrigeradores-2013}
		\label{fig:compresionelec}
	\end{figure}
	\item Refrigerador por absorción\\
	Los refrigeradores (que funcionan con propano o queroseno) son ideales para áreas sin electricidad o donde los recursos eléctricos son limitados (ver Figura \ref{fig:refexper} ).
	
	
	\begin{figure}[H]
		\centering
		\includegraphics[width=0.6\linewidth]{figures/refexper}
		\caption{Refrigerador experimental difusión-absorción.}
		Fuente:  \cite{belamn-2015}
		\label{fig:refexper}
	\end{figure}
	
	\item Refrigerador fotovoltaico (energía solar)
	
	
	Los dispositivos solares son ideales para almacenar y conservar vacunas o medicamentos en áreas de difícil acceso, especialmente en áreas donde los recursos energéticos convencionales no están disponibles o son difíciles de obtener. Funciona con energía solar que se almacena en una batería para alimentar el refrigerador (ver Figura \ref{fig:refsolar}.
	
	
	\begin{figure}[H]
		\centering
		\includegraphics[width=0.6\linewidth]{figures/refsolar}
		\caption{Solarchill, el refrigerador Solar que salva vidas.}
		Fuente: \cite{ecoinventos-2022}
		\label{fig:refsolar}
	\end{figure}
	\item Equipos frigoríficos de pared de hielo (ice-lined refrigerators)\\
	
	Las unidades de enfriamiento de pared de hielo consisten en tubos o bolsas de enfriamiento llenos de líquido dispuestos alrededor de las paredes interiores del gabinete (Figura \ref{fig:reficeline}). Lo más importante es que tarda al menos 48 horas (+8 ºC) en calentarse en caso de corte de luz.
	
	\begin{figure}[H]
		\centering
		\includegraphics[width=0.6\linewidth]{figures/reficeline}
		\caption{Refrigerador con Revestimiento de Hielo Vertical.}
		Fuente: \cite{hunuo-technology}
		\label{fig:reficeline}
	\end{figure}
	
	
	\section{Contexto normativo}
	
	En un proyecto de diseño de ingeniería como el de la magnitud del presente (diseño de una cámara de refrigeración), es crucial que se conozcan las normas vigentes oficiales relacionadas con este tipo de proyectos. Esto se debe a que las normativas establecen estándares específicos que regulan diversos aspectos del diseño, construcción y operación de las cámaras de refrigeración.
	
	Por un lado las normas vigentes abordan aspectos relacionados con la seguridad, tanto para el personal que trabaja con la cámara como para los productos que se almacenan en ella. En tanto así, las normas pueden incluir requisitos para la instalación de sistemas de alarma, de ventilación y procedimientos de emergencia en caso de fallos en el equipo. Además deben incluir requisitos de temperatura específicos para diferentes tipos de productos, así como pautas para el monitoreo y registro de las condiciones del almacenamiento.
	
	Así mismo, las normas deben abordar aspectos relacionados con la eficiencia energética y el impacto ambiental de la cámara de refrigeración.
		
	\subsection{Normas Oficiales Mexicanas (NOM)}
	
		Las Normas Oficiales Mexicanas (NOM) son leyes técnicas de obligado cumplimiento emitidas por las autoridades competentes, cuyo objetivo es establecer las condiciones que deben cumplirse si un proceso o servicio representa un riesgo para la seguridad humana, o pone en peligro la salud humana. También contiene información sobre los términos y referencias para el cumplimiento y aplicación de estos términos \cite{salud-2022}.
		
		
		\begin{itemize}
			\item NOM-12-ENER-2019, Eficiencia energética de unidades condensadoras y evaporadoras para refrigeración. Límites, métodos de prueba y etiquetado.\\
			Esta Norma Oficial Mexicana establece los requisitos de eficiencia energética que deben cumplir las unidades condensadoras y evaporadoras, así como los métodos de prueba para verificar su cumplimiento, el etiquetado y el procedimiento para evaluar la conformidad de los productos.
			\begin{enumerate}
				\item Unidades condensadoras para refrigeración, que son fabricadas para su instalación al aire libre o en interiores con potencia frigorífica, mayor o igual que 746 W (2 547 BTU/h) y menor que 26 000 W (88 716 BTU/h) en media temperatura, y menor que 9 500 W (32 415 BTU/h) en baja temperatura.
				\item Unidades evaporadoras para refrigeración de bajo perfil que son destinadas para operar con un refrigerante y alimentados por expansión directa en condiciones húmedas y/o secas con capacidades nominales de enfriamiento, mayor o igual que 300 W (1 023 BTU/h) y menor que 40 000 W (136 482 BTU/h) en media temperatura, y menor que 13 000 W (44 397 BTU/h) en baja temperatura \cite{dof-2010}.
			\end{enumerate}
			
			\item  	NOM-008-SCFI-2002, Sistema General de Unidades de Medida. 
			
			Esta Norma Mexicana establece las definiciones, símbolos y reglas de notación para las unidades del Sistema Internacional de Unidades (SI) y otras unidades fuera de este sistema reconocidas por la CGPM, que en conjunto constituyen un sistema general de unidades de medida utilizadas en diversos campos en ciencia, tecnología, industria, educación y comercio \cite{dof-2010-nom008}.
			
			\item NOM-015-SSA2-2010, Para la prevención, tratamiento y control de la diabetes mellitus.
			
			Esta Norma Oficial Mexicana tiene por objeto establecer los procedimientos para la prevención, tratamiento, control de la diabetes y la prevención médica de sus complicaciones.
			También esta norma es de observancia obligatoria en el territorio nacional para los establecimientos y profesionales de la salud de los sectores público, social y privado que presten servicios de atención a la diabetes en el Sistema Nacional de Salud \cite{dof-2010-nom015}.
			
			\item NOM-028-STPS-2012, Sistema para la administración del trabajo-Seguridad en los procesos y equipos críticos que manejen sustancias químicas peligrosas.
			
			Establecer los elementos de un sistema de administración para organizar la seguridad en los procesos y equipos críticos que manejen sustancias químicas peligrosas, a fin de prevenir accidentes mayores y proteger de daños a las personas, a los centros de trabajo y a su entorno.
			
			\item	NOM-176-SSA1-1998, Validación de proveedores de fármacos y materias primas para la elaboración de medicamentos de uso humano.
			
			Esta Norma Oficial Mexicana establece los requisitos sanitarios que deben reunirse para la aprobación de proveedores de fármacos y materias primas de fabricación nacional o extranjera, utilizadas para la elaboración de medicamentos de uso humano.
			
			\item NOM-197-SSA1-2000, Que establece los requisitos mínimos de infraestructura y equipamiento de hospitales y consultorios de atención médica especializada.
			
			Esta Norma Oficial Mexicana tiene por objeto establecer los requisitos mínimos de infraestructura y de equipamiento para los hospitales y consultorios que presten atención médica especializad
			
					
		\end{itemize}	
\end{enumerate}

\subsection{Sociedad Americana de Ingenieros de Calefacción, Refrigeración y Aire Acondicionado (ASHRAE)}

La ASHRAE es la principal organización enfocada en el análisis técnico de las edificaciones desde hace más de 30 años. Formalmente se consolidó en el año 1984 y actualmente en 2024 cuenta con más de 57,000 miembros en todo el mundo. Estos integrantes conformados entre asociados y miembros se han centrado en la investigación, redacción de norma y publicación continua en sistemas de edificación, eficiencia energética, calidad del aire y sostenibilidad al interior de la industria \cite{ashrae-about}.


\begin{itemize}
	\item ANSI/ASHRAE Standard 34-2022, Designación y clasificación de seguridad de los refrigerantes.
	La Norma ASHRAE, crea una nomenclatura sencilla para referirse a los refrigerantes comunes, con el fin de no utilizar su nombre químico, fórmula o nombre comercial.
	 \begin{itemize}
	 
	 
	\item Los refrigerantes se numeran con una R-, seriada con un número propuesto por la ASHRAE.
	\item Una letra minúscula al final, si el refrigerante tiene un isómero (misma fórmula química, pero no de estructura química igual)
	\item Una letra mayúscula al final, si el refrigerante tiene los componentes puros iguales.
	\item Se añade “xxx” después de “R-” si el refrigerante es una mezcla con al menos otro refrigerante más. 
\end{itemize}
\end{itemize}


\subsection{Normas ISO}

La Organización Internacional de Normalización, es la organización internacional que gracias a todos sus expertos en todas las áreas, desde gestión de calidad a la inteligencia artificial se encargan de asegurarnos productos y servicios, seguros, fiables y de alta calidad.

\begin{itemize}
	\item ISO 5149:2014 – Sistemas de refrigeración mecánicos utilizados para enfriamiento y calefacción – Requisitos de seguridad.
	
	Especifica los requisitos relativos a la seguridad de las personas y los bienes para el diseño, fabricación, instalación y operación de sistemas de refrigeración, y pone el acento en reducir al mínimo las fugas de refrigerante a la atmósfera. 
	Es aplicable a todo tipo de sistemas de refrigeración en los que el refrigerante se evapora y se condensa en un circuito cerrado. 
	\item	ISO 817:2014 Refrigerantes – Designación y clasificación de seguridad
	
	Facilita un claro sistema de numeración y asignación de prefijos para la designación de la composición de los refrigerantes (p. ej., el prefijo CFC se utiliza para designar a los clorofluorocarbonos).
	Clasificaciones de seguridad de los refrigerantes según inflamabilidad y toxicidad.
	
	\item	ISO 11650:1999 Rendimiento de equipos para la recuperación y/o el reciclado de refrigerantes.
	
	Especificación de los aparatos de prueba, las pruebas de las mezclas de gases, los procedimientos de muestreo y las técnicas analíticas que se utilizarán para determinar el rendimiento de equipos para la recuperación y/o el reciclado de refrigerantes.
	
\end{itemize}

\section{Contexto geográfico}\label{sec:contex_geografico}
	Tener en cuenta el contexto geográfico en un proyecto de refrigeración es fundamental para adaptar el diseño y las especificaciones técnicas a las condiciones climáticas locales. Esto permite garantizar un rendimiento óptimo y eficiente de los sistemas de refrigeración, así como una mayor durabilidad y fiabilidad en diferentes entornos ambientales.
	
	
\subsection{Ubicación geográfica}

El proyecto será diseñado bajo condiciones específicas que residen en el interior de la Clínica número 40, la cual a sus vez es una Unidad Médica Familiar del IMSS, ésta se ubica en la avenida Hidalgo número 24 colonia entre Cerrada de Hidalgo y la Era en la colonia Santa Bárbara de la alcaldía Azcapotzalco Ciudad de México, CP 02230. En las figuras \ref{fig:mapsumf40} y \ref{fig:sateliteumf41} se aprecian la ubicación a través de Google Maps, (nivel de cuadra y calles) y en vista satelital (nivel Ciudad de México) respectivamente. 


\begin{figure}[H]
	\centering
	\includegraphics[width=0.6\linewidth]{figures/mapsumf40}
	\caption{Unidad Médica Familiar 40, Santa Bárbara Azcapotzalco Ciudad de México.}
	\label{fig:mapsumf40}
\end{figure}

\begin{figure}[H]
	\centering
	\includegraphics[width=0.6\linewidth]{figures/sateliteumf41}
	\caption{Vista satelital de la ubicación de la Unidad Médica Familiar 40 en Ciudad de México.}
	\label{fig:sateliteumf41}
\end{figure}

\subsection{Variables climáticas de la región}

Por propiedades espaciales para cuestiones climatológicas se hará uso a nivel de la Ciudad de México, puesto que las variaciones en la zona especificada del diseño son despreciables. 

En la Ciudad de México, de acuerdo a datos oficiales de INEGI, la mayor parte del territorio se tienen presencia de un clima Templado subhúmedo (87\%), en su parte restante se tiene un clima Seco y semiseco (7\%) y en Templado húmedo (6\%) (Ver figuras \ref{fig:climadf} y \ref{fig:calordf}).

Así mismo se sabe por estadísticas que la temperatura media anual es de 16°C, mientras que la más alta es mayor a los 25°C en los meses de marzo a mayo  y la menor está cercana a 5°C en enero. Las lluvias se presentan en verano, la precipitación total anual es variable: en la región seca es de 600 mm y en la parte templada húmeda (Ajusco) es de 1 200 mm anuales.
La zona urbana ocupa la mayor parte del territorio, pero hacia la parte sur y sureste se encuentran zonas agrícolas, principalmente de temporal, donde se cultiva maíz, frijol, avena y nopal entre otras, siendo importantes también las hortalizas y la floricultura. \cite{clima-df}.


\begin{figure}[H]
	\centering
	\includegraphics[width=0.6\linewidth]{figures/climadf}
	\caption{Condiciones climáticas de la Ciudad de México.}
	Fuente: \cite{clima-df}
	\label{fig:climadf}
\end{figure}

\begin{figure}[H]
	\centering
	\includegraphics[width=0.6\linewidth]{figures/calordf}
	\caption{Mapa y radar del tiempo para Mexico city, DF (Weather Channel, 2024)}
	\label{fig:calordf}
\end{figure}


\section{Contexto del producto}


La diabetes mellitus (DM) forma un grupo de enfermedades crónico-degenerativa caracterizadas por hiperglucemia debido a defectos en la secreción y/o acción de la insulina. La hiperglucemia crónica se asocia con daños a largo plazo en varios órganos, especialmente los ojos, riñones,  los nervios, vasos sanguíneos y el corazón \cite{alfaro2002}.

La RAE\footnote{Real Academia Española}, (2023) y NCI\footnote{National Cancer Institute} , (s. f), coinciden que la insulina es aquella hormona desarrollada por los islotes de Langerhans en el páncreas (figura 2.10 (a)), que regula la cantidad de glucosa existente en la sangre.


\begin{figure}[H]
	\centering
	\includegraphics[width=0.36\linewidth]{figures/structureinsuline}	\includegraphics[width=0.36\linewidth]{figures/insulinbottle}\\
	(a) \hspace*{5cm} (b)
	\caption{a) Estructura de la insulina y su interacción con partes del cuerpo (Mandal, 2010) . (b) Frasco de insulina Inyectable comercial}
	Fuente: \cite{healthcentral2022}. 
	\label{fig:structureinsuline}
\end{figure}

Los productos de insulina alojados en frascos o cartuchos (figura 2.10 (b)) proporcionados por los fabricantes, tanto abiertos como cerrados, pueden mantenerse a temperatura ambiente, entre 59 °F y 86 °F, durante un máximo de 28 días sin que su eficacia se vea comprometida. Sin embargo, si la insulina ha sido diluida o extraída del envase original del fabricante, debe ser desechada dentro de un plazo de dos semanas \cite{research2017}.

Como ya se dijo antes, la insulina es una hormona polipeptídica, la cual estructuralmente en química está formada por 2 cadenas, una de 21 aminoácidos, la A y otra de 30 aminoácidos, la B, unidas por 2 enlaces disulfuro y existe un tercer enlace disulfuro dentro de la cadena A. \cite{gonzalez2017}.


\subsection{Almacenamiento de la insulina}

 
Los fabricantes de medicamentos ofrecen insulina en dos presentaciones básicas: viales y plumas.
 
Los requisitos generales de almacenamiento de la insulina enlistadas aquí siguen de una norma general que contemplan los distribuidores y consumidores a través de la cadena de suministro de la insulina (Ver figura \ref{fig:suministro}):

\begin{enumerate}
	 \item	La insulina congelada debe desecharse.
	\item	No utilice nunca la insulina después de la fecha de caducidad indicada en el vial, la pluma o el cartucho suministrado por el fabricante.
	\item	Nunca expongas la insulina al calor o la luz directa.
	\item Inspeccione la insulina antes de cada uso. No debe utilizarse ninguna insulina que presente grumos o partículas blancas sólidas.
\item La insulina debe ser transparente no debe tener ningún aspecto turbio.

	Formas de determinar si la insulina está en buenas condiciones, consulte la tabla 1 de alguno de los productos de insulina.
\begin{enumerate}
\item[i.]	La insulina sin abrir y sin usar debe almacenarse en un frigorífico a una temperatura de 2 a 8 °C.
\item[ii.]	La insulina abierta y en uso debe almacenarse a una temperatura ambiente inferior a 30°C, tal que la consuma en el menor tiempo posible.
\item[iii.]	Si recibe insulina por envío a domicilio, confirme siempre que la insulina se va a almacenar en condiciones adecuadas.
	\end{enumerate}
\end{enumerate}



\begin{figure}[H]
	\centering
	\includegraphics{figures/suministro}
	\caption{Cadena de suministro más común de la insulina Industria a Paciente}
	Fuente: Elaboración propia, basado de \cite{opsonu}.
	\label{fig:suministro}
\end{figure}



\begin{table}[H]
	\centering
	\caption{Calidad y caducidad específica aplicable a productos de insulina.}
	Fuente: \cite{jacob2023}.\\
	\begin{tabular}{@{}cccc@{}}
		\toprule
		\multicolumn{1}{l}{\textbf{Producto}} & \multicolumn{1}{l}{\textbf{Sin abrir}} & \textbf{\begin{tabular}[c]{@{}c@{}}Abierto \\ (Temperatura ambiente $T_{amb}$ )\end{tabular}} & \multicolumn{1}{l}{\textbf{Sello dañado}} \\ \midrule
		\underline{\textbf{Viales:}}                & \multicolumn{3}{l}{}                                                                                                                                                    \\
		Fiasp®                                & Fecha de caducidad                     & 28 días                                                                            & 28 días                                   \\
		Humulin R                             & Fecha de caducidad                     & 31 días                                                                            & 31 días                                   \\
		Novolin                               & Fecha de caducidad                     & 42 días                                                                            & 42 días ($T_{amb}$)            \\
		\underline{\textbf{Bolígrafos:}}   & \multicolumn{3}{l}{} \\
		Fiasp®                                & Fecha de caducidad                     & 28 días                                                                            & 28 días                                   \\
		Humulin N Pen                         & Fecha de caducidad                     & 14 días                                                                            & 14 días                                   \\
		Lyuumjev®                             & Fecha de caducidad                     & 28 días                                                                            & No refrigerar                             \\ \bottomrule
	\end{tabular}
	\label{tabla:caducidades}
\end{table}


\textbf{Nota}: La tabla \ref{tabla:caducidades} es una  referencia de investigación no está destinada a usarse como herramienta para hacer prescribir insulina\footnote{La insulina pierde eficacia al exponer a temperaturas elevadas, lo que  puede con llevar a la disminución del control de glucosa en la sangre, no se debe exponer insulina a más de 30°C}.









	 

	
	%\newpage
 
%	\setcounter{page}{11}
	
	
	 
 
 \clearpage
 \newpage
% \addcontentsline{toc}{chapter}{\hfill 34}
 \addtocontents{toc}{\protect\contentsline{chapter}{CAPÍTULO III. Meta diseño  \hfill  34}{}{}}
 

 \begin{titlepage}

 
	\centering
	\begin{tikzpicture}%opacity=0.5
		\node[inner sep=0pt, ] (image) at (0,0) {\includegraphics[width=\textwidth]{figures/design-enfria2.jpg}};
		\fill [white,path fading=south] (-5,-4) rectangle (5,4);
		\node[black,font=\Huge\bfseries] at (0,3) {Capítulo III. Meta diseño};
		\node[black,font=\Large\bfseries] at (0,1) {Diseño preliminar del proyecto};
	\end{tikzpicture}
\end{titlepage}




\newpage 
\section*{Introducción}
\addcontentsline{toc}{section}{Introducción}\rsp
\setcounter{chapter}{3}
\setcounter{section}{0}
\setcounter{figure}{0}
\setcounter{page}{35}
\setcounter{table}{0}

El diseño óptimo de frigoríficos usados en la industria para enfriar y congelar productos cada vez más requieren urgentemente de, (i) información técnica de los diversos sistemas existentes, de sus componentes así como de sus aspectos técnicos y operativos del mismo equipo; (ii) memorias de cálculos de análisis energético y exergético para optimizar el diseño del sistema; (iii) uso de las metodologías de refrigeración correctas, (iv) análisis del rendimiento de los componentes; (v) análisis de datos de refrigeración para el diseño de estos mismos sistemas existentes \cite{Dincer2010-re}. 

Derivado de lo anterior, en este capítulo se abordan los fundamentos del diseño óptimo de una cámara de refrigeración para el almacenamiento de insulina, considerada como una parte crítica en la cadena de suministro de medicamentos sensibles a la temperatura. Se hace hincapié en la necesidad de un enfoque multidisciplinario y colaborativo para abordar los desafíos relacionados con el almacenamiento de insulina y se destacan las oportunidades futuras para la mejora continua y la innovación en este campo crucial de la atención médica.

Comprender los principios fundamentales de la preservación de la eficiencia y el diseño de cámaras frigoríficas en ingeniería es esencial para garantizar la integridad y la eficacia de la insulina, un medicamento vital para millones de personas con diabetes en nuestro país.

En el desarrollo de este capítulo, se presenta una estructura clara y detallada para abordar la complejidad de esta tarea. En primer lugar, se analiza exhaustivamente la problemática específica que se busca resolver con este proyecto de ingeniería, centrada en la falta de equipo frigorífico para insulina en la Unidad Médica Familiar 40, ubicada en Azcapotzalco, Ciudad de México. Esta carencia no solo representa un desafío logístico, sino que también compromete la calidad y la eficacia del tratamiento para los pacientes que dependen de este medicamento para controlar su enfermedad (la diabetes).

A continuación, se ahonda en las herramientas del diseño, explorando aspectos técnicos clave para comprender mejor el problema y la solución propuesta. Este análisis incluye la evaluación de requisitos de temperatura, capacidad de almacenamiento, eficiencia energética y otros aspectos relevantes para el diseño y funcionamiento óptimo de la cámara de refrigeración.

Luego, se detallan postulados o marcos de referencia relevantes para la solución de la problemática. Desde el manejo adecuado y seguro de la insulina hasta conceptos teóricos avanzados en el campo de la refrigeración, esta sección proporciona un contexto teórico sólido para informar el diseño y la implementación de la solución.

El cuarto apartado se centra en el ecodiseño, destacando la importancia de considerar criterios de eficiencia y sostenibilidad ambiental en todas las etapas del proceso de diseño y fabricación. Al abordar estos aspectos desde el inicio del proyecto, se busca minimizar el impacto ambiental del sistema de refrigeración y promover prácticas responsables en el uso de recursos naturales.

Seguidamente, se presentan las restricciones técnicas, tanto internas como externas, que deben tenerse en cuenta en soluciones de ingeniería de este tipo. Desde la confiabilidad y seguridad del sistema hasta las políticas y regulaciones que rigen su implementación, estas restricciones juegan un papel crucial en la viabilidad y el éxito del proyecto.

En el sexto apartado, se examinan las regulaciones normativas y legales relacionadas con el diseño de ingeniería y el manejo de la insulina como medicamento destinado al uso de pacientes con diabetes. Cumplir con estas regulaciones es fundamental para garantizar la calidad y seguridad del sistema de refrigeración, así como para mantener la integridad de los medicamentos almacenados.

Posteriormente, se proporciona un bosquejo general del proyecto y su listado de componentes, detallando los elementos clave que formarán parte de la solución final. Esto incluye desde los componentes mecánicos y eléctricos hasta los sistemas de monitoreo y control que garantizarán el funcionamiento eficiente y confiable de la cámara de refrigeración.

Cabe mencionar que gran parte del trabajo desarrollado en este capítulo se logró de manera eficiente por la ayuda otorgada por el Coordinador de Farmacia Familiar en la Unidad Médica Familiar 40, el señor David Ledesma.

Finalmente, se presentan las conclusiones, resumiendo los hallazgos clave y destacando la importancia del desarrollo de este capítulo en el contexto más amplio del proyecto. \rsp
\newpage


\section{Identificación y estructuración de la necesidad detectada}\rsp
\subsection{Necesidad básica}\rsp{}
Los requerimientos son inherentes a cualquier entidad biológica y se refieren a una percepción generada por la noción de carencia, ya sea en términos materiales, físicos o emocionales. Constituyen uno de los pilares esenciales de la existencia, no solo del ser humano sino también de otras formas de vida. Es la exigencia la que impulsa a los organismos a emprender acciones en búsqueda de metas que les permitan cubrir aquello que consideran necesario.
En este sentido, una carencia tecnológica se describe como la discrepancia entre la tecnología existente y la necesaria para potenciar el rendimiento y la competitividad del sector industrial \cite{marrelli-2011,imp-2018}.

 En el contexto del sistema de refrigeración para la conservación de insulina, la necesidad tecnológica se centra en la falta de equipos frigoríficos adecuados para garantizar la estabilidad y eficacia de la insulina almacenada en  la Unidad Médica Familiar 40 (U.M.F 40) en Azcapotzalco, Ciudad de México.
\subsection{Metodología de identificación de la necesidad}
La identificación de las necesidades tecnológicas se llevó a cabo mediante una entrevista personal con Ledesma D. Coordinador de Farmacia Familiar en la U.M.F 40 el día 03 de mayo de 2024 en las instalacione de dicha unidad. En esta visita se recabó información de manera exhaustiva de la situación actual en la unidad, así como a través de un par de consultas con profesionales de la salud dentro de esta misma unidad. Se realizaron visitas al lugar para evaluar las condiciones de almacenamiento de insulina y se recopiló información sobre los problemas y desafíos existentes.


\subsection{Necesidades tecnológicas específicas}
De acuerdo a la entrevista ya mencionada y las visitas realizadas en más de una ocasión, se enlistan algunas de las carencias tecnológicas que se han detectado al interior de la farmacia de medicina familiar y en la red de frío de la U.M.F 40.
\begin{enumerate}
\item Infraestructura: Falta de equipos frigoríficos adecuados y mantenidos para mantener la temperatura requerida para la conservación de insulina. "Se cuenta con una sola unidad de refrigeración para insulina con más de 20 años operando".  
\item Conocimiento: Escasez de información o capacitación sobre el diseño y operación de sistemas de refrigeración para la conservación de insulina. "Solo dos personas aquí entendemos qué pasa internamente con la cámara cuando deja de funcionar". 
\item Metodologías de trabajo: Falta de procedimientos estandarizados para la gestión y distribución de insulina en unidades médicas- "La logistica realmente está en otros hospitales, entonces muchas veces el personal actúa según sus criterios o conocimientos". \cite{david-umf}
\item Herramientas: Carencia de tecnología de refrigeración específicamente diseñada y adaptada para las necesidades de conservación de insulina en entornos médicos. "Como ya se mencionó, el equipo con el que se cuenta es demasiado obsoleto". \cite{david-umf}

\end{enumerate}\rsp
\section{Interrogantes del diseño artefactual}\rsp
Tener un compendio de preguntas en base al diseño que se desee realizar, sirven como herramientas fundamentales para comprender, evaluar y mejorar el proceso de creación de equipos de ingeniería con propósitos específicos. Estas preguntas también servirán de guía en el camino del diseño.\\[-1cm]

\begin{enumerate}
	\item ¿\textit{Cuál es el producto y/o artículo que se va a manejar}?\\ De acuerdo a la información recabada en la visita a la unidad 40 del IMSS se sabe que a los pacientes diabéticos se les proporciona dos tipos de insulina para su tratamiento, \textbf{Humalog Insulina Lispro} Figura \ref{fig:lispro-insul} e \textbf{Insulina Lantus}  Figura \ref{fig:lantus-insul}, este par de medicamentos se suministran directamente por parte del gobierno por lo que la logistica del proceso en la cadena de suministro está debidamente regulada. 
	
	Nota: 100 u/ml es el equivalente para 100 unidades/mililitros por solución inyectable en un vial para ambos productos. Como se aprecia en las cuadros de información. (Vea Cuadros  \ref{tabla:humalog} y \ref{tabla:lantus})  las magnitudes de medida y envasado son muy similares en ambos medicamentos de distinta marca, pero se consideran para cuestiones cálculos de capacidad en el siguiente capítulo.\rsp
		
\begin{table}[H]
	\centering
		\caption{Características de Humalog Insulina lispro. }\cite{lispro-2006}
	\begin{tabular}{ll}
	\toprule
		\textbf{Característica}     & \textbf{Humalog Insulina lispro (100  u/ml )} \\ \midrule
		Peso                        & 100g             \\ % \hline
		Tipo de envasado            & Caja + frasco de vidrio ($12cm\times 3{.}5cm$)     \\ %\hline
		Rendimiento                 & Alta                             \\% \hline
		Precisión                   & (>95\%)                                 \\ %\hline
		Refrigeración       &  Entre 2$^\circ$C y 8$^\circ$C                           \\% \hline
		Tipo de producto & Líquido en frasco de vidrio ambar DIN18\\\bottomrule
	\end{tabular}
	\label{tabla:humalog}
\end{table}\rsp
\begin{table}[H]
	\centering
	\caption{Características de Lantus Insulina Glargina}\cite{lantus-2015}
	\begin{tabular}{ll}
	\toprule
	\textbf{Característica}     & \textbf{Lantus\textsuperscript{\textregistered} Insulina} \\ \midrule
	Peso                        & 100g             \\ % \hline
	Tipo de envasado            & Caja + frasco de vidrio  ($12cm\times 3{.}5cm$)     \\ %\hline
	Rendimiento                 & Alta                             \\% \hline
	Precisión                   & (>95\%)                                 \\ %\hline
	Refrigeración       &  Entre 2$^\circ$C y 8$^\circ$C                            \\% \hline
	Tipo de producto & Líquido en frasco ampula\\ \bottomrule
\end{tabular}
	\label{tabla:lantus}
\end{table}
\newpage
\begin{multicols}{2}
 
\begin{figure}[H]
	\centering
	\includegraphics[width=0.6\linewidth]{figures/lispro-insul}
	\caption{Envase de insulina Humalog 10ml}
	\label{fig:lispro-insul} Fuente: \cite{sanpablo}
\end{figure}
\begin{figure}[H]
	\includegraphics[width=0.6\linewidth]{figures/lantus-insul}
	\caption{Envase de insulina Lantus 10ml}
	\label{fig:lantus-insul}   Fuente: \cite{ahorro}
\end{figure}

\end{multicols}
 



	\item \textit{¿Qué tipo de producción se va a realizar?}\\
	El principal motivo de la cámara frigorífica será la \textbf{conservación de la insulina}, específicamente los dos tipos de insulina ya descritos. También el proceso de conservación implica actividades como las que se mencionan en la pregunta 5.	Estas actividades de producción se centran en asegurar que la insulina se mantenga en condiciones óptimas para su uso seguro y efectivo en el tratamiento de la diabetes.
	
	\item \textit{¿Qué tipo de línea de producción?}\\
En el caso de la conservación de insulina, se usará \textbf{una línea de producción específica} y dedicada exclusivamente a este medicamento. Es fundamental evitar la mezcla de insulina con otros productos farmacéuticos debido a razones de seguridad y eficacia. La insulina es un medicamento vital para el tratamiento de la diabetes, y cualquier contaminación cruzada o confusión en su distribución puede tener consecuencias graves para la salud de los pacientes. Además, el manejo de múltiples productos en la misma línea de producción aumenta el riesgo de errores humanos y puede comprometer la calidad y pureza del medicamento. Por lo tanto, se recomienda encarecidamente que la línea de producción de insulina se mantenga separada y dedicada exclusivamente a este medicamento, garantizando así la seguridad y eficacia de su producción y distribución. \cite{david-umf}
	\item \textit{¿Qué condiciones ambientales se requieren?}\\
	Dado que el equipo frigorífico está destinado para ubicarse en la U.M.F 40, se tomarán variables metereológicas de la Ciudad de México espacialmente ubicándose en las coordenadas de la alacaldía Azcapotzalco, puesto que en \cite{cressie-1993} se menciona que las variables medidas espacialmente tienen dependencia y correlación es válido tomar datos medidos a nivel general de la CDMX. (Vea Cuadro \ref{tabla:clima}) 
	 
		
		 
			\begin{table}[H]
				\centering
				\caption{Resumen de información sobre el clima y características del valle de México.}\cite{weather-cdmx}
				\begin{tabular}{p{5cm}p{7cm}}
					\toprule
					\textbf{Información} & \textbf{Detalle} \\
					\midrule
				\begin{center}
					 \vspace*{0.5cm}Clima
				\end{center}	 & 
					\begin{itemize}
						\item Templado subhúmedo: 87\%
						\item Seco y semiseco: 7\%
						\item Templado húmedo: 6\%
					\end{itemize} \\
					%\midrule
					Temperatura media anual & 16°C \\
					%\midrule
					\begin{center}
						\vspace*{0.5cm} Temperaturas extremas
					\end{center}	
					 & 
					\begin{itemize}
						\item Más alta (>25°C): marzo a mayo
						\item Más baja (alrededor de 5°C): enero
					\end{itemize} \\
					%\midrule
					\begin{center}
						\vspace*{0.5cm}Precipitación total anual
					\end{center}	
					 & 
					\begin{itemize}
						\item Región seca: 600 mm
						\item Parte templada húmeda: 1 200 mm
					\end{itemize} \\
				%	\midrule
					\begin{center}
						\vspace*{0.5cm}Ecosistemas
					\end{center}	 & Avance de la mancha urbana ha puesto en peligro todos los ecosistemas, especialmente los lagos. \\
					%\midrule
					\begin{center}
						\vspace*{1.2cm}Uso de suelo
					\end{center}	 & 
					\begin{itemize}
						\item Zona urbana: mayor parte del territorio
						\item Zonas agrícolas: principalmente al sur y sureste, con cultivos de maíz, frijol, avena, nopal, hortalizas y floricultura.
					\end{itemize} \\
					\bottomrule
				\end{tabular}
				\label{tabla:clima}
			\end{table}
		 
		
	\item \textit{¿Cual es el proceso de trabajo que tiene que realizar el sistema artefactual?}
	\begin{itemize}
		\item Recepción: Ingreso de la insulina como penúltimo punto de la cadena de suministro descrita en la \hyperlink{figura-2-15}{figura 2.15} para poder ser entregada al paciente cuando lo solite. 
		\item Pre-enfriamiento: Reducimos la temperatura del medicamento y mantenemos en un valor fijo de temperatura, la cual se decide en cuestión del consumo energético de la zona.
		\item Almacenar: Mantener la insulina a la temperatura adecuada (2$^\circ$C a 8$^\circ$C) para garantizar su estabilidad y eficacia .
		\item Conservar: Proteger la insulina de condiciones ambientales adversas que puedan afectar su calidad, en nuestro caso los sismos (temblores).
		\item Supervisar: Monitorizar constantemente la temperatura y otras condiciones dentro de la cámara para asegurar que se mantengan dentro de los rangos aceptables, se adjuntará un sensor de temperatura para que cuando la temperatura en el sistema comience a subir este baje la temperatura a 2$^\circ$C nuevamente.
		\item Registrar: Mantener registros precisos de los niveles de inventario, fechas de vencimiento y otros datos relevantes para el seguimiento y la trazabilidad del producto. Actividades destinadas al personal para control interno y para temas de mantenimiento del equipo.
		\item Organizar: Distribuir y organizar la insulina de manera eficiente dentro de la cámara para facilitar su acceso y manejo. Se dispondrán de divisiones horizontales comunes para poder diferenciar ambas marcas de insulina. 
	\end{itemize}		
	\item \textit{¿Cuántas etapas de trabajo tiene que realizar el sistema artefactual?}\\
	Según \citeauthor{cengel-2009}\citeyear{cengel-2009},  el ciclo real de refrigeración por compresión consta de los siguientes procesos vea Figura la \ref{ciclo-ref}: evaporación, compresión, condensación y expansión, por lo cual siguiendo este ciclo se destinarán las etapas de trabajo que va a realizar nuestro sistema. Entonces el diseño realizará \textbf{6 etapas} para llevar a cabo la refrigeación y conservación de la insulina.
	
	% revisar aquí en futuro para mejorar
	\begin{enumerate}
		\item[i)] Inicio del Sistema: Cuando se enciende el sistema de refrigeración, el compresor es el primer componente en activarse. El compresor es el motor del sistema y su función es comprimir el refrigerante gaseoso.
		\item[ii)] Compresión del Refrigerante: El compresor aumenta la presión y la temperatura del refrigerante gaseoso. Este proceso demanda una cantidad considerable de energía, especialmente para superar la resistencia de la compresión.
		\item[iii)] Transferencia de Calor en el Condensador: El refrigerante comprimido se dirige hacia el condensador, donde cede calor al entorno circundante. El condensador, un intercambiador de calor, convierte el refrigerante gaseoso en líquido al disipar el calor hacia el exterior.
		\item[iv)] Expansión del Refrigerante en la Válvula de Expansión: Después de enfriarse en el condensador, el refrigerante pasa a través de la válvula de expansión. Esta válvula reduce la presión del refrigerante y lo dirige hacia un conducto más estrecho, lo que resulta en una rápida disminución de la temperatura.
		\item[v)] Absorción de Calor en el Evaporador: El refrigerante líquido y frío ingresa al evaporador, donde absorbe el calor del entorno interior de la cámara de refrigeración. Durante este proceso, el refrigerante se evapora y extrae calor del ambiente, manteniendo así una temperatura baja en el interior de la cámara.
		\item[vi)]Retorno del Refrigerante al Compresor: Una vez que el refrigerante ha absorbido el calor, regresa al compresor en forma de vapor para reiniciar el ciclo de refrigeración.
	\end{enumerate}
	Durante todo este proceso, diversos componentes del sistema, como los tubos y conexiones, el aceite lubricante, los controladores y sensores, y el aislamiento, colaboran para garantizar un funcionamiento eficiente y confiable. \cite{neoattack-2023,cengel-2009}
	
	\begin{figure}[H]
		\centering
		\includegraphics[width=0.5\linewidth]{figures/ciclo-ref}
		\caption{Diagrama de un sistema de refrigeración típico con R-134a}
		\label{fig:ciclo-ref} Fuente: \cite{agustin}
	\end{figure}
	
	
	\item \textit{¿El diseño del sistema artefactual puede ser modular y flexible?}
	\textbf{Sí} De acuerdo con los sitios web de las empresas \citeauthor{hogartecnocasa-2016} y \citeauthor{neoattack-2023} un sistema modular es aquél que puede ser manipulado mendiante un ordenador o software especializado y que en sí mismo puede ser dividido en sistemas más pequeños, en los cimientos de este proyecto es un aspecto que no se considerará como parte del estudio por las premuras del tiempo, sin embargo es algo que sin duda puede considerarse. No obstante el proyecto en cuestión sí que es flexible por sus dimensiones que serán designadas bajo un análisis profundo en el siguiente capítulo. 
	
	
\item 	\textit{¿Se puede aplicar la inteligencia artificial y la robotización en el sistema?}\\
\textbf{Sí}, es factible aplicar tanto la inteligencia artificial (IA) como la robotización en un sistema de refrigeración destinado a la preservación de la insulina. Estas tecnologías pueden aportar una serie de ventajas significativas en términos de eficiencia, automatización y optimización del rendimiento del sistema. En \citeauthorNP[et. al.]{madriz-ramirez-2022} \citeyear{madriz-ramirez-2022}, se ha trabajado en un diseño especializado en refrigeración que resulta novedosa y práctica, este es un ejemplo práctico de como técnicas modernas pueden coadyuvar a mejorar la eficiencia y optimización de proyectos en ingeniería que a lo largo de los años han ido adaptándose también y como era de esperarse la ingeniería mecánica aplicada a los sistemas de refrigeración no serían la excepción.

\item 	\textit{¿Cómo podemos aprovechar las nuevas fuentes de datos?}\\
El gran campo que se ha creado entorno a los datos era conocida como "big data" por los investigadores sin lugar a dudas puede ser de gran utilidad para el diseño óptimo de sistemas de refrigeración ya que de bases de datos históricas podemos determinar el tiempo de fallo de componentes o tiempos de vida útil, así como de ellas se pueden sacar ventajas para la predicción de la falla de componentes, la monitorización en tiempo real también puede ser una herramienta novedosa aplicada al campo de estudio que conciernea este proyecto. Finalmente en el siguiente artículo  hecho por investigadores de China muestran estudios de lo potenciales que pueden ser estas nuevas bases de datos y modelos en monitorización de temperaturas en distintas temporadas del año, vea \cite{liu-2021} para más detalles.


\item	Conexión a un HOST (computadoras, tabletas móviles, portátiles)\\
De igual manera, la conexión a estos dispositivos que cada día son más frecuentes en el uso cotidiano sin dudarlo ayudarían mucho a los sistemas de vigilancia remotas en tiempo real del sistema de refrigeración, por ejemplo en tiempos de epidemia en donde el contacto con material de salud suele ser muy delicado para evitar contagios se pueden aplicar a la regulación de la temperatura y de humedad desde algún dispositivo móvil.



\item 	¿Se puede aplicar el desmantelamiento y reciclado del sistema al ciclo de uso?
Esta es una de las problemáticas principales que se siguen trabajando en los diseños de ingeniería en el campo farmaceútico, debido a que por normas oficiales los instrumentos y equipos de trabajo destinados a la salud pública tienen estrictamente prohibido la reutilización en actividades que no sean internas de la instancia de gobierno (clínica, UMF, hospitales, etc), sin embargo es claro que para fines del mismo equipo con autorización de expertos en área de salud y diseño farmaceútico estos componentes o equipos pueden ser reutilizados en un equipo nuevo o averiado.
	
\end{enumerate}


\section{Marco teórico }


La insulina, al igual que otros fármacos basados en hormonas peptídicas, está sujeta a la influencia de las condiciones ambientales, lo cual impacta su estabilidad y eficacia terapéutica. A pesar de la omnipresencia de la insulina en múltiples etapas de la cadena de suministro y su uso diario por personas con diabetes, existe una notable falta de investigación sobre cómo varía su potencia durante la cadena de frío y hasta el momento de su administración.

Para garantizar la calidad y eficacia de la insulina, es fundamental comprender cómo el calor y el frío afectan su estabilidad. Las recomendaciones de almacenamiento reflejan la importancia de mantener condiciones específicas de temperatura para preservar la integridad del medicamento. Sin embargo, la falta de datos públicos sobre la estabilidad de la insulina y su potencia en diferentes condiciones de almacenamiento dificulta la evaluación de los riesgos asociados con su uso.  \textcolor{white}{ \citeyearNP{heinemann-2020}} \cite{heinemann-2020}

 Además, mejorar la eficiencia energética en sistemas de refrigeración es crucial, especialmente considerando que los sistemas HVAC\&R son notoriamente grandes consumidores de energía. La adopción de tecnologías más ecológicas, como compresores de corriente continua y el uso de energías renovables, ofrece oportunidades significativas para reducir costos y minimizar el impacto ambiental de estos sistemas. Investigaciones en este campo están en curso, buscando desarrollar sistemas de refrigeración más eficientes y sostenibles.
 
 Por último, en el diseño de cámaras de refrigeración, es esencial tener en cuenta tanto los cálculos teóricos como la teoría detrás de los principios de refrigeración. Esto garantiza que las cámaras cumplan con los estándares de temperatura necesarios para preservar la integridad de los productos farmacéuticos, como la insulina, y mantener la eficiencia energética de los sistemas de refrigeración. Un diseño bien fundamentado asegura un almacenamiento adecuado y seguro de medicamentos sensibles a la temperatura, contribuyendo así a la salud y bienestar de los pacientes. \cite{parise-2005}


	\subsection{Carga del producto }	

En la termodinámica clásica, nos interesan principalmente los cambios y las transferencias de energía y masa entre ciertos cuerpos (por ejemplo, un tanque de agua o una cantidad de gas, etc.). Para clarificar las discusiones, llamamos al cuerpo particular de interés (o región del espacio) el sistema. El sistema a menudo (pero no necesariamente) está contenido dentro de algún límite físico. Cuando el límite del sistema permite que la materia (masa) entre o salga del sistema, se dice que es un sistema abierto. Cuando el límite del sistema no permite que la masa entre o salga del sistema, pero permite que el trabajo y/o el calor se transfieran a través del límite, se dice que es un sistema cerrado. En el caso inusual de que el límite del sistema no permita que ni la masa ni la energía (en forma de calor o trabajo) crucen el límite, se dice que es un sistema aislado.

La distinción entre sistemas abiertos y cerrados es importante, ya que los análisis termodinámicos de sistemas cerrados tienden a ser más simples, debido al hecho de que no tenemos que considerar los cambios en la energía de un sistema producidos simplemente por la transferencia de masa con su energía intrínseca dentro o fuera del sistema. Este es el caso del refrigerante en un refrigerador o bomba de calor (siempre que no haya fugas), y por lo tanto solo necesitamos considerar los efectos de transferencia de trabajo y calor. \cite{james-2013}

De acuerdo con \citeauthor{bohn}, \citeyear{bohn} algunas consideraciones importantes para el diseño frigorífico en base un producto perecedero la carga de producto recae en lo siguiente. 
\begin{enumerate}
	\item Cuando un artículo se introduce en una cámara de refrigeración o congelación, su temperatura disminuye hasta alcanzar la temperatura deseada, (vea el \hyperref[anexo:bohn-perecederos]{anexo 4})  , para conocer algunas características comunes a considerar en el diseño. Este proceso involucra tres componentes de carga térmica:
	\begin{enumerate}
		\item El calor específico se refiere a la cantidad de calor que se debe extraer de una libra de producto para reducir su temperatura en 1°F. Este valor varía dependiendo de si el producto está por encima o por debajo del punto de congelación.
		\item El calor latente, conocido como calor de fusión, es la cantidad de calor que se debe eliminar para congelar una libra de producto. La mayoría de los productos tienen un punto de congelación típico, y se puede estimar en 28°F si no se conoce exactamente.
		\item La respiración es el calor generado por frutas y verduras frescas mientras están almacenadas, debido a su actividad metabólica. Este calor varía según el tipo y la temperatura del producto y se mide en unidades BTU por libra por día.
	\end{enumerate}
	\item Cuando se calcula la carga del producto con un tiempo de abatimiento diferente a 24 horas, se debe aplicar un factor de corrección, que es la relación entre 24 horas y el tiempo de abatimiento, a la carga del producto.
	\item Es importante tener en cuenta que, aunque se pueda calcular el abatimiento de temperatura del producto, no se puede garantizar la temperatura final debido a diversos factores externos e incontrolables, como el tipo de empaque, la posición de la carga y el método de almacenamiento.
\end{enumerate}
\subsection{Cargas térmicas de climatización}
Es importante para el control clínico de la insulina mantener adecuadas la temperatura y la humedad en condiciones ideales para la preservación segura de este medicamento.

El análisis de las cargas térmicas es un proceso llevado a cabo por expertos para evaluar las demandas de climatización de un entorno, sin importar su propósito, ya sea residencial, comercial o industrial.
La carga térmica se refiere a la cantidad de energía requerida para mantener o alcanzar ciertas condiciones de temperatura y humedad en un espacio específico, adaptándose a su uso particular, ya sea en un hogar para determinar las necesidades de calefacción, o en un almacén de alimentos congelados para garantizar sistemas de refrigeración confiables. \cite{sampp-2023}
\subsubsection{Cargas sensibles}
Las cargas sensibles incluyen la transmisión de calor a través de los cerramientos
opacos y traslúcidos, la radiación solar, la ventilación o infiltración de aire, la
ocupación del local, la iluminación y la maquinaria presente en el espacio.

\begin{itemize}
	\item \textbf{Cargas por transmisión a través de cerramientos opacos}
\begin{equation} \label{eq:carga_transmision_opacos}
	Q = U \cdot A \cdot \Delta T
\end{equation}
Siendo:
\begin{align*}
	Q & : \text{carga térmica por transmisión (W)} \\
	U & : \text{transmitancia térmica del muro (W/m}^2 \, ^\circ\text{C)} \\
	A & : \text{superficie del muro expuesta a la diferencia de temperaturas (m}^2\text{)} \\
\Delta T & : \text{diferencia de temperaturas, corregida según la orientación del muro y su peso}
\end{align*}

\item \textbf{Cargas por transmisión a través de cerramientos traslúcidos}
\begin{equation} \label{eq:carga_transmision_traslucidos}
	Q = U \cdot A \cdot \Delta T
\end{equation}
Donde:
\begin{align*}
	\Delta T & : \text{diferencia de temperaturas entre las caras interior y exterior del cerramiento (}^\circ\text{C)}
\end{align*}

\item \textbf{Cargas térmicas por radiación solar}
\begin{equation} \label{eq:carga_radiacion_solar}
	Q = S \cdot R \cdot f
\end{equation}
Siendo:
\begin{align*}
	S & : \text{superficie traslúcida expuesta a la radiación (m}^2\text{)} \\
	R & : \text{radiación solar que atraviesa un vidrio sencillo (W/m}^2\text{)} \\
	f & : \text{factores de corrección de la radiación en función del tipo de vidrio}
\end{align*}
En la tabla \ref{table:carrier} se dan los valores de las máximas aportaciones solares a través de vidrio sencillo.\\
En cuanto a los coeficientes de corrección $f$, los que se aplican habitualmente son el de marco metálico ($f = 1{.}17$) y el factor solar del vidrio, que los fabricantes indican en sus fichas técnicas como $g$. En caso de que queramos aplicar dos coeficientes de corrección, deberemos multiplicarlos.
\item \textbf{Carga sensible por ventilación o infiltración de aire exterior}

\begin{equation} \label{eq:carga_ventilacion_infiltracion}
	Q = V \cdot 0{.}34 \cdot \Delta t
\end{equation}
Donde:
\begin{align*}
	V & : \text{caudal de aire infiltrado o de ventilación (m}^3/\text{h)} \\
	\Delta t & : \text{diferencia de temperatura entre el ambiente exterior y el interior (}^\circ\text{C)}
\end{align*}

\begin{table}[H]
	\centering
	\caption{valores de las máximas aportaciones solares a través de vidrio sencillo } Fuente: \cite{carrier-1980}
	\begin{tabular}{@{}lcccccccccc@{}}
		\toprule
		& \multicolumn{1}{l}{} & \multicolumn{9}{c}{Orientación}                                             \\ \midrule
		& \multicolumn{1}{l}{} & \multicolumn{9}{c}{Máximas aportacionas solares R (W/m\textasciicircum{}2)} \\ \cmidrule(l){3-11} 
		\multicolumn{1}{c}{Latitud Norte} & Mes                  & N     & NE     & E      & SE     & S     & SO    & O     & NO    & Horiz.   \\ \midrule
		\multirow{3}{*}{}                 & Junio                & 63    & 437    & 506    & 283    & 66    & 283   & 506   & 437   & 786      \\
		& Julio y Mayo         & 50    & 412    & 515    & 314    & 94    & 314   & 515   & 412   & 774      \\
		& Agosto y Abril       & 34    & 339    & 519    & 405    & 197   & 405   & 519   & 339   & 739      \\
		\multicolumn{1}{c}{30°}           & Sept.y Marzo         & 28    & 283    & 496    & 478    & 329   & 478   & 496   & 283   & 666      \\
		& Oct. y Febrero       & 24    & 122    & 425    & 513    & 456   & 513   & 425   & 122   & 563      \\
		& Nov. y Enero         & 22    & 50     & 364    & 509    & 500   & 509   & 364   & 50    & 456      \\
		& Diciembre            & 19    & 37     & 329    & 509    & 513   & 509   & 329   & 37    & 412      \\ \midrule
		& Junio                & 53    & 418    & 509    & 349    & 169   & 349   & 509   & 418   & 745      \\
		& Julio y Mayo         & 46    & 399    & 515    & 393    & 217   & 393   & 515   & 399   & 732      \\
		& Agosto y Abril       & 34    & 320    & 509    & 458    & 320   & 459   & 509   & 320   & 673      \\
		\multicolumn{1}{c}{40°}           & Sept.y Marzo         & 28    & 182    & 469    & 509    & 440   & 509   & 469   & 182   & 575      \\
		& Oct. y Febrero       & 22    & 109    & 383    & 513    & 509   & 513   & 383   & 109   & 405      \\
		& Nov. y   Enero       & 15    & 37     & 314    & 491    & 522   & 491   & 314   & 37    & 324      \\
		& Diciembre            & 15    & 31     & 270    & 465    & 519   & 465   & 270   & 31    & 267      \\ \bottomrule
	\end{tabular}
\end{table}
	El siguiente escenario fue realizado bajo una temperatura ambiente constante de 43ºC, sin la presencia de productos en su interior. Como se evidencia en la figura \ref{fig:cargas-ter} la pendiente de la carga térmica en el conservador es mayor en el congelador. Este comportamiento se debe principalmente a las dimensiones del conservador las cuales dan lugar a un mayor intercambio de aire con el entorno circundante a la nevera.  \cite{rio}

\begin{figure}[H]
	\centering
	\includegraphics[width=0.46\linewidth]{figures/cargas-ter}
	\caption{Carga térmica vs humedad relativa}
	\label{fig:cargas-ter} Fuente: Este experimento fue realizado por \citeNP{rio}
\end{figure}




\label{table:carrier}
\begin{itemize}
	\item \textbf{Cargas generadas por la iluminación del local}
	
	Se considerará que la potencia integrada de la lámpara se transformará en calor sensible; en el caso de las lámparas de descarga (fluorescentes) se incrementará el valor obtenido en un 25\% para tener en cuenta el cebador y el balasto.	
	\item \textbf{Lámparas incandescentes o LED} 
	\begin{equation} \label{eq:carga_iluminacion_incandescente}
		Q =  {Pot}
	\end{equation}	
	\item \textbf{Lámparas de descarga} 
	\begin{equation} \label{eq:carga_iluminacion_descarga}
		Q = 1{.}25 \cdot {Pot}
	\end{equation}	
	 {Donde:}
	\begin{align*}
		Q & : \text{carga térmica por iluminación (W)} \\
		 {Pot} & : \text{potencia de las lámparas (W)}
	\end{align*}
	\item \textbf{Cargas generadas por las máquinas presentes en el local}
	
	Se considerará que las pérdidas de la maquinaria se transforman íntegramente en calor sensible.	
\item \textbf{Carga generada por maquinaria} 
	\begin{equation} \label{eq:carga_maquinaria}
		Q = \eta \cdot {Pot}
	\end{equation}
	
	 {Donde:}
	\begin{align*}
		Q & : \text{carga térmica por maquinaria (W)} \\
		\eta & : \text{rendimiento de la máquina} \\
		\text{Pot} & : \text{potencia de la maquinaria (W)}
	\end{align*}
\end{itemize}
\subsubsection{Cargas latentes}
\begin{itemize}
	
	\item \textbf{Carga latente por ventilación o infiltración de aire exterior}
	\textbf{Carga latente de ventilación o infiltración.}
	\begin{equation} \label{eq:carga_latente_ventilacion}
		Q = V \cdot 0.63 \cdot \Delta w
	\end{equation}
	
	 {Donde:}
	\begin{align*}
		Q & : \text{carga térmica latente por ventilación o infiltración de aire (W)} \\
		V & : \text{caudal de aire infiltrado o de ventilación (m}^3/\text{h)} \\
		\Delta w & : \text{diferencia de humedad absoluta entre el ambiente exterior y el interior (ºC)}
	\end{align*}
	
\item \textbf{Carga latente por ocupación del local}
	
	Esta carga se determina multiplicando una valoración del calor latente emitido por la persona tipo por el número de ocupantes previstos para el local. La cantidad de calor emitido por persona se obtiene de la tabla que aparece en el apartado donde se describe la Carga sensible por ocupación del local.
	
\end{itemize}
Las formulas descritas se tomaron de \cite{bruno-2024,carrier-1980,sampp-2023}\rsp
\end{itemize}
\section{Ecodiseño}\rsp
El reglamento de ecodiseño se aplica a una amplia gama de equipos, incluyendo unidades condensadoras y centrales de compresores para refrigeración a media y baja temperatura. Nuestros productos cumplen con los requisitos establecidos para coeficiente de rendimiento (COP) y rendimiento estacional normalizado (SEPR), este factor es utilizado para comparación de equipos funcionando en modo refrigeración para escenarios de refrigeración para procesos industriales, garantizando así un funcionamiento eficiente y respetuoso con el medio ambiente (vea el cuadro \ref{tabla:condensadores}) \cite{intarcon-2023}.\\
En el caso de las enfriadoras de proceso , los requisitos de ecodiseño se aplican a equipos de cualquier potencia, con enfriamiento por aire o por agua, tanto para media como para baja temperatura (cuadro \ref{tabla:enfriadores-med-baj}). También se han establecido requisitos para las enfriadoras de proceso de alta temperatura, que entrarán en vigor a partir de enero de 2018 y enero de 2021. (cuadro \ref{tabla:enfriadores-alta})\\
Por ejemplo, para unidades condensadoras con una potencia nominal de hasta 5 kW y 2 kW en media y baja temperatura respectivamente, se exige un COP mínimo a partir del 1 de julio de 2016. Este requisito se incrementa para equipos de mayor potencia, donde el SEPR se convierte en el parámetro de referencia.\\
El ecodiseño no solo se limita a cumplir con los estándares de rendimiento energético, sino que también aborda otros aspectos importantes, como la reducción de la carga refrigerante, la sectorización de los sistemas de refrigeración, la minimización del riesgo de fugas y la utilización de refrigerantes de bajo potencial de calentamiento atmosférico. Estas medidas tienen como objetivo reducir el impacto ambiental de los sistemas de refrigeración y promover un futuro sostenible \cite{caloryfrio-2018}.  
\begin{figure}[H]
	\centering
	\includegraphics[width=0.3\linewidth]{figures/edodis}
	\caption{El Grupo CIAT implementa el Ecodiseño en sus sistemas de climatización y refrigeración.} Fuente: \cite{ciat}
	\label{fig:edodis}
\end{figure}

\begin{table}[H]
	\centering
	\caption{Requisitos de ecodiseño para unidades condensadoras}\cite{intarcon-2023}
	\begin{tabular}{lllcc}
		\toprule
		\textbf{Temperatura} & \textbf{Pot. nominal} & \textbf{Factor} & \textbf{Valor mínimo }  & \textbf{Valor mínimo } \\
		
		 & & & \textbf{desde 01/07/2016} & \textbf{desde: 01/07/2018} \\
		
		\midrule
		Media & $0,2 \, \text{kW} \leq \text{PA} \leq 1 \, \text{kW}$ & COP & 1,20 & 1,40 \\
		& $1 \, \text{kW} < \text{PA} \leq 5 \, \text{kW}$ & COP & 1,40 & 1,60 \\
		& $5 \, \text{kW} < \text{PA} \leq 20 \, \text{kW}$ & SEPR & 2,25 & 2,55 \\
		& $20 \, \text{kW} < \text{PA} \leq 50 \, \text{kW}$ & SEPR & 2,35 & 2,65 \\
		\midrule
		Baja & $0,1 \, \text{kW} \leq \text{PA} \leq 0,4 \, \text{kW}$ & COP & 0,75 & 0,80 \\
		& $0,4 \, \text{kW} < \text{PA} \leq 2 \, \text{kW}$ & COP & 0,85 & 0,95 \\
		& $2 \, \text{kW} < \text{PA} \leq 8 \, \text{kW}$ & SEPR & 1,50 & 1,60 \\
		& $8 \, \text{kW} < \text{PA} \leq 20 \, \text{kW}$ & SEPR & 1,60 & 1,70 \\
		\bottomrule
	\end{tabular}
	\label{tabla:condensadores}
\end{table}


\begin{table}[htbp]
	\centering
	\caption{Requisitos de ecodiseño para enfriadoras de proceso de media y baja temperatura}\rsp 
	\cite{intarcon-2023}
	\begin{tabular}{lllcc}
		\toprule
		\textbf{Condensación} & \textbf{Temperatura} & \textbf{Potencia nominal PA} & \multicolumn{2}{c}{\textbf{Valor mínimo a partir del}} \\
		& & & \textbf{01/07/2016} & \textbf{01/07/2018} \\
		\midrule
		Aire (a 35ºC) & Media (-8ºC) & PA $\leq$ 300 kW & 2,24 & 2,58 \\
		& & PA $>$ 300 kW & 2,80 & 3,22 \\
		&Baja (-25ºC) & PA $\leq$ 200 kW & 1,48 & 1,70 \\
		& & PA $>$ 200 kW & 1,60 & 1,84 \\
		Agua (a 30ºC) & Media (-8ºC) & PA $\leq$ 300 kW & 2,86 & 3,29 \\
		& & PA $>$ 300 kW & 3,80 & 4,37 \\
		&Baja (-25ºC) & PA $\leq$ 200 kW & 1,82 & 2,09 \\
		& & PA $>$ 200 kW & 2,10 & 2,42 \\
		\bottomrule
	\end{tabular}
	\label{tabla:enfriadores-med-baj}
\end{table}

\begin{table}[H]
	\centering
	\caption{Requisitos de ecodiseño para enfriadoras de proceso de alta temperatura}
	\cite{intarcon-2023}
	\begin{tabular}{lllll}
		\toprule
		\textbf{Condensación} & \textbf{Temperatura} & \textbf{Potencia nominal PA} & \multicolumn{2}{c}{\textbf{Valor mínimo a partir del}}\\
		& & & \textbf{01/01/2018}& \textbf{01/01/2021}\\
		\midrule
		Aire (a 35ºC) & Alta (7ºC) & PA $<$ 400 kW & 4.5&5.0 \\
		& & PA $>$ 400 kW & 5&5.5 \\
		Agua (a 30ºC) & Alta (7ºC) & PA $<$ 400 kW & 6,5&7.0 \\
		& & $400 \, \text{kW} \leq \text{PA} < 1500 \, \text{kW}$ & 7.5 &8.0 \\
		& & PA $>$ 1500 kW & 8.0 &8.5 \\
		\bottomrule
	\end{tabular}
	\label{tabla:enfriadores-alta}
\end{table}

\section{Restricciones técnicas internas y externas} \rsp
 \subsection{Factor de Seguridad}
 Después de calcular las cuatro principales fuentes de calor, se agrega un factor de seguridad del 10\% a la carga total de refrigeración. Este factor se incorpora para considerar posibles omisiones o inexactitudes mínimas en los cálculos, proporcionando así una capa adicional de seguridad o reserva que puede provenir del rendimiento del compresor y las fluctuaciones en la carga promedio.
 \subsection{Dimensionamiento y Aislamiento de la Cámara de Refrigeración}
 
 Para realizar el dimensionamiento y aislamiento de la cámara de refrigeración en la Unidad Médica Familiar 40 en Azcapotzalco, CDMX, se seguirán las normas oficiales de México que regulan este tipo de instalaciones, asegurando así el cumplimiento de estándares de calidad y seguridad. A continuación, se detalla el proceso:
 
 \begin{enumerate}
 	\item \textbf{Determinación de requisitos}: Se comenzará identificando los requisitos específicos de la cámara de refrigeración para el almacenamiento de insulina en la Unidad Médica Familiar 40. Esto incluirá el volumen de almacenamiento necesario, la temperatura requerida para mantener la integridad de la insulina, y cualquier otro requisito relevante.
 	
 	\item \textbf{Cálculo de espacio y distribución interna}: Basado en los requisitos identificados, se calculará el espacio necesario y se diseñará la distribución interna de la cámara. Se determinará la disposición de las estanterías y los espacios necesarios para el almacenamiento eficiente de la insulina, teniendo en cuenta factores como el acceso fácil y la circulación del aire.
 	
 	\item \textbf{Selección de materiales y equipos}: Se seleccionarán los materiales adecuados para las paredes, techo y suelo de la cámara, teniendo en cuenta las normas oficiales de México sobre conductividad térmica, resistencia a la humedad y resistencia estructural. Además, se elegirán los equipos de refrigeración, como el evaporador, compresor y condensador, garantizando su compatibilidad con las necesidades de temperatura y espacio de la cámara.
 	
 	\item \textbf{Diseño conceptual y distribución de equipos}: Con base en los cálculos y selecciones anteriores, se elaborará un diseño conceptual detallado de la cámara de refrigeración, se usará software de diseño de ingeniería. Esto incluirá la distribución de los equipos de refrigeración dentro de la cámara, así como la ubicación de la puerta y componentes para facilitar el acceso y la ventilación adecuada.
 	
 	\item \textbf{Consideraciones adicionales}: Se tendrán en cuenta otras consideraciones importantes, como la instalación de sistemas de monitoreo de temperatura y humedad, la instalación de sistemas de respaldo de energía para evitar interrupciones en el suministro de energía, y la implementación de medidas de seguridad para proteger tanto la insulina almacenada como el personal que trabaja en la unidad médica.
 \end{enumerate}

 Y para empezar con el dimensionamiento de la cámara frigorífica, se da a conocer 
 las medidas de cada empaque de la insulina en sus dos versiones, consulte las tablas \ref{tabla:humalog} y \ref{tabla:lantus}, también el peso que puede contener cada
 empaque con el fin de almacenar una cantidad grande de insulina o refrigerar al menos la cantidad promedio de medicamento que llega a la unidad.\\
 Con dicho calculo se procederá a hacer un diseño conceptual, de la cámara donde
 especificará el espacio que se ocupara, la distribución de la estantería, el producto
 a almacenar, así como indicar donde se colocaran las puertas y ventanas, para
 después colocar la distribución del equipo de refrigeración, que son el evaporador,
 compresor, condensador, y todos los equipos adicionales que se requieran para el
 óptimo funcionamiento de nuestra cámara frigorífica.\\
 Después de recabar la información del dimensionamiento obtenemos las cargas térmicas para el equipo en base a la sección 3.3.2 en este capítulo. Además consideramos las cargas que afectan la temperatura del sistema, las cuales son
 
 \subsubsection{Cargas por conducción en paredes y techo}
 
 Los paredes y techos de las cámaras frigoríficas suelen ser superficies planas constituidas por varias capas de materiales (materiales estructurales + capas de aislantes térmicos). Las superficies externa e interna de los parámetros están en contacto con un fluido en movimiento que es el aire. Por tanto, en las superficies externas e internas de los parámetros se producirán transferencias de calor por convección. Mientras que la transferencia de calor entre los materiales que conforman las capas internas se realizará por conducción. El calor que se transfiere por cada parámetro se calcula con la ecuación 1 donde $A$ es el área del mismo, $T_{ext}$ y $T_{int}$ son las temperaturas en el interior y exterior de la cámara en el lado de ese parámetro, y $Kg$ es una constante de proporcionalidad que se denomina coeficiente global de transmisión. Sus unidades en el SI son $  {W m}^{-2}  {K}^{-1} $. Para $n$ capas de materiales en el parámetro se tendrá:
 \begin{equation}
 	\dot{Q}_p = Kg \cdot A \cdot (T_{ext} - T_{int})
 \end{equation}
 \begin{equation}
 	Kg = \dfrac{1}{\frac{1}{h_{p.int}} + \frac{e_1}{k_1} + \frac{e_2}{k_2} + \dots + \frac{e_n}{k_n} + \frac{1}{h_{p.ext}}}  =\dfrac{1}{  \frac{1}{h_{p.int}} + \frac{1}{h_{p.ext}} +\sum_{i=1}^{n} \frac{e_i}{k_i}   }
 \end{equation}
 Donde $e_i$ es el espesor de cada capa que conforma el parámetro, $k_i$ es la conductividad térmica de cada material, $h_{ext}$ y $h_{int}$ son los coeficientes de transferencia de calor por convección exterior e interior.\\
  Para calcular el calor transferido por una pared, la superficie del mismo $A$ siempre es conocida; lo mismo la diferencia de temperaturas entre el ambiente de un lado y del otro, que son valores de diseño. \\
 Por tanto, es necesario fijar el espesor y la conductividad de los materiales que conforman las distintas capas, junto con los coeficientes de transferencia por convección de ambos lados del parámetro. Los coeficientes de película en las superficies de cámaras frigoríficas suelen ser en el interior de $7{.}5 \, {W m}^{-2} \, {K}^{-1}$ y en el exterior $25 \, {W m}^{-2} \, {K}^{-1}$.
 
 \begin{figure}[H]
 	\centering
 	\includegraphics[width=0.45\textwidth]{figures/paredes-refri.png}
 	\caption{Vista general de la cámara de refrigeración} Fuente: \cite{intarcon-2023}
 	\label{fig:paredes-refri}
 \end{figure}
 
 En la figura \ref{fig:paredes-refri} las flechas indican el intercambio de calor producido por los componentes y el aire del sistemas
 
 
 \subsubsection{Cargas por infiltración de aire }
 
 El aire que entra en la cámara desde el exterior estará a una temperatura más alta que el aire que está en el interior. Por tanto, el sistema de refrigeración debe tener potencia suficiente para bajar la temperatura del aire de entrada hasta la temperatura de la cámara, vea la tabla \ref{tabla:renov-air} para tener una idea de cuántas veces en promediose renueva el aire en una cámara al día. Esa potencia se calculará con el producto del flujo másico de aire de renovación \( \dot{m}_{aire} \) (kg/s) por la diferencia de entalpías específicas en las condiciones correspondientes (\( h_{aire_{ext}} - h_{aire_{int}} \)) (kJ/kg).
 \begin{equation}
 	\dot{Q}_{aire} = \dot{m}_{aire} \cdot (h_{aire_{ext}} - h_{aire_{int}})
 \end{equation}
 La entalpía del aire será la suma de la correspondiente al aire seco y la del vapor de agua mezclado con el mismo:
 \begin{equation}\label{eq:entalp-air}
 	h_{aire \, húmedo} = h_{aire} + \omega_{aire} \cdot h_{vapor}
 \end{equation}
 La entalpía específica del aire a presión atmosférica se puede calcular como el producto del calor específico a esa presión por la temperatura (\( C_{p_{aire}} = 1.005 \, \text{kJ/kg} \, ^\circ \text{C} \)):
 \begin{equation}
 	h_{aire} = C_{p_{aire}} \cdot T_{aire}
 \end{equation}
 \begin{equation}
 	h_{aire} = 1.005 \cdot T \, \text{kJ/kg} \, ^\circ \text{C} 
 \end{equation}
 La entalpía del vapor por unidad de masa de aire seco se calcula como el producto de la humedad específica del aire (\( \omega_{aire} \)) por la suma del calor latente absorbido en el proceso de evaporación a presión atmosférica (\( \lambda = 2503 \, \text{kJ/kg} \, \text{agua} \)) y el calor específico a esa presión por la temperatura:
 \begin{equation}
 	h_{vapor} = \omega_{aire} \cdot (\lambda + C_{p_{vapor}} \cdot T)
 \end{equation}
 \begin{equation}
 	\lambda = 2503 \, \text{kJ/kg} \, \text{agua}
 \end{equation}
 \begin{equation}
 	C_{p_{vapor}} = 1.86 \, \text{kJ/kg} \, \text{agua}\, ^\circ \text{C} 
 \end{equation}
 De ahí se obtiene la Ecuación \ref{eq:entalp-air} por la que se puede calcular la entalpía del aire húmedo a partir de su temperatura y su humedad absoluta:
 \begin{equation}
 	h_{aire} = C_{p_{aire}} \cdot T + \omega_{aire} \cdot (\lambda + C_{p_{vapor}} \cdot T)
 \end{equation}
 Si se desea expresar la entalpía en relación al volumen de aire, se debe multiplicar por la densidad del mismo \( \rho \):
 \begin{equation}
 	hV_{aire} = (C_{p_{aire}} \cdot T_{vapor} + \omega_{aire} \cdot (\lambda + C_{p_{vapor}} \cdot T_{vapor})) \cdot \rho
 \end{equation}
 La densidad se  calcula con la ecuación:
 \begin{equation}
 	\rho = \frac{PM_{aire}}{R \cdot T}
 \end{equation}
 Donde: \( PM_{aire} \) es el peso molecular del aire (28.93 g/mol), \( T \) es la temperatura del aire en K, y \( R \) es la constante universal de los gases (0.082 litros Pa/mol K).\\
 Sabiendo el volumen que se renueva periódicamente en la cámara, se puede calcular el calor empleado en el enfriamiento del aire mediante la ecuación:
 \begin{equation}
 	\dot{Q}_{aire} = n \cdot V \cdot (hV_{aire_{ext}} - hV_{aire_{int}})
 \end{equation}

 
 \begin{table}[H]
 	\centering
 	\caption{Estimación del número de veces que se renueva el aire en las cámaras por día}\cite{Martin}
 	\begin{tabular}{cccccc}
 		\toprule
 		Renovaciones por día & Renovaciones por día & Volumen (m\textsuperscript{3}) & T <0ºC & T >0ºC & Volumen (m\textsuperscript{3}) \\
 		\midrule
 		2,5 & 52 & 70 & 100 & 6,8 & 9 \\
 		3 & 47 & 63 & 150 & 5,4 & 7 \\
 		4 & 40 & 53 & 200 & 4,6 & 6 \\
 		5 & 35 & 47 & 250 & 4,1 & 5,3 \\
 		7,5 & 28 & 38 & 300 & 3,7 & 4,8 \\
 		10 & 24 & 32 & 400 & 3,1 & 4,1 \\
 		15 & 19 & 26 & 500 & 2,8 & 3,6 \\
 		20 & 16,5 & 22 & 600 & 2,5 & 3,2 \\
 		25 & 14,5 & 19,5 & 800 & 2,1 & 2,8 \\
 		30 & 13 & 17,5 & 1000 & 1,9 & 2,4 \\
 		40 & 11,5 & 15 & 1500 & 1,5 & 1,95 \\
 		50 & 10 & 13 & 2000 & 1,3 & 1,65 \\
 		60 & 9 & 12 & 2500 & 1,1 & 1,45 \\
 		80 & 7 & 10 & 3000 & 1,05 & 1,05 \\
 		\bottomrule
 	\end{tabular} \label{tabla:renov-air}
 \end{table}
 \subsubsection{Cargas del producto a refrigerar}
 La carga térmica más importante en una cámara proviene de los productos que se pretenden refrigerar. El calor empleado en el proceso depende de la temperatura de entrada de los productos y la temperatura a la que se desea realizar la conservación en frío. Este calor se puede dividir en cuatro partes:
 \begin{enumerate}
 	\item Calor de enfriamiento por encima del punto de congelación:
  \begin{equation}
 	\dot{Q}_{\text{enf1}} = \dot{m} \cdot C_e \cdot (T_e - T_{\text{cong}})
 \end{equation}
 \item Calor de congelación

 \begin{equation}
 	\dot{Q}_{\text{cog}} = \dot{m} \cdot L_{\text{cong}}
 \end{equation}
 
 El calor latente de congelación de un producto ($L_{\text{cong}}$) puede ser obtenido en la bibliografía p.ej \cite{bohn}, donde existen numerosas tablas para diferentes tipos de productos; o puede ser estimado a partir de su contenido de agua con la siguiente fórmula: $L_{\text{cong}} = 3{.}335 \cdot a$
 \item Calor de enfriamiento por debajo del punto de congelación
 \begin{equation}
 	\dot{Q}_{\text{enf2}} = \dot{m} \cdot C_{e\text{cong}} \cdot (T_{\text{cong}} - T_f)
 \end{equation}
 \item Calor de respiración del producto
 \begin{equation}
 	\dot{Q}_{\text{resp}} = \dot{m} \cdot L_{\text{resp}}
 \end{equation}
\end{enumerate}
 
 En \citeNP{Martin} se mencionan otras cargas que se podrían considerar como secundarias, pero no podemos excluirlas del cálculo son:
\begin{itemize}
	\item  Carga por radiación solar
	\item Carga por ocupación (personal laborando)
	\item Por iluminación
	\item  Por motores
	
\end{itemize}

\subsection{Sistema de refrigeración}
\subsubsection{Selección del evaporador}
Luego se continua con la elección del evaporador, para ello debemos conocer:
\begin{itemize}
	\item Temperatura Saturada de Succión \begin{equation}
		T_{sat.succ} = T_{camara} - 10^\circ F
	\end{equation}
	\item Temperatura Saturada de Evaporación \begin{equation}
		T_{sat.succ} =T_{sat.evap}
	\end{equation}	
\end{itemize}

\begin{figure}[H]
	\centering
	\includegraphics[width=0.5\linewidth]{figures/evap-func}
	\caption{Funcionamiento del evaporador}
	\label{fig:evap-func}
\end{figure}
\begin{figure}[H]
	\centering
	\includegraphics[width=0.576\linewidth]{figures/evap-cad1}\includegraphics[width=0.5\linewidth]{figures/evap-cad2}
	\caption{Vistas de un sistema de evaporador. Creado en AutoCAD, basado en} \cite{bibliocad}
	\label{fig:evap-cad}
\end{figure}
 
 
 Con este dato, y el dato de los BTU de requerimiento de la carga térmica, se selecciona el evaporador más conveniente, también  considerando el diseño espacial de la cámara frigorífica, en donde podremos elegir  entre los siguientes tipos de evaporador, vea las figuras \ref{fig:evap-func} y \ref{fig:evap-cad}, para comprender los tipos de evaporador que se mencionan aquí:(a) Evaporadores perfil bajo; (b) Evaporadores perfil medio; (c) Evaporadores perfil alto; (d) Evaporadores perfil industrial y (e) Evaporadores de techo.
 
 \subsubsection{Selección de motor}
 
 Una vez seleccionado el evaporador y determinado el número de motores-ventilador, procedemos a calcular el calor que estos generan. Si no se conocen las cargas debidas a los motores, se pueden hacer las siguientes suposiciones:
 \begin{itemize}
 	\item Un motor de 1 HP por cada 16,000 ft\textsuperscript{3} en cámaras de refrigeración.\vspace*{-0.2cm}
 	\item Un motor de 1 HP por cada 12,500 ft\textsuperscript{3} en cámaras de congelación.
 \end{itemize}\vspace*{-0.2cm}
 La selección de un motor adecuado implica considerar varios factores para garantizar un rendimiento óptimo en una aplicación específica, como los requisitos de la aplicación, el tipo de motor, la eficiencia y el rendimiento. Esto se facilita mediante la tabla \ref{tabla:motores}.
 

 
 
 Entonces, para calcular el calor emitido por los motores del evaporador, usamos la fórmula:
 \begin{equation}\label{eq:emision_calor_motores}
 	Q_{EME} = \text{No. de Motores en HP} \times \text{Calor equivalente} \times 24
 \end{equation}
 Con esta carga recalculamos la carga térmica parcial agregándole este valor, para luego agregarle el factor de seguridad del 10\%, y al finalizar, multiplicarlo por el número de horas estimadas.
 
 Cuando se ha determinado la carga térmica total por hora, se puede seleccionar el equipo basado en la información obtenida del trabajo inicial de reconocimiento. Algunos otros factores que afectan la selección del equipo son:\vspace*{-0.2cm}
 \begin{itemize}
 	\item Balance del equipo\vspace*{-0.2cm}
 	\item Diferencial de temperatura del sistema (°DT)\vspace*{-0.2cm}
 	\item Control de la capacidad/seguridad del producto\vspace*{-0.2cm}
 	\item Tipo de operación/flujo del aire.
 \end{itemize}\vspace*{-0.2cm}
 Balance del equipo. La capacidad de la unidad condensadora debe ser seleccionada a una temperatura saturada de succión que esté balanceada con el evaporador(es) a un diferencial de temperatura entre el refrigerante en el evaporador y el aire en la cámara de almacenamiento refrigerado   \cite{intarcon-2023}.
  \begin{table}[H]
 	\centering
 	\caption{Selección de motores de acuerdo a su carga} Fuente: \cite{suarez-2022}\\
 	\begin{tabular}{ccccc}
 		\hline
 		HP      & Tipo de        & RPM     & Eficiencia a                    & BTU/hr \\
 		Nominal & motor          & Nominal & \multicolumn{1}{r}{Plena Carga} &        \\ \hline
 		0.5     & Polo Sombreado & 1500    & 35                              & 360    \\
 		0.08    &                &         &                                 & 580    \\
 		0.125   &                &         &                                 & 900    \\
 		0.16    &                &         &                                 & 1160   \\
 		0.25    & Fase dividida  & 1750    & 54                              & 1180   \\
 		0.33    &                &         & 56                              & 1500   \\
 		0.5     &                &         & 60                              & 2120   \\
 		0.75    & Tres fases     & 1750    & 72                              & 2650   \\ \hline
 	\end{tabular}\\
 	\label{tabla:motores}
 	Nota: Los evaporadores de bajo perfil ADT o BME usan motores EBM. El calor que emiten es equivalente a 273.73 BTU/hr
 \end{table}
\subsubsection{Bases teóricas de la refrigeración.}
\subsubsection{Energía.}
La energía $E$ es la capacidad para realizar trabajo. La energía de un sistema consta de energías internas $U$, cinéticas $K$ y potenciales $P$. La energía interna se compone de energías térmicas (sensible y latente), químicas y nucleares. A menos que haya una reacción química o nuclear, el cambio interno de un sistema se debe a un cambio en la energía térmica. El cambio total de energía de un sistema se expresa como:
\begin{equation}
	\Delta E = E_2 - E_1 = \Delta U + \Delta KE + \Delta PE
	\label{eq:energia_total}
\end{equation}
Para la mayoría de los casos, las energías cinética y potencial no cambian durante un proceso y el cambio de energía se debe al cambio en la energía interna:
\begin{equation}
	\Delta E = \Delta U = m(u_2 - u_1)
	\label{eq:energia_interna}
\end{equation}
La energía se mide en kJ o Btu (1 kJ = 0.94782 Btu). La energía por unidad de tiempo es la tasa de energía y se expresa como:
\begin{equation}
	\dot{E} = \frac{E}{t} \quad (\text{kW o Btu/h})
	\label{eq:energia_tasa}
\end{equation}
La unidad de la tasa de energía es kJ/s, que es equivalente a kW o Btu/h (1 kW = 3412.14 Btu/h). La energía por unidad de masa se llama energía específica; tiene la unidad de kJ/kg o Btu/lbm (1 kJ/kg = 0.430 Btu/lbm).
\begin{equation}
	e = \frac{E}{m} \quad (\text{kJ/kg o Btu/lbm})
	\label{eq:energia_especifica}
\end{equation}
La energía puede transferirse hacia o desde un sistema en tres formas: masa, calor y trabajo. Se describen brevemente en las siguientes secciones.

\subsubsection{Primera ley de la termodinámica}

Es bien sabido que la termodinámica es la ciencia de la energía y la entropía, y que la base de la termodinámica es la observación experimental. En la termodinámica, tales observaciones se formularon en cuatro leyes básicas de la termodinámica llamadas la cero, primera, segunda y tercera leyes de la termodinámica. Las primeras y segundas leyes de la termodinámica son las herramientas más comunes en la práctica, debido al hecho de que las transferencias y conversiones de energía están gobernadas por estas dos leyes, y en este capítulo nos enfocamos en estas dos leyes.

La primera ley de la termodinámica (PLT) se puede definir como la ley de la conservación de la energía, y establece que \textit{la energía no puede ser creada ni destruida}. Se puede expresar para un sistema general como el cambio neto en la energía total de un sistema durante un proceso es igual a la diferencia entre la energía total que entra y la energía total que sale del sistema:
\begin{equation}
	E_{\text{in}} - E_{\text{out}} = \Delta E_{\text{system}}
	\label{eq:primera_ley_energia}
\end{equation}
En forma de tasa,
\begin{equation}
	\dot{E}_{\text{in}} - \dot{E}_{\text{out}} = \Delta \dot{E}_{\text{system}}
	\label{eq:primera_ley_energia_tasa}
\end{equation}
Para un sistema cerrado que experimenta un proceso entre estados inicial y final que involucran interacciones de calor y trabajo con los alrededores (Figura \ref{fig:sistema_cerrado}),
\begin{equation}
	E_{\text{in}} - E_{\text{out}} = \Delta E_{\text{system}}
	\label{eq:primera_ley_energia_sistema_cerrado}
\end{equation}
$(Q_{\text{in}} + W_{\text{in}}) - (Q_{\text{out}} + W_{\text{out}}) = \Delta U + \Delta KE + \Delta PE$
\begin{figure}[H]
	\centering
	\begin{tikzpicture}
		% Dibujo del sistema cerrado
		\draw (0,0) rectangle (4,3) node[pos=.5] {System};
		
		% Etiquetas de los estados
		%	\node at (2,2.5) {Estado 1};
		%	\node at (2,0.5) {Estado 2};
		
		% Etiquetas de las interacciones
		\node at (-1,1.5) {$W_{in}$};
		\node at (5,1.5) {$W_{out}$};
		\node at (2,4) {$Q_{in}$};
		\node at (2,-1) {$Q_{out}$};
		
		
	\end{tikzpicture}
	\caption{Representación de un sistema cerrado con interacciones de calor $Q$ y Trabajo $W$ sin intercambio de masa.  }(Fuente: Propia, basado de \cite{james-2013}) 
	\label{fig:sistema_cerrado}
\end{figure}
Si no hay cambio en las energías cinética y potencial,
\begin{equation}
	(Q_{\text{in}} + W_{\text{in}}) - (Q_{\text{out}} + W_{\text{out}}) = \Delta U = m(u_2 - u_1)
	\label{eq:primera_ley_energia_sistema_cerrado_2}
\end{equation}
Consideremos un volumen de control que involucra un proceso de flujo estable. La masa está entrando y saliendo del sistema y hay interacciones de calor y trabajo con los alrededores (Figura \ref{fig:sistema_abierto}). Durante un proceso de flujo estable, el contenido total de energía del volumen de control permanece constante, y por lo tanto, el cambio total de energía del sistema es cero. Entonces, la PLT puede expresarse como:
\begin{equation}
	\dot{E}_{\text{in}} - \dot{E}_{\text{out}} = \Delta \dot{E}_{\text{system}} = 0
\end{equation}
\begin{equation}
	\dot{E}_{\text{in}} = \dot{E}_{\text{out}}
\end{equation}
\begin{equation}
	\dot{Q}_{\text{in}} + \dot{W}_{\text{in}} + \dot{m}h_{\text{in}} = \dot{Q}_{\text{out}} + \dot{W}_{\text{out}} + \dot{m}h_{\text{out}}
\end{equation}
Aquí, las energías cinética y potencial son despreciadas.


\begin{figure}[H]
	\centering
	\begin{tikzpicture}
			\tikzstyle{arrow} = [thick,->,>=stealth]
		\tikzstyle{line} = [draw, -latex']
		\draw (0,0) rectangle (4,3) node[pos=.5] {Sistema};
		
		% Etiquetas de las interacciones
		\node at (-1,1.5) {$W_{\text{in}}$};
		\node at (5,1.5) {$W_{\text{out}}$};
		\node at (2,4) {$Q_{\text{in}}$};
		\node at (2,-1) {$Q_{\text{out}}$};
		
		% Flechas de las interacciones s.abierto
		
		
			\path [line] (-0.5,1.5) -- (0.5,1.5);
			\path [line] (4.5,1.5) -- (3.5,1.5);
			\path [line](2,2) -- (2,2.9);
			\path [line] (2,0) -- (2,1);
		
		% Línea de cambio de masa
		%                       izq           derecha
			\path [line] (-3,2.5) -- (1,2);
		\path [line] (6,3) -- (3,2);
		
		% Etiqueta de cambio de masa
		\node at (-3,2.8) {$\dot{m}$};
		\node at (6,3.2) {$\dot{m}$};
	\end{tikzpicture}
	\caption{Representación de un sistema abierto con interacciones de calor $Q$ y Trabajo $W$. Los intercambios de masa son las líneas con la $\dot{m}$ fuera del sistema. }(Fuente: Elaboración propia, basado de \cite{james-2013})
	\label{fig:sistema_abierto}
\end{figure}

\subsubsection{Segunda ley de la termodinámica}
La mayoría de los procesos de refrigeración y bombas de calor operan de manera cíclica, devolviendo periódicamente el refrigerante a un estado dado. Por lo tanto, podemos decir que durante el ciclo el cambio neto en la energía interna es cero, es decir:
\begin{equation}
	0 = \Delta \int_{\text{cycle}} U \equiv \oint\Delta U = 0
\end{equation}
y por lo tanto, la primera ley para un proceso cerrado y cíclico puede escribirse como:
\begin{equation}
	W_\text{net }  = Q_\text{net } 
\end{equation}
En los ciclos de refrigeración y bomba de calor, el trabajo se realiza solo durante una etapa del ciclo, mientras que el calor se transfiere durante dos etapas, una a baja temperatura y otra a alta temperatura (las temperaturas son "altas" y "bajas" relativas entre sí, en lugar de términos absolutos). Por lo tanto, la forma más útil de la Primera Ley para analizar ciclos de refrigeración o bomba de calor es:
\begin{equation}
	   W =Q_H -Q_C
\end{equation}
donde $W$ es el trabajo realizado sobre el refrigerante, y los subíndices $H$ y $C$ se refieren a los procesos de transferencia de calor de temperatura alta (caliente) y baja (fría), respectivamente.


\subsubsection*{Afirmaciones de la Segunda Ley}

La Segunda Ley de la Termodinámica es quizás ligeramente más abstracta y difícil de entender que la Primera Ley. Parafraseando a Rudolf Clausius (1822 a 1888), la Segunda Ley establece que:
\begin{center}
		\textit{$"$Es imposible construir un dispositivo que opere de manera cíclica cuyo único propósito sea transferir calor de un depósito de temperatura baja a un depósito de temperatura alta.$"$}
\end{center}
A primera vista, esta afirmación podría sugerir que la refrigeración (que es un proceso que tiene el efecto neto de transferir calor de una temperatura baja a una temperatura alta) es una imposibilidad física; sobre este punto más adelante.

Para obtener una expresión matemática de la Segunda Ley, primero necesitamos recordar la definición de entropía (como se usa en la termodinámica clásica) cortesía de Clausius:
\begin{equation}
	  \Delta S =\int \frac{Q_{rev}}{T}  
\end{equation}

También, la Segunda Ley puede escribirse como:
\begin{equation}\label{eq:sec-law-system}
  \Delta S_{\text{universe}} = \Delta S_{\text{system}} - \Delta S_{\text{surrounding}} \geq 0	
\end{equation}

Podemos expresar verbalmente la Ecuació \ref{eq:sec-law-system} diciendo que para cualquier proceso, la entropía del universo solo puede aumentar, o en el caso límite (procesos reversibles) puede permanecer sin cambios. La Segunda Ley establece que la entropía del universo nunca puede disminuir. \cite{james-2013}


\subsubsection{Formalismo de Refrigeración}

Un refrigerador es un dispositivo utilizado para transferir calor de un medio de baja temperatura a uno de alta temperatura. Son dispositivos cíclicos. La Figura \ref{fig:ref-comp} muestra el esquema de un ciclo de refrigeración por compresión de vapor (el tipo más común). Un fluido de trabajo (llamado refrigerante) ingresa al compresor en estado de vapor y es comprimido a la presión del condensador. El refrigerante de alta temperatura se enfría en el condensador al rechazar calor a un medio de alta temperatura. El refrigerante ingresa al dispositivo de expansión en estado líquido. Se expande en el dispositivo de expansión y su presión y temperatura disminuyen. El refrigerante es una mezcla de vapor y líquido en la entrada del evaporador. Absorbe calor de un medio de baja temperatura a medida que fluye en el evaporador. \cite{boles-2010}\\
El ciclo se completa cuando el refrigerante sale del evaporador en estado de vapor y entra en el compresor. 
Un balance de energía para un ciclo de refrigeración, basado en la ecuación \ref{eq:primera_ley_energia} de la PLT, da como resultado:
\begin{equation}
	QH = QL + W \label{eq:energy_balance}
\end{equation}

El indicador de eficiencia para un ciclo de refrigeración es el coeficiente de rendimiento (COP), que se define como el calor absorbido del espacio enfriado dividido por el trabajo de entrada en el compresor:

\begin{equation}
	COP_R = \frac{QL}{W} \label{eq:COP_definition}
\end{equation}

Esto también se puede expresar como:

\begin{equation}
	COP_R = \frac{QL}{QH - QL} = \frac{1}{QH / QL - 1} \label{eq:COP_expression}
\end{equation}

\begin{figure}[H]
	\centering
	\includegraphics[width=0.7\textwidth]{figures/ref-comp.png}
	\caption{Diagrama esquemático del ciclo de refrigeración por compresión de vapor.} Fuente: \cite{theoms-2023}
	\label{fig:ref-comp}
\end{figure} 
\subsection{Distribución del Aire}
El diseño de sistemas HVAC a menudo se basa en decisiones predeterminadas, como suministrar agua fría a 5°C o aire acondicionado a 13°C, debido a su efectividad comprobada. Sin embargo, estas decisiones tienen implicaciones directas en los costos de tuberías, conductos y equipos, lo que afecta el gasto de capital y la comodidad \cite{hvac-2016}
\begin{itemize}
	\item Equipos de manejo de aire más pequeños reducen el gasto de capital y liberan espacio adicional debido a su huella más pequeña.
\item La distribución de aire frío puede ahorrar espacio significativo. Por ejemplo, un manejador de aire típico que necesita 34,485 pies cúbicos por minuto (cfm) para aire a 13°C solo requiere 20,900 cfm para aire a 7°C. Esto reduce el área frontal de la bobina y permite conductos más pequeños y sencillos.
\item| Conductos más pequeños significan menos chapa metálica, instalación más fácil y más espacio en el techo para bandejas de cables. Además, los conductos redondos permiten velocidades de aire más altas y reducen aún más el tamaño del conducto.
\item Una altura de piso a piso más corta, gracias a los conductos más pequeños, puede reducir los costos de vidrio y acero en edificios de varios pisos, e incluso agregar un piso de espacio alquilable.
\item Una menor potencia del ventilador reduce los costos de instalación eléctrica y operativos durante la vida útil del edificio, y puede disminuir el ruido del ventilador \cite{tecfrinor}.
\end{itemize}
El cálculo del flujo de aire se realiza multiplicando el área total de la habitación por la altura del techo y el número de cambios de aire por hora (ACH), y dividiendo todo entre 60 minutos para obtener los CFM (pies cúbicos por minuto) \cite{bercofred-2021}.

\begin{equation}
	 {CFM} = \frac{ {Área} \times  {Altura} \times  {ACH}}{60}
\end{equation}

En una cámara refrigerada estándar, la humedad relativa (HR) generalmente se mantiene entre el 90\% y el 100\%. Cuando la HR alcanza niveles del 90\% al 100\%, la formación de condensación es casi inevitable. La presencia no deseada de humedad en el aire crea un ambiente propicio para el crecimiento de contaminantes, lo que puede representar una amenaza para los productos almacenados.
\subsection{Dimensionamiento de puerta}

Las puertas desempeñan un papel importante en el equilibrio termodinámico de la cámara de refrigeración, ya que representan puntos de entrada y salida de aire que pueden afectar significativamente la eficiencia energética y la estabilidad térmica de la cámara. Especialmente en el caso de la refrigeración de insulina, donde se requiere un control preciso de la temperatura, las puertas deben diseñarse y ubicarse estratégicamente para minimizar la pérdida de frío y mantener un ambiente interno constante y seguro para el almacenamiento de medicamentos sensibles a la temperatura. Por lo tanto, elegir puertas adecuadas y gestionar su uso de manera eficiente es fundamental para garantizar un rendimiento óptimo del sistema de refrigeración y la integridad de los productos almacenados. 
 
 
 

\section{Lineamientos de regulación del sistema artefactual (Normativas y regulaciones)}

Las Normas Oficiales Mexicanas (NOM) son leyes técnicas de obligado cumplimiento emitidas por las autoridades competentes, cuyo objetivo es establecer las condiciones que deben cumplirse si un proceso o servicio representa un riesgo para la seguridad humana, o pone en peligro la salud humana. También contiene información sobre los términos y referencias para el cumplimiento y aplicación de estos términos \cite{salud-2022}.
\subsection{Normas oficiales Mexicanas}
\begin{itemize}
	\item NOM-12-ENER-2019, Eficiencia energética de unidades condensadoras y evaporadoras para refrigeración. Límites, métodos de prueba y etiquetado.
	\item NOM-008-SCFI-2002, Sistema General de Unidades de Medida.
	\item NOM-197-SSA1-2000, Que establece los requisitos mínimos de infraestructura y equipamiento de hospitales y consultorios de atención médica especializada.
	\item NOM-028-STPS-2012, Sistema para la administración del trabajo-Seguridad en los procesos y equipos críticos que manejen sustancias químicas peligrosas.
	\item NOM-025-STPS-2008, Condiciones de iluminación en los
	centros de trabajo. Esta norma establece los requisitos para la iluminación en los
	centros de trabajo, con el fin de garantizar la seguridad y salud de los trabajadores.
\end{itemize}
Para mayor detalles del contexto normativo revise la \hyperref[sec:contexto-normativo]{sección 2.2}.
De acuerdo a la NOM-025-STPS-2008 se establecen los siguientes requisitos:


\begin{itemize}
	\item La iluminación debe ser suficiente para permitir a los trabajadores realizar sus actividades sin riesgos.
	\item Se requiere que la iluminación sea uniforme y sin deslumbramientos.
	\item La iluminación debe adaptarse adecuadamente al tipo de actividad que se realiza en el centro de trabajo.
\end{itemize}

\subsection{Recomendaciones para Cámaras Frigoríficas}

Además de cumplir con los requisitos normativos, se sugieren las siguientes recomendaciones para garantizar una iluminación adecuada en cámaras frigoríficas:

\begin{itemize}
	\item Emplear luminarias altamente eficientes en términos energéticos.
	\item Ajustar la altura de las luminarias para lograr una distribución uniforme de la luz.
	\item Seleccionar luminarias con un ángulo de apertura apropiado para el área que se va a iluminar.
\end{itemize}

\subsection{Consideraciones Específicas para Cámaras Frigoríficas}

Al diseñar e instalar el sistema de iluminación en cámaras frigoríficas, se deben tener en cuenta consideraciones específicas:

\begin{itemize}
	\item La temperatura dentro de la cámara puede afectar el rendimiento de las luminarias, por lo que es crucial seleccionar aquellas diseñadas para funcionar en entornos fríos.
	\item La humedad también puede influir en el rendimiento de las luminarias, por lo que se deben elegir luminarias resistentes a la humedad.
	\item La condensación es un problema potencial en estas cámaras, por lo que se deben seleccionar luminarias diseñadas para prevenir la acumulación de humedad.
\end{itemize}

\subsection{Normativa y requisitos para cámaras frigoríficas comerciales}

La Norma Oficial Mexicana NOM-001-SCFI-2018 establece requisitos de seguridad y métodos de prueba para aparatos electrónicos, incluyendo cámaras frigoríficas comerciales. Esta norma incluye los siguientes requisitos específicos:

\begin{itemize}
	\item \textbf{Protección contra Riesgos Eléctricos:} Se requiere un interruptor diferencial de alta sensibilidad para proteger los circuitos eléctricos, y los aparatos deben conectarse a tomas de corriente con conexión a tierra.
	\item \textbf{Protección contra Riesgos Mecánicos:} Las puertas y ventanas deben estar diseñadas para prevenir atrapamientos y lesiones, y deben contar con sistemas de cierre seguro.
	\item \textbf{Protección contra Riesgos Térmicos:} Se exige un adecuado sistema de aislamiento térmico para evitar quemaduras y exposición a temperaturas extremas.
	\item \textbf{Protección contra Riesgos de Incendio:} Se requiere un sistema de alarma de fugas de refrigerante y un sistema de extinción de incendios adecuado.
	\item \textbf{Seguridad de Puertas:} Las puertas deben poder abrirse desde el interior sin necesidad de llave, incluso si tienen cerraduras externas. En temperaturas bajo cero, se deben instalar dispositivos de calentamiento para evitar la congelación de las puertas.
\end{itemize}
\section{Conceptualización De Sistema Artefactual (bosquejo general)}
El diseño inicial del equipo a diseñar representará de forma muy simplificada las ideas subyacentes deñ diseño. Para el diseño preliminarse hará uso de Inventor 2025 siguiendo alguno de estos pasos.
\begin{enumerate}
	\item \textbf{Definir Dimensiones:} Establecer las dimensiones generales hipotéticas de la cámara de refrigeración, incluyendo altura, anchura y profundidad.
	\item \textbf{Diseñar Paredes Aisladas:} Utilizar las herramientas de modelado para diseñar las paredes aisladas de la cámara, asegurándose de considerar el material aislante adecuado.
	\item \textbf{Ubicar Unidades de Refrigeración:} Colocar las unidades de refrigeración dentro del espacio de la cámara, asegurándose de dejar suficiente espacio para la circulación del aire.
	\item \textbf{Agregar Puntos de Acceso:} Incluir en el diseño posibles puntos de acceso, como puertas o paneles desmontables, asegurándose de que estén ubicados estratégicamente para facilitar el acceso al interior de la cámara.
	\item \textbf{Revisión y Ajustes:} Revisar el bosquejo preliminar para asegurarse de que cumple con los requisitos y especificaciones del diseño.
\end{enumerate}

El bosquejo preliminar (ver figura \ref{fig:design1}) proporciona una representación inicial de la disposición y estructura general de la cámara de refrigeración, permitiendo visualizar y comunicar conceptos antes de avanzar hacia diseños más detallados y completos. Este proceso creativo es fundamental para desarrollar una solución efectiva y funcional para la refrigeración de insulina.



\subsection{Bosquejo general del sistema: cámara frigorífica}\rsp
\begin{figure}[H]
	\centering
	\includegraphics[width=0.6\linewidth]{figures/design-cooler2}
	\caption{Vista frotal 3D de la cámara vacía y sus componentes escenciales}
	\label{fig:design1}
\end{figure}
\begin{figure}[H]
	\centering
	\includegraphics[width=0.6\linewidth]{figures/design-latder}
	\caption{Vista lateral derecha  de la cámara y sus componentes de enfriamiento}
	\label{fig:design-latder}
\end{figure}
La figura \ref{fig:design1} muestra el diseño desde un panorama completo y sus componenentes esenciales, así mismo se muestra el espacio dentro de la cámara.
Mientras que en la figura \ref{fig:design-latder} se muestra una vista en donde puede apreciarse el sistema de enfriamiento y el sistema de tubos dentro de la cámara, más adelante se proporcionarán vistas específico.


\subsection{Diagrama a bloques del proceso de trabajo}

Los diagramas de bloques o de flujo son herramientas visuales que permiten representar de manera clara y concisa el flujo de un proceso o sistema, así como las interacciones entre sus diferentes componentes. \cite{lucidchart-2022}\\
\textbf{Importancia de los diagramas de bloques o de flujo:}
\begin{itemize}
	\item Claridad y comprensión.
	\item Identificación de problemas.
	\item Diseño y planificación.
	\item Documentación y estandarización.
	\item Toma de decisiones.
\end{itemize}
Vea la figura \ref{fig:flujo_insulina} a \ref{fig:diagrama-procesos} para tener más detalles del proceso que se seguirá y de las condiciones que se tomarán en cuenta.

\subsubsection{Diagrama de flujo}

\begin{figure}[H]
	\centering
	\begin{tikzpicture}[node distance=2cm]
		\tikzstyle{startstop} = [ellipse, minimum width=3cm, minimum height=1cm, text centered, draw=black, fill=blue!10]
		\tikzstyle{process} = [rectangle, minimum width=3cm, minimum height=1cm, text centered, draw=black, fill=green!10]
		\tikzstyle{decision} = [diamond, minimum width=3cm, minimum height=1cm, text centered, draw=black, fill=red!10]
		\tikzstyle{arrow} = [thick,->,>=stealth]
		\tikzstyle{line} = [draw, -latex']
		\node (start) [startstop] {Inicio};
		\node (reception) [process, below of=start] {Recepción de insulina};
		\node (storage) [process, below of=reception] {Almacenamiento en UMF40};
		\node (distribution) [process, below of=storage] {Distribución y reordenamiento};
		\node (decision) [decision, below of=distribution, yshift=-1.5cm] {¿Cámara llena?};
		\node (inventory) [process, right of=decision, xshift=3cm] {Realizar inventario};
 
		\path [line] (start) -- (reception);
		\path [line] (reception) -- (storage);
		\path [line] (storage) -- (distribution);
		\path [line] (distribution) -- (decision);
		\path [line] (decision) -- node[anchor=south] {Sí} (inventory);
		\path [line] (decision.west) -- ++(-1,0) |- node[anchor=south east] {No} (reception);
		
	\end{tikzpicture}
	\caption{Diagrama de flujo para el proceso de gestión de insulina en una cámara frigorífica.}  Fuente: Elaboración propia, basado y hecho desde Tikz de \LaTeX \cite{Tan12}.
	\label{fig:flujo_insulina}
\end{figure}
 

 
\begin{figure}[H]
	\centering
	\begin{tikzpicture}[node distance=2cm]
		\tikzstyle{startstop} = [ellipse, minimum width=3cm, minimum height=1cm, text centered, draw=black, fill=blue!10]
		\tikzstyle{process} = [rectangle, minimum width=3cm, minimum height=1cm, text centered, draw=black, fill=green!10]
		\tikzstyle{decision} = [diamond, minimum width=3cm, minimum height=1cm, text centered, draw=black, fill=red!10]
	%	\tikzstyle{arrow} = [thick,->,>=stealth]
		\tikzstyle{line} = [draw, -latex']
		
		\node (start) [startstop] {Inicio};
		\node (precooling) [process, below of=start] {Pre-enfriamiento};
		\node (efficiency) [process, below of=precooling] {Evitar pérdida de eficiencia};
		\node (coldstorage) [process, below of=efficiency] {Almacenamiento en frío};
		\node (control) [process, below of=coldstorage] {Control de variables};
		\node (monitoring) [process, below of=control] {Sistema de monitoreo};
		\node (decision1) [decision, below of=monitoring, yshift=-1.5cm] {¿Se mantiene la eficacia?};
		\node (highdemand) [decision, below left of=decision1, xshift=-2.5cm, yshift=-2.5cm] {¿Hay alta demanda?};
		\node (distribute) [process, below of=highdemand, yshift=-2cm] {Se distribuye el medicamento};
		\node (removal) [process, below right of=decision1, xshift=2.5cm, yshift=-2.5cm] {Se retira el producto};
		\node (end) [startstop, below of=removal] {Fin};
		
		\path [line] (start) -- (precooling);
		\path [line] (precooling) -- (efficiency);
		\path [line] (efficiency) -- (coldstorage);
		\path [line] (coldstorage) -- (control);
		\path [line] (control) -- (monitoring);
		\path [line] (monitoring) -- (decision1);
		\path [line] (decision1.west) -| node[anchor=south east] {Sí} (highdemand);
		\path [line] (decision1.east) -| node[anchor=south west] {No} (removal);
		\path [line] (highdemand) -- node[anchor=east] {Sí} (distribute);
		\path [line] (highdemand.west) -- ++(-1,0) |- node[anchor=south east] {No} (start);
		\path [line] (removal) -- (end);
		
	\end{tikzpicture}
	\caption{Diagrama de flujo para el proceso de gestión de insulina en una cámara frigorífica.} Fuente: Elaboración propia, basado y hecho desde Tikz de \LaTeX  \cite{Tan12}. 
	\label{fig:flujo_ref-insu}
\end{figure}
 


\subsection{Diagrama de funcionamiento general de sistema}


\begin{figure}[H]
	\centering
	\begin{tikzpicture}[mindmap, grow cyclic, every node/.style={concept, font=\small,  }, level 1/.append style={sibling angle=90, font = \small, font=\bfseries}, level 2/.append style={sibling angle=60}]
		\node[concept, text=white, font=\bfseries]{Diseñar una cámara de refrigeración para conservar insulina en la CDMX}
		child [concept color=red!30] {
			node {Conservar la eficiencia de la insulina}
			child [concept color=green!20, level distance=4cm] { node {Medicamentos de calidad} }
			child [concept color=green!20, level distance=4cm] { 
				node {Evitar pérdidas de producto}
				child [concept color=yellow!30, level distance=4cm, font= \small] { node {Controlar pérdidas por temperaturas altas} }
			}
		}
		child [concept color=blue!30] {
			node {Apoyar a pacientes diabéticos}
			child [concept color=orange!30, level distance=4cm] { 
				node {Impulsar la calidad en medicamentos de UMF 40}
				child [concept color=violet!30, level distance=4cm] { node {Inocuidad pública de calidad} }
			}
			child [concept color=orange!30, level distance=4cm] { 
				node {Reducir las tasas de mortalidad por la diabetes}
				child [concept color=violet!30, level distance=4cm] { node {Tratamientos seguros} }
				child [concept color=violet!30, level distance=4cm, font = \tiny] { node {Control de la degradación de la insulina} }
			}
			child [concept color=orange!30, level distance=4cm] { node {Apoyar en aforo de pacientes a UMF cercanas} }
		};
	\end{tikzpicture}
	\caption{Mapa mental para el diseño de una cámara de refrigeración para conservar insulina en la CDMX.}
	\label{fig:diagrama-procesos} Fuente: Elaboración propia, basado y hecho desde Tikz de \LaTeX \cite{Tan12}.
\end{figure}
  
\section{Listado de partes de los componentes del sistema}
 
En la figura \ref{fig:design1} se muestran los componentes principales a trabajar y se enumeran de  la siguiente forma:
\begin{enumerate}
	\item  Paneles superiores: elementos que formarán parte del techo de la cámara. (Ver figura \ref{fig:design1})
	\item  Paneles laterales: elementos que funcionarán como paredes y contendrán la
	temperatura en condiciones óptimas. (Ver figuras \ref{fig:design1} y \ref{fig:design-latder})
	\item  Accesos: Puertas/ventanas: permitirán el control y distribución del aire dentro (Ver figura \ref{fig:design-front} )
	y fuera de la cámara frigorífica.
	\item  Equipo de enfriamiento: elemento encargado de proporcionar frio y 
	transmitirlo al interior del recinto frigorífico. (Ver figuras \ref{fig:design-enfria} y \ref{fig:design-ventils})
	\item  Recubrimiento lateral: Material que recubre las paredes; estos ayudarán a 
	aislar y conservar la temperatura del recinto.
	\item  Recubrimiento superior: Material que recubre el techo; estos ayudarán a 
	aislar y conservar la temperatura del recinto.
\end{enumerate}
En la siguiente figura (\ref{fig:design-front}) podemos apreciar la vista frontal de la cámara en donde se contará con una puerta corrediza, con las consideraciones ya descritas con anterioridad.
 
 \begin{figure}[H]
	\centering
	\includegraphics[width=0.3125\linewidth]{figures/design-front}
	\caption{Vista frontal de la cámara.} Fuente: Elaboración propia basado de \citeNP{bibliocad}
	\label{fig:design-front}
\end{figure}



\newcolumntype{J}{>{\raggedright\arraybackslash}p{10cm}}


\newpage 


\section{Identificación del conjunto y/o subconjunto por área tecnológica} %3.9
\textit{¿Cómo se relacionan entre sí los elementos componentes?}\\
De la sección anterior (3.8) se tiene la siguiente lista de componentes principales del sistema de refrigeración a diseñar, los cuales son señalados en la figura \ref{fig:listacomponentes} y descritos en la tabla \ref{tabla:lista-componentes}

\begin{figure}[H]
	\centering
	\includegraphics[width=0.6\linewidth]{figures/lista_componentes}
	\caption{Elementos componentes principales del sistema de refrigeración a diseñar.}Fuente: Elaboración propia basado de \citeNP{bibliocad}
	\label{fig:listacomponentes}
\end{figure}

\begin{table}[H]
	\centering
	\caption{Identificación de elementos principales de forma general del sistema.}
	\begin{tabular}{@{}cl@{}}
		\toprule
		\textbf{Número}           & \textbf{Nombre del elemento o componente}           \\
		\midrule
		1                         & Paneles superiores (techo)                          \\
		2                         & Paneles laterales (paredes)                         \\
		3                         & Puertas/ventanas                                    \\
		4                         & evaporador                                          \\
		5                         & condensador                                         \\
		6                         & compresor                                           \\
		7                         & Válvula de expansión                                \\
		8                         & Recubrimiento lateral                               \\
		9                         & Recubrimiento superior                              \\ \bottomrule
	\end{tabular}
	\label{tabla:lista-componentes}
\end{table}

Como se ha descrito en capítulos previos, las cámaras de refrigeración son fundamentales en la cadena de frío, especialmente para productos sensibles como la insulina, que requiere condiciones específicas de almacenamiento para mantener su eficacia y seguridad. Conocer los elementos de una cámara de refrigeración por área tecnológica es crucial para asegurar que estos medicamentos se conserven adecuadamente. A continuación, se detallan las áreas tecnológicas y la importancia de cada una en el contexto del almacenamiento de insulina.
\newpage

La \cite{UG2022}, describe a través de su blog (un documento en extenso) los componentes de un sistema de refrigeración tradicional los cuales han sido adaptados para el sistema del presente proyecto, se detallan en las tablas de \ref{tabla:mecanicos} a \ref{tabla:ergonomia}.



\begin{table}[H]
	\centering
	\caption{Conjunto 1: Listado de componentes mecánicos/térmicos del sistema de refrigeración}
	Fuente: Elaboración propia basado de la \citeNP{UG2022}.
	\begin{tabular}{@{}cl@{}}
		\toprule
		%\textbf{Número} & 
		\begin{minipage}[t]{0.25\linewidth}\textbf{Nombre del elemento o componente}\end{minipage} & \begin{minipage}[t]{0.7\linewidth}\textbf{Descripción del conjunto o integración de subsistema}\end{minipage} \\
		
		\midrule
		%	1               &
		Compresor                                 & \begin{minipage}[t]{0.7\linewidth}El compresor es el corazón del sistema de refrigeración, esencial para mantener el refrigerante en movimiento.\end{minipage} \\
		%	2               &
		Evaporador                                & \begin{minipage}[t]{0.7\linewidth}Absorbe el calor del interior de la cámara, enfriando el aire para mantener una temperatura adecuada.\end{minipage} \\
		%	3               & 
		Condensador                               & \begin{minipage}[t]{0.7\linewidth}Disipa el calor del refrigerante al exterior, ayudando a mantener una temperatura constante en la cámara.\end{minipage} \\
		%	4               & 
		Ventiladores                              & \begin{minipage}[t]{0.7\linewidth}Aseguran la circulación de aire uniforme dentro de la cámara de refrigeración, evitando zonas de diferentes temperaturas.\end{minipage} \\
		%5               &
		Puertas y sellos                          & \begin{minipage}[t]{0.7\linewidth}Mantienen la integridad de la cámara, evitando la entrada de aire caliente y humedad.\end{minipage} \\
		
		Recubrimiento térmico              
		
		
		& \begin{minipage}[t]{0.7\linewidth}
			Contiene la temperatura  	dentro de la cámara 		  	frigorífica evitando perdidas 		  	de calor (paredes laterales y superiores).
		\end{minipage} \\
		
		
		\bottomrule
	\end{tabular}
	\label{tabla:mecanicos}
	
\end{table}


\begin{table}[H]
	\centering
	\caption{Conjunto 2: Listado de componentes eléctricos y electrónicos del sistema de refrigeración}
	Fuente: Elaboración propia basado de \citeNP{UG2022}.
	\begin{tabular}{@{}cl@{}}
		\toprule
		%	\textbf{Número} &
		\begin{minipage}[t]{0.25\linewidth}\textbf{Nombre del elemento o componente}\end{minipage} & \begin{minipage}[t]{0.50\linewidth}\textbf{Descripción del conjunto o integración de subsistema}\end{minipage} \\
		
		\midrule
		%	1               & 
		Termostato                                & \begin{minipage}[t]{0.5\linewidth}Controla y monitoriza la temperatura interna de la cámara, asegurando condiciones óptimas.\end{minipage} \\
		%2               &
		Sensores de temperatura                   & \begin{minipage}[t]{0.5\linewidth}Proporcionan lecturas precisas de la temperatura, fundamentales para el control del sistema.\end{minipage} \\
		%	3               & 
		Fuente de alimentación ininterrumpida & \begin{minipage}[t]{0.5\linewidth}Garantiza el funcionamiento continuo durante cortes de energía, evitando fluctuaciones de temperatura.\end{minipage} \\
		%4               &
		Generador de respaldo                     & \begin{minipage}[t]{0.5\linewidth}Proporciona energía durante apagones prolongados, asegurando el correcto funcionamiento del sistema.\end{minipage} \\ \bottomrule
	\end{tabular}
	\label{tabla:electricos}
\end{table}

\begin{table}[H]
	\centering
	\caption{Conjunto 3: Listado de componentes de automatización del sistema de refrigeración}
	Fuente: Elaboración propia basado de \citeNP{UG2022}.
	\begin{tabular}{@{}cl@{}}
		\toprule
		%\textbf{Número} & 
		\begin{minipage}[t]{0.25\linewidth}\textbf{Nombre del elemento o componente}\end{minipage} & \begin{minipage}[t]{0.60\linewidth}\textbf{Descripción del conjunto o integración de subsistema}\end{minipage} \\
		
		\midrule
		%	1               & 
		Controlador electrónico                   & \begin{minipage}[t]{0.6\linewidth}Regula el funcionamiento del sistema de refrigeración basado en lecturas de los sensores.\end{minipage} \\
		%	2               & 
		Sistema de monitoreo remoto               & \begin{minipage}[t]{0.6\linewidth}Permite la supervisión del sistema de refrigeración a distancia, facilitando la gestión y el mantenimiento.\end{minipage} \\
		%	3               & 
		Grabador de datos históricos              & \begin{minipage}[t]{0.6\linewidth}Registra las condiciones de temperatura a lo largo del tiempo, ayudando a mantener un registro detallado.\end{minipage} \\
		%4               &
		Sistemas de alarma                        & \begin{minipage}[t]{0.6\linewidth}Alertan sobre cualquier desviación de la temperatura, permitiendo acciones correctivas rápidas.\end{minipage} \\ \bottomrule
	\end{tabular}
	\label{tabla:automatizacion}
\end{table}




\begin{table}[H]
	\centering
	\caption{Conjunto 4: Listado de componentes ergonómicos del sistema de refrigeración}
	Fuente: Elaboración propia basado de \citeNP{UG2022}.
	\begin{tabular}{@{}cl@{}}
		\toprule
		%	\textbf{Número} &
		\begin{minipage}[t]{0.25\linewidth}\textbf{Nombre del elemento o componente}\end{minipage} & \begin{minipage}[t]{0.60\linewidth}\textbf{Descripción del conjunto o integración de subsistema}\end{minipage} \\ \midrule
		%	1               &
		Puertas de fácil acceso                   & \begin{minipage}[t]{0.6\linewidth}Diseñadas para un acceso sencillo y seguro a la cámara.\end{minipage} \\
		%2               &
		Panel de control intuitivo                & \begin{minipage}[t]{0.6\linewidth}Facilita la operación del sistema, con interfaces fáciles de entender y manejar.\end{minipage} \\
		%	3               &
		Iluminación interior adecuada             & \begin{minipage}[t]{0.6\linewidth}Proporciona una visibilidad clara dentro de la cámara, mejorando la seguridad y la eficiencia operativa.\end{minipage} \\
		%4               & 
		Racks ajustables                          & \begin{minipage}[t]{0.6\linewidth}Permiten la organización flexible del espacio de almacenamiento, optimizando el uso del espacio disponible.\end{minipage} \\ \bottomrule
	\end{tabular}
	\label{tabla:ergonomia}
\end{table}
















































\newpage
\section{Descripción de la interacción del elemento componente del sistema.}

Teniendo en cuenta que para la interacción entre elementos componentes debe existir algún tipo de movimiento, ya sea por transmisión o de transformación, los cuales en un equipo de refrigeración son poco comunes, entonces, para este apartado no se describirá una lista de componentes con este enfoque.



\section{Proceso del meta diseño de un sistema artefactual tecnológico}

\textit{Para que el producto tecnológico cumpla adecuadamente su función, ¿qué partes son esenciales y cuáles no?}\\


Se explica a continuación de acuerdo al autor \cite{de-leon-no-date} para que una cámara de refrigeración cumpla adecuadamente su función de almacenar y conservar la insulina, es vital realizar un meta diseño detallado. Este análisis permite identificar los componentes esenciales que aseguran el correcto funcionamiento del sistema de refrigeración y cuáles podrían ser prescindibles o de menor prioridad. Los componentes esenciales incluyen aquellos que directamente afectan la capacidad de mantener la temperatura adecuada, la seguridad y la estabilidad operativa del sistema.

\textbf{Importancia del metadiseño}
\begin{itemize}
	\item Optimización del funcionamiento: Al entender la función de cada componente, se pueden optimizar sus características y ubicaciones para mejorar el rendimiento del sistema.
	\item Eficiencia energética: Identificar componentes clave ayuda a mejorar la eficiencia energética del sistema, reduciendo costos operativos.
	\item Fiabilidad y durabilidad: Asegurar que los componentes esenciales sean de alta calidad aumenta la fiabilidad y durabilidad del sistema.
	\item Costos de mantenimiento: Identificar partes no esenciales permite reducir costos de mantenimiento y reparación, enfocándose en componentes críticos.
	\item Seguridad y cumplimiento: Un estudio detallado asegura que el sistema cumpla con las normas de seguridad y regulaciones pertinentes.
	
\end{itemize}


\textbf{Partes esenciales y no esenciales:}
\begin{itemize}
	\item Partes esenciales: Son aquellas cuyo fallo impactaría directamente en el funcionamiento de la cámara de refrigeración, como el compresor, el evaporador y los ventiladores.
	
	\item Partes no esenciales: Son componentes que, aunque necesarios, su fallo no detendría el sistema completo, como ciertos tornillos, pernos o elementos de soporte.
\end{itemize}

\newpage

\begin{landscape}
	En la tabla \ref{tabla:esenciales} se retoma la información y características principales de los elementos del sistema
	artefactual para su correcta función.
	
	
	\begin{table}[H]
		\centering
		\caption{Identificación de las partes esenciales de los componentes del sistema}
		Fuente: Elaboración propia basado de \cite{bohn}
		\begin{tabular}{@{}ccccccccc@{}}
			\toprule
			\textbf{Número} & \textbf{Componente}                                                              & \textbf{Duradero} & \textbf{Cambiable o} & \textbf{Flexible}    & \textbf{Comercial} & \textbf{Comercial}     & \multicolumn{2}{c}{\textbf{Selección}} \\  
			\textbf{}       & \textbf{}                                                                        & \textbf{}         & \textbf{modular}     & \textbf{o compuesto} & \textbf{nacional}  & \textbf{internacional} & \textbf{Sí}        & \textbf{No}       \\
			\midrule
			1               & \begin{tabular}[c]{@{}c@{}}Paneles \\ superiores\\ (techo)\end{tabular}          & 15 años           & cambiable            & flexible             & x                  &                        & x                  &                   \\
			2               & \begin{tabular}[c]{@{}c@{}}Paneles\\ laterales\\ (paredes)\end{tabular}          & 15 años           & cambiable            & flexible             & x                  &                        & x                  &                   \\
			3               & \begin{tabular}[c]{@{}c@{}}Puertas y \\ ventanas\end{tabular}                    & 15 años           & cambiable            & flexible             & x                  &                        & x                  &                   \\
			4               & Evaporador                                                                       & 10-12 años        & modular              & compuesto            & x                  & x                      & x                  &                   \\
			5               & Condensador                                                                      & 10-12 años        & modular              & compuesto            & x                  & x                      & x                  &                   \\
			6               & Compresor                                                                        & 10-12 años        & modular              & compuesto            & x                  & x                      & x                  &                   \\
			7               & \begin{tabular}[c]{@{}c@{}}Válvula\\ de expansión\end{tabular}                   & 10-12 años        & modular              & compuesto            & x                  & x                      & x                  &                   \\
			8               & \begin{tabular}[c]{@{}c@{}}Recubrimientos\\ (lateral y \\ superior)\end{tabular} & 15 años           & cambiable            & flexible             & x                  &                        & x                  &                   \\ \bottomrule
		\end{tabular}
		\label{tabla:esenciales}
	\end{table}
\end{landscape}
\newpage
\section{Alternativas tecnológicas de los componentes}
Evaluar alternativas tecnológicas para los componentes del sistema de refrigeración, planteando los requerimientos técnicos y mostrando las restricciones operativas y de seguridad esenciales para cada componente. A continuación se describen algunas de las características de este apartado que posteriorme se desglosa en la tabla \ref{tabla:alternativas} \cite{salvatore-2015}. \\
El autor también describe algunas restricciones que se deben considerar en el proceso de diseño de la cámara frigorífica.
\begin{itemize}
	\item Las restricciones operativas se refieren a las limitaciones y condiciones bajo las cuales los componentes y el sistema en su totalidad deben operar de manera efectiva. Estas restricciones aseguran que el sistema funcione dentro de los parámetros óptimos, manteniendo su rendimiento y evitando fallos.
	
	\begin{itemize}
		\item {Condiciones Ambientales:} Los componentes deben ser capaces de operar en las condiciones ambientales esperadas, como temperatura y humedad. En el caso de una cámara de refrigeración, esto incluye temperaturas extremadamente bajas y ambientes húmedos.
		\item {Carga y Capacidad:} Los componentes deben soportar la carga y capacidad de trabajo esperadas sin sufrir daños o degradación. Esto es especialmente importante para partes mecánicas como tornillos, pernos y racks ajustables.
		\item {Durabilidad y Ciclo de Vida:} Los componentes deben tener una durabilidad adecuada y un ciclo de vida que cumpla con los requisitos del sistema, minimizando el mantenimiento y reemplazo frecuente.
		\item {Compatibilidad:} Los componentes deben ser compatibles entre sí y con el sistema en general, asegurando una integración efectiva y sin problemas.
	\end{itemize}
	
	
	
	\item Las restricciones de seguridad son las limitaciones y requisitos destinados a proteger a los usuarios y a la integridad del sistema. Estas restricciones buscan prevenir accidentes, fallos catastróficos y daños a las personas y al medio ambiente.
	
	
	\begin{itemize}
		\item {Protección contra Fallos:} Los componentes deben estar diseñados para prevenir y resistir fallos, como sobrecalentamientos, cortocircuitos y fugas. Esto es crucial para partes como compresores y fuentes de alimentación.
		\item {Materiales Seguros:} Los materiales utilizados en los componentes deben ser seguros para el entorno en el que se utilizan, no liberando sustancias tóxicas o peligrosas. Esto incluye materiales no reactivos y seguros para alimentos.
		\item {Diseño a Prueba de Manipulaciones:} Los componentes deben estar diseñados para prevenir manipulaciones no autorizadas o accidentales, asegurando que solo el personal capacitado pueda acceder a ellos.
		\item  {Normativas y Regulaciones:} Los componentes deben cumplir con las normativas y regulaciones de seguridad aplicables, como normas eléctricas, mecánicas y sanitarias. Esto asegura que el sistema cumpla con los estándares legales y de calidad.
	\end{itemize}
	
	
\end{itemize}

%
%\begin{landscape}
\begin{table}[H]
	\centering
	\caption{Alternativas tecnológicas de los componentes}
	Fuente: Elaboración propia,  basado de \cite{salvatore-2015}
	\begin{tabular}{ccc}
		\hline
		\textbf{Componente}                                                              & \textbf{Restricción}                                                                                                                                              & \textbf{Restricción}                                                                                                                          \\
		\textbf{}                                                                        & \textbf{operacional}                                                                                                                                              & \textbf{de seguridad}                                                                                                                         \\ \hline
		\begin{tabular}[c]{@{}c@{}}Paneles \\ superiores\\ (techo)\\[0.5cm]\end{tabular}      & \begin{tabular}[c]{@{}c@{}}Resistencia a la intemperie,\\ capacidad de carga\end{tabular}                                                                         & -                                                                                                                                             \\
		\begin{tabular}[c]{@{}c@{}}Paneles\\ laterales\\ (paredes)\\[0.5cm]\end{tabular}          & \begin{tabular}[c]{@{}c@{}}Resistencia estructural, \\ aislamiento térmico\end{tabular}                                                                           & -                                                                                                                                             \\
		\begin{tabular}[c]{@{}c@{}}Puertas y \\ ventanas\end{tabular}                    & \begin{tabular}[c]{@{}c@{}}Estanqueidad, aislamiento \\ térmico, resistencia \\ a la intemperie\\[0.5cm]\end{tabular}                                                      & \begin{tabular}[c]{@{}c@{}}Seguridad contra\\ intrusiones, bloqueo\\ seguro\end{tabular}                                                      \\
		Evaporador                                                                       & \begin{tabular}[c]{@{}c@{}}Temperatura de evaporación,\\ capacidad de intercambio\\  de calor, flujo de aire\\[0.5cm]\end{tabular}                                         & \begin{tabular}[c]{@{}c@{}}Seguridad contra\\ fugas, protección\\ contra congelamiento\end{tabular}                                           \\
		Condensador                                                                      & \begin{tabular}[c]{@{}c@{}}Temperatura de condensación,\\ capacidad de intercambio\\ de calor, flujo de aire\\[0.5cm]\end{tabular}                                         & \begin{tabular}[c]{@{}c@{}}Seguridad eléctrica,\\ protección contra\\ sobrecalentamiento\end{tabular}                                         \\
		Compresor                                                                        & \begin{tabular}[c]{@{}c@{}}Rango de operación de presión,\\ temperatura ambiente, aceite de\\ lubricación, eficiencia\\[0.5cm]\end{tabular}                                & \begin{tabular}[c]{@{}c@{}}Seguridad eléctrica,\\ protección contra\\ sobrecalentamiento,\\ dispositivos de paro de\\ emergencia\end{tabular} \\
		\begin{tabular}[c]{@{}c@{}}Válvula\\ de expansión\end{tabular}                   & \begin{tabular}[c]{@{}c@{}}Control de flujo de refrigerante,\\ compatibilidad con refrigerante,\\ ajuste de supercalentamiento\\ y/o subenfriamiento\\[0.5cm]\end{tabular} & -                                                                                                                                             \\
		\begin{tabular}[c]{@{}c@{}}Recubrimientos\\ (lateral y \\ superior)\end{tabular} & \begin{tabular}[c]{@{}c@{}}Resistencia a la corrosión,\\ durabilidad, compatibilidad\\ con la estructura\end{tabular}                                             & \begin{tabular}[c]{@{}c@{}}Seguridad contra\\ corrosión, materiales\\ no tóxicos\end{tabular}                                                 \\ \hline
	\end{tabular}
	\label{tabla:alternativas}
\end{table}
%\end{landscape}




%\newpage
\vspace*{2.8cm}
\section{Selección de los componentes del sistema artefactual tecnológico}\vspace*{-0.5cm}
\begin{landscape}
	\begin{table}[H]
		\centering
		\caption{Selección de los componentes del sistema artefactual tecnológico}
		Fuente: Elaboración propia,  basado de \cite{salvatore-2015} 
		\begin{tabular}{@{}cccl@{}}
			\toprule
			\textbf{Componente}                                                              & \textbf{Función en}                                                                                   & \textbf{Criterios de}                                                                                              & \multicolumn{1}{c}{Funcionamiento en}                                                                                                                                  \\
			\textbf{}                                                                        & \textbf{el sistema}                                                                                   & \textbf{selección}                                                                                                 & \multicolumn{1}{c}{el subsistema}                                                                                                                                      \\ \midrule
			\begin{tabular}[c]{@{}c@{}}Paneles \\ superiores\\ (techo)\end{tabular}          & \begin{tabular}[c]{@{}c@{}}Protección estructural\\  y resistencia al clima\end{tabular}              & \begin{tabular}[c]{@{}c@{}}Material resistente a la\\ intemperie, capacidad de carga\end{tabular}                  & \begin{tabular}[c]{@{}l@{}}Forman la envolvente exterior, protegiendo\\  contra condiciones climáticas y proporcionando \\ resistencia estructural.\end{tabular}       \\
			\begin{tabular}[c]{@{}c@{}}Paneles\\ laterales\\ (paredes)\end{tabular}          & \begin{tabular}[c]{@{}c@{}}Protección estructural\\ y aislamiento térmico\end{tabular}                & \begin{tabular}[c]{@{}c@{}}Material resistente,\\ capacidad de aislamiento térmico\end{tabular}                    & \begin{tabular}[c]{@{}l@{}}Proporcionan estructura al sistema y ayudan a\\  mantener la temperatura interna mediante\\  el aislamiento térmico\end{tabular}            \\
			\begin{tabular}[c]{@{}c@{}}Puertas y \\ ventanas\end{tabular}                    & \begin{tabular}[c]{@{}c@{}}Acceso controlado y\\ protección térmica\end{tabular}                      & \begin{tabular}[c]{@{}c@{}}Estanqueidad, aislamiento\\ térmico, seguridad\end{tabular}                             & \begin{tabular}[c]{@{}l@{}}Permiten el acceso al sistema, aseguran\\  la estanqueidad y contribuyen al aislamiento térmico.\end{tabular}                               \\
			Evaporador                                                                       & \begin{tabular}[c]{@{}c@{}}Absorción de calor y\\ evaporación del\\ refrigerante\end{tabular}         & \begin{tabular}[c]{@{}c@{}}Capacidad de intercambio de\\ calor, flujo de aire, seguridad contra fugas\end{tabular} & \begin{tabular}[c]{@{}l@{}}Absorbe el calor del entorno, permite que el refrigerante\\ evapore y extrae calor del espacio a enfriar.\end{tabular}                      \\
			Condensador                                                                      & \begin{tabular}[c]{@{}c@{}}Disipación de calor y\\ cambio de fase del\\ refrigerante\end{tabular}     & \begin{tabular}[c]{@{}c@{}}Capacidad de\\ intercambio de\\ calor, seguridad\\ eléctrica\end{tabular}               & \begin{tabular}[c]{@{}l@{}}Libera el calor absorbido por el evaporador al entorno,\\  permitiendo que el refrigerante cambie de fase.\end{tabular}                     \\
			Compresor                                                                        & \begin{tabular}[c]{@{}c@{}}Compresión del\\ refrigerante y\\ circulación en el\\ sistema\end{tabular} & \begin{tabular}[c]{@{}c@{}}Capacidad de intercambio de\\ calor, seguridad eléctrica\end{tabular}                   & \begin{tabular}[c]{@{}l@{}}Comprime el refrigerante,\\  aumentando su presión y temperatura\\ para facilitar el intercambio de calor.\end{tabular}                     \\
			\begin{tabular}[c]{@{}c@{}}Válvula\\ de expansión\end{tabular}                   & \begin{tabular}[c]{@{}c@{}}Regulación del flujo\\ de refrigerante\end{tabular}                        & \begin{tabular}[c]{@{}c@{}}Control de flujo,\\ compatibilidad\\ con refrigerante\end{tabular}                      & \begin{tabular}[c]{@{}l@{}}Regula el flujo del refrigerante hacia el \\ evaporador, controlando así  la cantidad \\ de refrigerante que entra al sistema.\end{tabular} \\
			\begin{tabular}[c]{@{}c@{}}Recubrimientos\\ (lateral y \\ superior)\end{tabular} & \begin{tabular}[c]{@{}c@{}}Protección estructural\\ y resistencia a la\\ corrosión\end{tabular}       & \begin{tabular}[c]{@{}c@{}}Resistencia a la\\ corrosión,\\ durabilidad\end{tabular}                                & \begin{tabular}[c]{@{}l@{}}Proporciona protección contra la corrosión y\\  daños físicos, contribuyendo a la integridad estructural.\end{tabular}                      \\ \bottomrule
		\end{tabular}
	\end{table}
\end{landscape}


\newpage
\section{Informática para el control del proceso}

La transición de un diseño conceptual a un diseño preliminar a nivel de componentes es una etapa crucial en el desarrollo de una cámara de refrigeración para la conservación de insulina. Su importancia se puede resumir en varios puntos clave:
\begin{enumerate}
	
	\item \textbf{Clarificación de la Viabilidad Técnica}: El diseño preliminar aborda la viabilidad técnica, incluyendo la selección y especificación de los componentes, asegurando que cada uno cumpla con los requisitos de rendimiento y durabilidad necesarios.
	
	\item  \textbf{Detallado de Especificaciones Técnicas}: En el diseño preliminar se detallan las especificaciones técnicas de cada componente, como el tipo de material, las dimensiones y las tolerancias. Esto asegura que todos los componentes funcionen en armonía y cumplan con los estándares de calidad y seguridad.
	
	\item  \textbf{Evaluación de Costos y Factibilidad Económica}: Esta etapa permite una evaluación más precisa de los costos, posibilitando hacer estimaciones más realistas y optimizar la relación costo-beneficio, garantizando la viabilidad económica del proyecto.
	
	\item \textbf{Identificación de Problemas Potenciales}: El diseño preliminar permite identificar y solucionar problemas potenciales antes de la fase de producción, ahorrando tiempo y recursos.
	
	\item  \textbf{Facilitación de la Comunicación y Colaboración}: Un diseño preliminar detallado proporciona una base común para la comunicación entre todos los involucrados en el proyecto, promoviendo una colaboración eficiente y efectiva.
	
	\item  \textbf{Preparación para la Fabricación y Ensamblaje}: Define cómo fabricar y ensamblar los componentes, asegurando la calidad y consistencia del producto final con instrucciones detalladas para fabricantes y ensambladores.
	
	\item  \textbf{Consideraciones de Ecodiseño}: Se pueden incorporar consideraciones de ecodiseño, como la selección de materiales sostenibles y la optimización de la eficiencia energética, contribuyendo a la sostenibilidad ambiental y ofreciendo ventajas económicas a largo plazo.
	
\end{enumerate}

Vea los gráficos \ref{fig:final1},\ref{fig:plano-fin}, \ref{fig:design-enfria-fin} y \ref{fig:design-ventils} y la tabla \ref{tabla:lista-componentes-fin}


\begin{figure}[H]
	\centering
	\includegraphics[width=0.6\linewidth]{figures/final1}
	\caption{Diseño preliminar final, vista frontal}
	\label{fig:final1}
\end{figure}



\begin{figure}[H]
	\centering
	\includegraphics[width=0.6\linewidth]{figures/axo-design-cableado}
	\caption{Vista de la cámara con el diseño final nivel de cableado}
	\label{fig:plano-fin}
\end{figure}


\begin{figure}[H]
	\centering
	\includegraphics[width=0.46\linewidth]{figures/design-enfria}
	\caption{Sistema de ventilación}
	\label{fig:design-enfria-fin}  Fuente: Elaboración propia basado de \citeNP{bibliocad}
\end{figure}

\begin{figure}[H]
	\centering
	\includegraphics[width=0.46\linewidth]{figures/design-ventils}
	\caption{Vista interior de la cámara de refrigeración (Sistema de ventilación )}
	\label{fig:design-ventils}  Fuente: Elaboración propia basado de \citeNP{bibliocad}
\end{figure} 
Consulte los anexos gráficos  para ver más detalles de los planos y del diseño.



\begin{table}[H]
	\centering
	\caption{Elementos principales seleccionados de forma general del sistema.}
	Fuente: Elaboración propia basada en \cite{caterpillar-2008}
	\begin{tabular}{cccc}
		\hline
		\textbf{Num} & \textbf{Nombre del componente} & \textbf{Material}                                                         & \textbf{Marca} \\ \hline
		1               & Paneles superiores (techo)                & \begin{tabular}[c]{@{}c@{}}Acero inoxidable con \\ aislamiento de poliuretano\end{tabular}  & Kingspan                 \\
		2               & Paneles laterales (paredes)               & \begin{tabular}[c]{@{}c@{}}Acero galvanizado con\\  aislamiento de poliuretano\end{tabular} & Metecno                  \\
		3               & Puertas/ventanas                          & \begin{tabular}[c]{@{}c@{}}Acero inoxidable con \\ aislamiento térmico\end{tabular}         & True                     \\
		4               & Evaporador                                & Aluminio                                                                                    & Alfa Laval               \\
		5               & Condensador                               & Cobre y aluminio                                                                            & Emerson                  \\
		6               & Compresor                                 & Acero                                                                                       & Copeland                 \\
		7               & Válvula de expansión                      & Latón                                                                                       & Danfoss                  \\
		8               & Recubrimiento lateral                     & Acero inoxidable                                                                            & Kingspan                 \\
		9               & Recubrimiento superior                    & Acero inoxidable                                                                            & Metecno                  \\ \hline
	\end{tabular}
	\label{tabla:lista-componentes-fin}
\end{table}


\newpage
\section{Conclusión}

En este capítulo se han atacado los planteamientos básicos del diseño y la implementación de una cámara de refrigeración para conservar insulina en la Ciudad de México, los cuales sin duda, son procesos complejos que demandan una atención meticulosa a diversos aspectos. Además de los aspectos técnicos y logísticos, es fundamental adoptar un enfoque integral que considere las necesidades y preocupaciones de los pacientes diabéticos, así como el impacto potencial en la salud pública.\\ 
Además de garantizar la calidad y la eficacia de la insulina almacenada, es importante prestar atención a la disponibilidad y accesibilidad de la insulina para los pacientes diabéticos. Esto incluye aspectos como la distribución eficiente de los medicamentos, la gestión de inventarios y la coordinación con las unidades de atención médica para garantizar un suministro continuo y adecuado. \\
Por otro lado, conocer los planos y bosquejos preliminares es crucial para garantizar el almacenamiento seguro y eficiente de medicamentos sensibles, como la insulina. Permiten la identificación de áreas críticas, como la distribución de unidades de refrigeración, el aislamiento térmico adecuado y los sistemas de monitoreo y control de temperatura.\\ 
Adicionalmente, la selección adecuada de los materiales es esencial para asegurar la durabilidad, eficiencia y seguridad de la cámara de refrigeración. Materiales como el acero inoxidable y el aluminio no solo ofrecen resistencia a la corrosión y excelentes propiedades térmicas, sino que también contribuyen a la sostenibilidad del sistema al ser reciclables.\\ 
Finalmente, la etapa de ecodiseño es crucial para minimizar el impacto ambiental a lo largo del ciclo de vida de la cámara de refrigeración. Esto implica la consideración de factores como el consumo energético, la selección de materiales sostenibles y la eficiencia de los procesos de fabricación y reciclaje.\\
Un enfoque de ecodiseño no solo contribuye a la conservación del medio ambiente, sino que también puede generar beneficios económicos a largo plazo al optimizar recursos y reducir costos operativos.





  
	
%\setcounter{page}{45}
\clearpage
%\pagenumbering{arabic}
\newpage
% \addcontentsline{toc}{chapter}{\hfill 34}
\addtocontents{toc}{\protect\contentsline{chapter}{CAPÍTULO IV. Propuesta de diseño   \hfill  83}{}{}}






\begin{titlepage}
	
	
	\centering
	\begin{tikzpicture}%opacity=0.5
		\node[inner sep=0pt, ] (image) at (0,0) {\includegraphics[width=\textwidth]{figures/front-chapetr4}};
		\fill [white,path fading=south] (-5,-4) rectangle (5,4);
		\node[black,font=\Huge\bfseries] at (0,3) {Capítulo IV. Propuesta de diseño};
		\node[black,font=\Large\bfseries] at (0,1) {Cálculo térmico y selección de componentes};
		\node[black,font=\Large\bfseries] at (0,0) {Diseño de propuesta en SolidWorks};
	\end{tikzpicture}
\end{titlepage}


 \newpage 
 
 \section*{Introducción}
 
 
 \addcontentsline{toc}{section}{{Introducción}} 
 \setcounter{chapter}{4}
 \setcounter{page}{84}   
 \setcounter{section}{0}
 \setcounter{figure}{0}
 \setcounter{table}{0}
 
 El núcleo de la investigación reside en la aplicación de la teoría físico-matemática, particularmente en los aspectos técnicos que involucran el cálculo de cada componente del sistema, así como sus parámetros fundamentales. Estos cálculos, producto de años de investigación, son esenciales para asegurar la correcta funcionalidad del equipo de refrigeración seleccionado.\\
 Para optimizar la distribución de los elementos del sistema, se calculan las cargas térmicas con base en tablas y gráficos científicos especializados en refrigeración. Esto nos permite seleccionar el equipo adecuado y diseñar un sistema que cumpla con las necesidades específicas del proyecto. Los antecedentes del lugar donde se instalará la cámara de refrigeración proporcionan información crucial para identificar las condiciones críticas a las que estará expuesta, como la temperatura ambiente y las posibles fuentes de calor externas.\\
 El cálculo preciso de las propiedades térmicas de la insulina y su entorno es esencial. Esto incluye no solo la temperatura de almacenamiento, sino también el material de la cámara de transporte y el acomodo óptimo del producto para minimizar la transferencia de calor desde el exterior. Estas propiedades, determinadas experimentalmente, nos proporcionan los datos necesarios para realizar los ajustes pertinentes en el diseño.\\
 Finalmente, se realizan las modificaciones necesarias al diseño inicial, que se propuso en el capítulo III (MetaDiseño), del semestre anterior, tomando en cuenta cualquier cambio surgido durante el proceso de cálculo y distribución del equipo. El resultado es una propuesta final que se presenta con planos detallados de las dimensiones del sistema, garantizando una funcionalidad eficiente y segura para la conservación de la insulina.\\
En los capítulos previos se establecen las bases fundamentales   para la comprensión del diseño térmico y de cámaras de conservación de insulina, así como el contexto específico del proyecto. En el \textbf{Capítulo 1}, se abordan las generalidades del diseño térmico, con especial atención a los principios físicos y matemáticos que rigen los sistemas de refrigeración, conocimientos esenciales estudiados en la \textit{Escuela Superior de Ingeniería Mecánica y Eléctrica} (ESIME) del \textit{Instituto Politécnico Nacional}. El \textbf{Capítulo 2} contextualiza el proyecto, ofreciendo una visión clara sobre la necesidad y relevancia de la conservación de insulina en clínicas como la UMF 40 en Azcapotzalco. Por último, el \textbf{Capítulo 3} se enfoca en el metadiseño, donde se presentan los lineamientos y directrices que guiarán la implementación técnica del sistema de refrigeración, fundamentados en los temas estudiados en ESIME, tales como el análisis de cargas térmicas y la selección de equipos especializados.
 \newpage
\section{Marco referencial}\rspitems
\subsection{Normas ISO en planos de ingeniería}
Basarse en normas ISO es fundamental para los proyectos de ingeniería, ya que estas normas proporcionan un marco reconocido internacionalmente que garantiza la calidad, seguridad y compatibilidad de los diseños. En particular, para el diseño de equipos de refrigeración médica, el cumplimiento de estas normas es esencial para asegurar que los sistemas funcionen de manera óptima y cumplan con los requisitos de conservación de productos sensibles, como la insulina. Las normas ISO permiten que los diseños sigan un estándar riguroso, lo que facilita la colaboración entre equipos internacionales y asegura que los productos cumplan con las exigencias de seguridad y eficacia del sector.\rspitems
\subsubsection{Norma ISO 5457 (Presentación de dibujos técnicos) }\rspitems
La Norma ISO 5457 es un estándar internacional que proporciona directrices para la presentación de dibujos técnicos en papel y en formato digital. Esta norma define los tamaños de papel, los márgenes, las líneas y las convenciones de representación gráfica, entre otros aspectos, con el objetivo de garantizar la legibilidad y la claridad de los dibujos técnicos \cite{manzanelli-2023}.\rspitems
\subsubsection{Norma ISO 6433:2012 (Documentación técnica del producto)}\rspitems
Esta Norma Internacional proporciona reglas para la presentación de referencias de piezas en representaciones de conjuntos. Por ejemplo, en dibujos de conjunto, para identificar las partes constituyentes en una lista de partes relacionadas \cite{iso_org-2024}.
También se toman las siguientes recomendaciones con el titular de la materia del proyecto 2, el Dr.  \citeauthor{sotomucino2024}, durante el curso 25/1.
\begin{enumerate}
	\item A cada pieza del conjunto se le debe asignar una marca única que sirva como referencia del elemento. Esta marca debe diferenciarse claramente de cualquier otra indicación presente en el dibujo.\rspitems	
	\item Los elementos idénticos dentro de un conjunto deben identificarse con la misma referencia, y si no hay riesgo de ambigüedad, se mencionarán únicamente una vez.	\rspitems
	\item En caso de que existan grupos de elementos, cada subconjunto debe recibir una referencia única que lo identifique.\rspitems
	\item Cada referencia debe estar conectada visualmente con el elemento correspondiente mediante una línea de referencia, que se extienda desde la marca hasta un punto o una flecha, de acuerdo con los principios generales de representación gráfica establecidos en los dibujos técnicos.\rspitems
	\item Las referencias deben colocarse de manera que garanticen la máxima claridad y legibilidad del dibujo, preferiblemente organizadas en filas y columnas alineadas.\rspitems
	\item El orden para numerar las referencias debe seguir un criterio definido, como por ejemplo:
	\begin{itemize}
		\item Orden de montaje posible.\rspitems
		\item Orden de importancia de los elementos.\rspitems
		\item Cualquier otro criterio lógico que se ajuste a las necesidades del diseño.\rspitems
	\end{itemize}
\end{enumerate}\rsp
\subsubsection{ISO 7573:2008 (Listas de piezas):}\rspitems
La norma ISO 7573:2008 establece los requisitos mínimos que deben cumplir las listas de piezas para proporcionar la información necesaria, por ejemplo, para la producción, la adquisición o el mantenimiento de las piezas. Abarca tanto las listas de piezas manuales como las generadas por ordenador.

 
 \section{Propuesta solución}

 
Considerando el objetivo del proyecto, que se centra en el almacenamiento para la conservación de insulina, partimos de la necesidad de la Unidad de Medicina Familiar (UMF 40) y de las clínicas ubicadas en un radio de 1 kilómetro, para garantizar el abastecimiento de este medicamento en la alcaldía Azcapotzalco, Ciudad de México. Actualmente, la UMF 40 cuenta con una sola cámara de refrigeración, diseñada según los estándares de los principales fabricantes en el mercado de la refrigeración médica. Sin embargo, esta cámara se utiliza para almacenar al menos tres tipos de medicamentos, lo que genera problemas relacionados con la seguridad y eficiencia en el manejo de los mismos. Por tanto, el propósito del proyecto es realizar un cálculo preciso de la carga energética y térmica, considerando todos los factores que pueden influir en la localización, recepción y manejo del medicamento. Esto permitirá asegurar un óptimo funcionamiento del equipo, reducir el consumo energético y evitar la adquisición de equipo innecesario, garantizando la calidad y efectividad de la insulina antes de su administración.\\
Las dimensiones de la Cámara Frigorífica fueron tomadas de acuerdo con la demanda, espacio disponible en el área de farmacia y flujo de recepción del producto que se presenta a los pacientes de la alcaldía. La insulina llega en cajas de plástico por paquetes, donde estás pasan un proceso de control de calidad y sanidad por personal especializado de la UMF 40, después se montan en charolas o plásticos comunes y corrientes (una mala práctica que se planea erradicar), con el fin de distribuir el medicamento en algún compartimento del refrigerador libre y etiquetar el medicamento de acuerdo a los lineamientos del hospital regidos por el IMSS.\\
La distribución de la cámara de refrigeración se detalla en la figura \ref{fig:4-propuestasol}. El contenedor del refrigerador está adaptado con láminas compuestas de poliuretano (película interna $f_i$ - Poliuretano - película externa $f_e$) en las cuatro paredes, con el objetivo de minimizar la pérdida de temperatura por transferencia térmica a través de las superficies. Esta aislación contribuye a evitar el uso de ventiladores, mejorando la eficiencia energética en las etapas iniciales del funcionamiento.\\
En la parte posterior de la cámara se integra la unidad de refrigeración, cuya función es proteger los componentes del sistema. Esta unidad alberga los elementos principales, como el compresor, el condensador y el dispositivo de expansión, que están conectados directamente al evaporador. El evaporador está situado en el interior de la cámara y conectado al serpentín, cuya función es asegurar una mejor distribución del refrigerante dentro de la cámara, lo que permite una disipación de calor más eficiente y uniforme.\\
Además, la cámara está equipada con diversas tapas y una cubierta de cristal, diseñadas para mantener el medicamento en condiciones óptimas de almacenamiento.
  
  \begin{figure}[H]
 	\centering
 	\includegraphics[width=0.8\linewidth]{figures/4-propuesta_sol}
 	\caption{Vistas de la cámara de refrigeración.}
 	Fuente: Elaboración propia usando \texttt{SolidWorks.}
 	\label{fig:4-propuestasol}
 \end{figure}
 
 \subsubsection{Unidad de refrigeración}
Como se muestra en la figura \ref{fig:4-coolerunit}, se ha dispuesto que la unidad de refrigeración se ubique en la parte trasera y exterior de la cámara de refrigeración. Esta elección es fundamental, ya que la unidad genera calor durante su operación, lo que, de no estar adecuadamente posicionada, podría elevar la temperatura interna de la cámara. Tal incremento de temperatura es crítico, ya que puede comprometer la integridad y efectividad de la insulina, que requiere condiciones específicas de almacenamiento para garantizar su estabilidad y seguridad.\\
La unidad de refrigeración alberga componentes esenciales del sistema, como el condensador, el compresor y la válvula de expansión. Cada uno de estos elementos desempeña un papel crucial en el ciclo de refrigeración. El compresor, por ejemplo, es responsable de comprimir el refrigerante, aumentando su presión y temperatura, mientras que el condensador permite que el refrigerante se enfríe y se condense, liberando el calor hacia el ambiente. La válvula de expansión regula el flujo del refrigerante hacia el evaporador, donde se produce la refrigeración efectiva del aire dentro de la cámara.\\
La adecuada ubicación y el funcionamiento eficiente de esta unidad son vitales para mantener un ambiente controlado en el interior de la cámara, garantizando así que la insulina se conserve en condiciones óptimas, protegiendo su eficacia y, en última instancia, la salud de los pacientes que dependen de este tratamiento.
 
 \begin{figure}[H]
 	\centering
 	\includegraphics[width=0.4\linewidth]{figures/4-cooler_unit}
 	\caption{Unidad de refrigeración.}
 	Fuente: Elaboración propia usando \texttt{SolidWorks.}
 	\label{fig:4-coolerunit}
 \end{figure}
  
 \subsubsection{Evaporador}
 
 
En la figura \ref{fig:4-evaporator}, el evaporador del sistema se ubica en la parte interior de la cámara, proporcionando un espacio de almacenamiento adecuado y una posición estratégica que optimiza su rendimiento. El evaporador está diseñado con un serpentín que se seleccionará en función de varios factores, como la carga térmica esperada, el tipo de refrigerante utilizado y las características específicas de la insulina a conservar. \\
Dado que la cámara tiene dimensiones de $60\times 60 \times 51{.}2$ centímetros, es fundamental considerar aspectos como el flujo de aire interno y la distribución térmica. La selección del serpentín garantizará una transferencia de calor eficiente y uniforme, minimizando las zonas frías o calientes que podrían afectar la integridad del producto. Además, en los cálculos de secciones posteriores se considera, la capacidad del evaporador para manejar la carga térmica en función de la cantidad y tipo de insulina almacenada, así como las condiciones ambientales externas propias de la Alcaldía. 
 \begin{figure}[H]
 	\centering
 	\includegraphics[width=0.4\linewidth]{figures/4-evaporator}\includegraphics[width=0.4\linewidth]{figures/4-evaporator2}
 	\caption{Vistas del evaporador (serpentín).}
 	Fuente: Elaboración propia usando \texttt{SolidWorks.}
 	\label{fig:4-evaporator}
 \end{figure}\rsp 
 
 \subsubsection{Serpentín del condensador}
 El serpentín del condensador de la figura  \ref{fig:4condenser}, es una parte fundamental del sistema de refrigeración, cuya función principal es la de disipar el calor absorbido por el refrigerante durante su ciclo de compresión. En el contexto de la conservación de insulina, la correcta operación del serpentín es crítica para mantener la temperatura interna de la cámara en niveles óptimos. El serpentín del condensador se ubica en la parte externa de la cámara de refrigeración para asegurar que el calor no regrese al compartimento donde se almacena la insulina.\\
 Una mala selección o una disposición ineficiente del serpentín podría resultar en una refrigeración inadecuada, generando fluctuaciones de temperatura que comprometerían la estabilidad del medicamento. Mantener una temperatura constante es crucial para la preservación de la insulina, ya que cambios bruscos pueden degradar su eficacia.
 \begin{figure}[H]
 	\centering
 	\includegraphics[width=0.6\linewidth]{figures/4condenser}
 	\caption{Vista del serpentín del condensador}
 	Fuente: Elaboración propia usando \texttt{SolidWorks.}
 	\label{fig:4condenser}
 \end{figure}
 
 
 \subsubsection{Compresor}
 En la figura \ref{fig:4-compressor}, el compresor del sistema se localiza en la parte externa de la cámara de refrigeración, siendo una de las piezas clave para el funcionamiento eficiente del sistema. Este componente es responsable de aumentar la presión del refrigerante, permitiendo su circulación a través del sistema de refrigeración y garantizando que el evaporador reciba el refrigerante en condiciones óptimas. \
 
 Para la selección del compresor, se tomarán en cuenta varios factores, tales como la carga térmica requerida para conservar la insulina, la eficiencia energética y el tipo de refrigerante utilizado. Es esencial que el compresor cuente con la capacidad suficiente para manejar la carga térmica, asegurando un funcionamiento constante y eficiente, especialmente considerando las variaciones de temperatura en la Alcaldía de Azcapotzalco. \
 
 Además, se evaluará la compatibilidad del compresor con el sistema de control de temperatura de la cámara, ya que un control adecuado es crucial para mantener las condiciones óptimas para la conservación de la insulina. La elección de un compresor eficiente no solo contribuirá a un menor consumo energético, sino que también asegurará la integridad y efectividad del medicamento almacenado.
 
 \begin{figure}[H] 
 	\centering 
 	\includegraphics[width=0.4\linewidth]{figures/4-compressor}\includegraphics[width=0.4\linewidth]{figures/4-compressor2}
 	\caption{Vistas del compresor.} 
 	Fuente: Elaboración propia usando \texttt{SolidWorks.} 
 	\label{fig:4-compressor}
 \end{figure}
 
\subsubsection{Válvulas de expansión}
En la figura \ref{fig:4-expvalves} se muestra la válvula de expansión de nuestro sistema, éstas válvulas juegan un rol indispensable en el sistema de refrigeración, ya que controlan el flujo de refrigerante hacia el evaporador, permitiendo que el refrigerante se expanda y disminuya su temperatura antes de entrar en contacto con la cámara interna. En este caso, la válvula de expansión debe estar calibrada con precisión para mantener la temperatura estable en el rango adecuado para la conservación de insulina, que como ya sabemos está típicamente entre 2 °C y 8 °C.

\begin{figure}[H]
	\centering
	\includegraphics[width=0.5\linewidth]{figures/4-expvalves}
	\caption{Vista de las válvulas de expansión}
	Fuente: Elaboración propia usando \texttt{SolidWorks.}
	\label{fig:4-expvalves}
\end{figure}


\subsection{Esquema de bloques del funcionamiento del sistema}
 
 \begin{figure}[H]
 \centering 
  \begin{tikzpicture}[node distance=2cm and 1cm]
  	
  	% Bloques del sistema
  	\node (input) [block] {Entrada de potencia};
  	\node (compressor) [block, right=of input] {Compresor};
  	\node (condenser) [block, right=of compressor] {Condensador};
  	
  	% Filtro de secado más abajo
  	\node (filter) [block, below=of condenser] {Filtro de Secado};
  	\node (evaporator) [block, left=of filter] {Evaporador};
  	\node (valve) [block, left=of evaporator] {Válvula de Expansión};
  	\node (output) [block, left=of valve] {Salida de potencia};
  	
  	% Flechas entre los bloques
  	\draw [arrow] (input) -- (compressor);
  	\draw [arrow] (compressor) -- (condenser);
  	
  	% Flecha hacia abajo
  	\draw [arrow] (condenser) -- (filter);
  	
  	% Flecha de vuelta a la izquierda
  	\draw [arrow] (filter) -- (evaporator);
  	\draw [arrow] (evaporator) -- (valve);
  	\draw [arrow] (valve) -- (output);
  	
  \end{tikzpicture}
 \caption{Esquema de bloques del funcionamiento del sistema.}
 Fuente: Elaboración propia usando \texttt{Tikz}.
 \label{fig:4-blocksche}
\end{figure} 
  Este esquema, figura \ref{fig:4-blocksche}, ilustra el funcionamiento de un sistema de refrigeración, mostrando cada uno de los componentes clave y su rol en el proceso. A continuación, se explica cada etapa:
  \begin{enumerate}
 \item Entrada de potencia: Es la fuente de energía que alimenta el sistema. En la mayoría de los casos, esta energía es eléctrica, aunque en algunos sistemas puede ser mecánica. Es necesaria para que todos los componentes del sistema funcionen correctamente.
  
 \item Compresor: Este componente es esencial, ya que comprime el refrigerante (que en este caso suele ser un gas). Al comprimirlo, aumenta tanto la presión como la temperatura del gas, lo que es fundamental para los pasos siguientes del ciclo.
  
  \item Condensador: Una vez que el refrigerante ha sido comprimido, el condensador se encarga de disipar el calor del refrigerante caliente. El calor se libera hacia el exterior del sistema y, al perder calor, el refrigerante cambia de estado, pasando de gas a líquido.
  
 \item  Filtro de secado: Este pequeño pero importante componente tiene la tarea de eliminar cualquier traza de humedad o impureza del refrigerante líquido. Esto es vital, ya que la presencia de humedad podría dañar otros componentes del sistema, como la válvula de expansión.
   \item  Evaporador: Este es el componente donde ocurre la verdadera refrigeración. El refrigerante frío se evapora dentro del evaporador, absorbiendo el calor del área que necesita ser enfriada. Aquí, el refrigerante cambia nuevamente de líquido a gas.  
 \item  Válvula de expansión: Una vez que el refrigerante ha sido filtrado, la válvula de expansión regula el flujo de refrigerante que entra al evaporador. Además, reduce la presión del refrigerante, lo que hace que se enfríe aún más antes de llegar al evaporador.
  \item  Salida de potencia: Representa la energía térmica que el refrigerante ha absorbido del espacio enfriado y que será liberada cuando el ciclo comience de nuevo en el compresor. El proceso es cíclico y continuo, manteniendo así la refrigeración estable.
 
  \end{enumerate}
  
  
\section{Antecedentes}
                                                                                                                             En esta sección del capítulo, se presenta la base teórica de los cálculos realizados, destacando las observaciones clave del proceso de análisis y selección de los parámetros más adecuados. Además, se ofrece un resumen de los manuales utilizados como referencia para los cálculos, proporcionando un contexto claro y justificado para las decisiones tomadas.\\
Como se describió en la sección \ref{sec:contex_geografico}, el contexto geográfico es fundamental para el análisis del balance térmico o energético.
\subsection{Ubicación}
Tomando en cuenta que para lograr un alto rango de efectividad es necesario considerar las condiciones críticas de cualquier parámetro relevante, en este proyecto la ubicación geográfica juega un papel fundamental. En particular, se toma como referencia el Pueblo de Santa Bárbara, en la alcaldía Azcapotzalco, Ciudad de México. Las figuras  \ref{fig:mapsumf40} a \ref{fig:calordf} muestran tanto la ubicación geográfica como las condiciones climáticas del área, las cuales se resumen a continuación (ver también el mapa general en la figura \ref{4-mexicomap}):


\begin{itemize}
	\item Temperatura de bulbo seco: 25 °C (77 °F)\rspitems
	\item Temperatura de bulbo húmedo: 20 °C (68 °F)	\rspitems
	\item Altitud: 2,240 msnm \rspitems
	\item Presión atmosférica: 1,023 Pa (1 atm)\rspitems
\end{itemize}

\begin{figure}[H]
	\centering
	\includegraphics[width=0.6\linewidth]{figures/4-mexicomap}
	\caption{Mapa de localización del municipio Santa Bárbara Azcapotzalclo en la Ciudad de México.}\cite{semovi-24}
	\label{fig:4-mexicomap}
\end{figure}

\subsection{Tipo de producto}
\textbf{Insulina}\\
De acuerdo a la información recabada en la visita a la unidad 40 del IMSS se sabe que a los pacientes diabéticos se les proporciona dos tipos de insulina para su tratamiento, \textbf{Humalog Insulina Lispro} Figura \ref{fig:lispro-insul} e \textbf{Insulina Lantus}  Figura \ref{fig:lantus-insul}, este par de medicamentos se suministran directamente por parte del gobierno por lo que la logistica del proceso en la cadena de suministro está debidamente regulada. Además visite las tablas \ref{tabla:humalog} y \ref{tabla:lantus} para más detalles descritos del producto. Un resumen de dicha información se muestra en \ref{tabla:condsinsulina}

\subsubsection{Condiciones del producto a refrigerar.}


\begin{table}[H]
	\centering
	\caption{Condiciones del producto}
	Fuente: Elaboración propia, tomando datos directamente en la UMF.40
  	\begin{tabular}{cccc}
 		\hline
 		\textbf{Producto} & \multicolumn{3}{c|}{\textbf{Condiciones de almacenamiento}}                                                                                                                                                                        \\ \hline
 		& \textbf{\begin{tabular}[c]{@{}c@{}}Temp. \\ Almacenamiento (°C)\end{tabular}} & \textbf{\begin{tabular}[c]{@{}c@{}}Humedad \\ relativa (\%)\end{tabular}} & \textbf{\begin{tabular}[c]{@{}c@{}}Vida aprox. \\ (días)\end{tabular}} \\ \hline
 		\textbf{Insulina} & \begin{tabular}[c]{@{}c@{}}2 - 8\\ (3)\end{tabular}                           & \begin{tabular}[c]{@{}c@{}}35 - 70\\ (40)\end{tabular}                    & \begin{tabular}[c]{@{}c@{}}28\\ (antes de abrir)\end{tabular}          \\ \hline
 	\end{tabular}
 \begin{tabular}{cccc}
 	\hline
 	\textbf{Producto}    & \multicolumn{3}{c|}{\textbf{Condiciones de almacenamiento}} \\
 	& \textbf{CPB}        & \textbf{CPA}        & \textbf{HL}     \\ \cline{2-4} 
 	\textbf{Insulina}    & \textbf{Btu/lb °F}  & \textbf{Btu/lb °F}  & \textbf{Btu/lb} \\
 	\multicolumn{1}{l}{} & 1                   & 0.35                & 0.2006          \\ \hline
 \end{tabular}
 	 \label{tabla:condsinsulina}
\end{table}



\subsection{Temperatura del diseño de almacenaje}
 
En todos los procesos de refrigeración, es fundamental centrarse en el producto que requiere conservación, ya que nuestro principal objetivo es asegurar que se mantenga a la temperatura adecuada. Por lo tanto, es crucial contar con información cuantitativa y cualitativa precisa sobre las condiciones óptimas de almacenaje de dicho producto para garantizar su integridad y eficacia.
 
 \subsubsection{Temperatura de almacenamiento de la Insulina}
 
 De acuerdo con los proveedores \citeauthor{lispro-2006}, \citeyear{lispro-2006} y \citeauthor{lantus-2015},\citeyear{lantus-2015}, la temperatura de almacenamiento de la insulina es un aspecto para garantizar su calidad y eficiencia a cada paciente. En los centros de distribución y almacenaje, se conservan en el refrigerador a una temperatura de 2.7°C, y en las farmacias para su distribución directa al paciente deben conservar una temperatura entre 2°C - 8°C, visite los cuadros resumen de cada marca, Humalog (\ref{tabla:humalog}) y Lantus (\ref{tabla:lantus}).
 
 \subsection{Tipo de empaque para su almacenaje}
 
Durante una entrevista realizada con el personal médico de la UMF 40, se obtuvo información sobre el manejo de insulina en el área de distribución y almacenamiento de la farmacia de la unidad. El medicamento se transporta en cajas que contienen un número determinado de frascos, según la marca, donde cada caja incluye una cantidad específica de dosis de insulina. Debido a su alta ergonomía, se han considerado charolas de almacenamiento que facilitan la organización de las dosis sin su empaque original. Este arreglo no solo optimiza el espacio disponible, sino que también permite una manipulación más ágil, especialmente en situaciones extraordinarias, como desastres naturales o fallas en los equipos de refrigeración. En la tabla \ref{tabla:almacenaje} se presentan los detalles de las charolas utilizadas para dicho almacenamiento, en breve, se usarán parrillas de acero inoxidable con charolas de plástico, que está compuesta de polietileno de alta densidad, con las dimensiones que se describen a continuación.
 \subsection{Capacidad de almacenaje}
 \begin{table}[H]
 	\centering
 	\caption{Datos de capacidad y dimensiones de los empaques de almacenaje}\rspitems Fuente: Elaboración propia
 	\begin{tabular}{ccc}
 		\hline
 		\textbf{Tipo de empaque para su almacenaje} & \textbf{Capacidad neta (kg)} & \textbf{Dimensiones (cm)} \\ \hline
 		Unidad de charolas & 2.5 kg & 20 x 20 x 3 \\  
 		30 charolas & 75 kg & 60 x 60 x 52 \\  
 		\hline
 	\end{tabular} 	
 	\label{tabla:almacenaje}
 \end{table}
 
 Se tomará en cuenta para la Cámara frigorífica un arreglo de 5 parrillas, cada parrilla
 tendrá 6 charolas, cada una con 2.4 kg de insulina más 100 gramos del peso de la charola, lo que equivale a 15 kg por parrilla, así entonces se tiene un total de 75 kg y 720 frascos.
 \begin{equation}
 \begin{aligned}
 	\text{Número de charolas} &= \dfrac{75 kg}{2.5kg} = 30 \text{charolas}\\
 		\text{Número de charolas por tarima} &= \dfrac{30\; charolas}{5\; tarimas} = 6 \text{charolas por tarima}
 \end{aligned}
\end{equation}
El fabricante nos recomienda que el apilamiento debe contener 30 kg como máximo y de acuerdo a los cálculos realizados, se está dentro de lo permitido.

\subsection{Flujo de recepción}

Hablando de un producto a refrigerar de la manera más eficiente, el caso más optimo a considerar es colocar todo el producto en el tiempo más corto posible. Flujo de recepción: 225 Kg / Día, en 3 visitas del centro de distribución del gobierno mexicano.


\subsection{Planos y diagramas}
El acomodo de las charolas puede realizarse de distintas formas. Sin embargo, es fundamental garantizar que estén colocadas de manera estable, evitando que se vuelquen o desplacen durante la administración o distribución a los pacientes. Por esta razón, se sugiere distribuirlas de forma uniforme sobre la superficie de la parrilla. A continuación, se presenta un ejemplo de cómo se apilan las charolas llenas de frascos de insulina, colocadas una sobre otra sin dejar espacios entre ellas, tal como se muestra en la figura \ref{fig:4-frontalcharolas}.


\begin{figure}[H]
	\centering
	\includegraphics[width=0.6\linewidth]{figures/4-frontalcharolas}
	\caption{Vista frontal de la cámara de refrigeración llena de la insulina.}
	Fuente: Elaboración propia usando \texttt{SolidWorks.}
	\label{fig:4-frontalcharolas}
\end{figure}


En la figura \ref{fig:4-lateralcharolas1} se muestran las vistas laterales del acomodo de las charolas dentro de de la cámara de refrigeración    .

\begin{figure}[H]
	\centering
	\includegraphics[width=0.4\linewidth]{figures/4-lateralcharolas1}\includegraphics[width=0.5\linewidth]{figures/4-lateralcharolas2}
	\caption{Vistas laterales de la insulina acomodada dentro de la cámara.}
		Fuente: Elaboración propia usando \texttt{SolidWorks.}
	\label{fig:4-lateralcharolas1}
\end{figure}
 

En la figura  \ref{fig:4-superiorlcharolas}, vemos otra vista del arreglo de las charolas dentro de la cámara de refrigeración.

\begin{figure}[H]
	\centering
	\includegraphics[width=0.5\linewidth]{figures/4-superiorlcharolas}\includegraphics[width=0.6\linewidth]{figures/4-superiorlcharolas2}
	\caption{Vista superior de la cámara de refrigeración llena.}
		Fuente: Elaboración propia usando \texttt{SolidWorks.}
	\label{fig:4-superiorlcharolas}
\end{figure}

 



\subsection{Tipo de aislamiento térmico}

Algunos de los factores clave a considerar en la selección del aislamiento térmico incluyen su frecuencia de uso, área de aplicación, costo, eficiencia y el espacio que ocupará. Dado estos parámetros, el poliuretano expandido se presenta como una opción ideal, ya que cumple con los requisitos mencionados de manera eficiente. En la tabla \ref{tabla:aislantes} se presenta un resumen comparativo de los datos de diversos aislantes térmicos disponibles en el mercado.\\
El frigerante a utilizar para la conservación de insulina dentro de la cámara de refrigeración, una opción recomendada será el  \textbf{R-142b} o \textbf{R-600a}. Ambos refrigerantes son eficientes, seguros y adecuados para sistemas de refrigeración médica que requieren un control preciso de la temperatura. El R-600a es una opción más ecológica, ya que tiene un bajo impacto ambiental y no daña la capa de ozono, mientras que el R-142b es comúnmente utilizado en aplicaciones médicas por su estabilidad y rendimiento.

\begin{table}[H]
	\centering
	\caption{Conductividad térmica del aislamiento de cámaras frigoríficas}Fuente Fuente: Extraído del manual de Fundamentos (AHSRAE,1967).
	\begin{tabular}{lc}
		\hline
		\textbf{Aislamiento} & \textbf{Conductividad térmica ($k,\; Btu\; in/h ft °F)$} \\
		 \hline
		Tablero de poliuretano (R-11 expandido)   & 0.16 a 0.18 \\  
		Polisocianurato celular (R-141b expandido) & 0.19 \\  
		Poliestireno extruido (R-142b)             & 0.24 \\  
		Poliestireno expandido (R-142b)            & 0.26 \\  
		Tablero de corcho                          & 0.30 \\  
		Vidrio espumado                            & 0.31 \\ 
		\hline 
	\end{tabular}
	\label{tabla:aislantes}
\end{table}



\section{Balances de carga térmica}
Tomando como referencia las tablas de datos mostrados en las tablas, \ref{tabla:condsinsulina} y \ref{tabla:aislantes} calcularemos la carga térmica total que estará abatiendo la cámara frigorífica.

\subsection{Condiciones exteriores de diseño:}  
 \begin{equation} 
 \begin{aligned}
 	 T_{BS} &= 25^\circ C [77^\circ F] \\
 	T_{BH} &= 20^\circ C [68^\circ F] \\
 	H_R &= 90\%\\
 	T_{alamacenamiento}& = T_{alm} = 3^\circ C [37.4\degree F]
 	 \end{aligned} 		 
 \end{equation}
 
	
 \subsection{Aislamiento térmico: Poliuretano expandido}
	Tomando en consideración la aplicación del poliuretano expandido, que es el principal aislante térmico en consideración la refrigeración de productos médicos, 
	
 \begin{equation}
 	 \begin{aligned}
 	 k &= 0.16\frac{BTU \cdot pulg}{ft^2 \cdot hr \cdot ^\circ F}\\
 	 e&=\Big(\frac{1}{5}\Big)\Delta T = \Big(\frac{1}{5}\Big)\big(T_{BS}- T_{alm}\big)\\ 	
 	 &=  \Big(\frac{1}{5}\Big)\times (25-3)^\circ =4{.}4 cm\\
 	 &= 4{.}4 cm \times \dfrac{1\; in}{2{.}54 cm} =1{.}732283\; in\\
 	 \therefore e&= 1{.}74\; in \text{, para cálculo máximo}
 \end{aligned}
 \end{equation}
 \subsection{Coeficiente de película}
	Este coeficiente de película es considerado con el manual de ASHRAE, gracias a los
	estudios cualitativos del comportamiento de los factores de calor por convección en
	condiciones promedio.	
	\begin{equation}
		\begin{aligned}
			h_i &= 1.6 \, \frac{{BTU}}{{pie}^2 \cdot {hr} \cdot \degree F} \\
			h_e &= 6\, \frac{{BTU} \cdot {pulg}}{{pie}^2 \cdot {hr} \cdot \degree F}
		\end{aligned}
	\end{equation}
	
 \subsubsection{Coeficiente de transferencia de calor}
Para las áreas de las paredes de la cámara frigorífica, se propone una tabla específica
las áreas largas, cortas, piso y techo, observe la figura \ref{fig:paredes-peliculas} para detalles de la ecuación.

  \begin{equation}
  	\begin{aligned}
  		U &= \frac{1}{\frac{1}{f_i} + \frac{e_{\text{lamina}}}{k_{\text{lamina}}} + \frac{e_{\text{poliuretano}}}{k_{\text{poliuretano}}} + \frac{e_{\text{lamina}}}{k_{\text{lamina}}} + \frac{1}{f_e}} \\
  		U &= \frac{1}{\frac{1}{f_i} + \frac{2 e_{\text{lamina}}}{k_{\text{lamina}}} + \frac{e_{\text{poliuretano}}}{k_{\text{poliuretano}}} + \frac{1}{f_e}} \\
  	\end{aligned}
  	\label{eq:coef-transf-calor}
  \end{equation}


\begin{figure}[H]
	\centering
	\begin{tikzpicture}	
		% Dibujar las capas de la pared
		% Capa de Poliuretano
		\fill[pattern=north east lines, pattern color=teal] (-1,0) rectangle (1,6);
	
				\node at (0,6.3) {\textbf{\textcolor{teal}{Poliuretano}}};
		
		% Láminas Pintro exteriores
		\draw[thick, gray] (-2,0) -- (-2,6);
		\draw[thick, gray] (2,0) -- (2,6);
		
		\node[gray] at (-2,-0.3) {\textbf{Lámina Pintro}};
		\node[gray] at (2,-0.3) {\textbf{Lámina Pintro}};
		
		% Coeficiente de Película interior (fi)
		\draw[blue, thick, <->] (-3,0) -- (-3,6);
		\node[blue, rotate=90] at (-3.8,3) {$f_i$ \textbf{Coeficiente}};
		\node[blue, rotate=90] at (-3.4,3) {\textbf{de Película}};
		
		% Coeficiente de Película exterior (fe)
		\draw[blue, thick, <->] (3,0) -- (3,6);
		\node[blue, rotate=90] at (3.4,3) {$f_e$ \textbf{Coeficiente}};
		\node[blue, rotate=90] at (3.8,3) {\textbf{de Película}};	
	\end{tikzpicture}
	\caption{Pared Compuesta propuesta para la cámara frigorífica} Fuente: Elaboración propia usando \LaTeX \; y \texttt{Tikz}
	\label{fig:paredes-peliculas}
\end{figure}
 
\subsection{Cálculo térmico de las paredes}
\subsubsection{Cálculo de áreas}
Se hará el cálculo de las áreas superficiales de cada elemento que mantiene almacenado
el producto a transportar, tomando por supuesto las áreas superficiales exteriores como
el punto de partida de dicho calculo.
\begin{equation}
	\begin{aligned}
		\text{Norte:}\quad A_n &=l\times  h=  0.6\, m \times 0.512\, m = 0.3072\, m^2 \times \frac{10.76\, ft^2}{1\, m^2} = 3.299\, ft^2 \\ 
		\text{Sur:}\quad A_s &=l\times h=  0.6\, m \times 0.512\, m = 0.3072\, m^2 \times \frac{10.76\, ft^2}{1\, m^2} = 3.299\, ft^2 \\ 
		\text{Oriente:} \quad A_o &=l\times h= 0.6\, m \times 0.6\, m = 0.36\, m^2 \times \frac{10.76\, ft^2}{1\, m^2} = 3.867\, ft^2 \\ 
		\text{Poniente:}\quad A_p &=l\times h= 0.6\, m \times 0.6\, m = 0.36\, m^2 \times \frac{10.76\, ft^2}{1\, m^2} = 3.867\, ft^2 \\ 
		\text{Piso y Techo:}\quad A_{pt} &=l\times h=  0.6\, m \times 0.6\, m = 0.36\, m^2 \times \frac{10.76\, ft^2}{1\, m^2} = 3.867\, ft^2
	\end{aligned}
	\label{eq:area-paredes}
\end{equation}

\textbf{Nota:}  Abatir carga térmica en un periodo de 22 horas, este parámetro fue seleccionado por
dos razones en el planteamiento del problema, primero, el producto llega a temperatura "ideal" de distribuidor, así que no
se retirará carga térmica de calor latente, entonces, la capacidad o riesgo de falla, es casi
nula, y por último el segundo factor, fue tomar en consideración que entre más rápido se
necesite abatir una carga térmica, más grande tendrá que ser el dispositivo.

\subsubsection{Transmisión de calor en paredes, pisos y techo}

Para este cálculo es necesario consultar la ecuación \ref{eq:coef-transf-calor} el cual nos indica un coeficiente
a considerar con los parámetros tanto de calor por convección, y el factor del aislante térmico.

\begin{equation}
	U = 
	\frac{1}{\left( \frac{1}{1.6} \right) + \left( \frac{2\times 0.017 }{30.04} \right)+\left( \frac{1.74}{0.17}\right) + \left( \frac{1}{6}\right)} 
	= 0.0808 \, \frac{ {BTU}}{ {ft}^2 \cdot  {hr} \cdot \degree F}
	\label{eq:u}
\end{equation}

De manera que en la figura \ref{fig:paredes-peliculas2} se resumen los valores de los coeficientes de transferencia. 

\begin{figure}[H]
	\centering
	\begin{tikzpicture}	
		% Dibujar las capas de la pared
		% Capa de Poliuretano
		\fill[pattern=north east lines, pattern color=teal] (-1,0) rectangle (1,6);
		
		\node at (0,7) {\textbf{\textcolor{teal}{Poliuretano}}};
		
		% Láminas Pintro exteriores
		\draw[thick, gray] (-2,0) -- (-2,6);
		\draw[thick, gray] (2,0) -- (2,6);
		
		\node[gray] at (-2,-0.3) {{\small \textbf{Lámina Pintro Cal 26}}};
		\node[gray] at (2,-0.3) {{\small  \textbf{Lámina Pintro Cal 26}}};
		
		% Coeficiente de Película interior (fi)
		\draw[blue, thick, <->] (-3,0) -- (-3,6);
		\node[blue, rotate=90] at (-4,3) {$f_i = 1.6 \frac{ {Btu}}{h \cdot ft^2 \cdot \degree F}$};
		\node[blue, rotate=90] at (-3.4,3) {\textbf{Coeficiente de Película}};
		
		% Coeficiente de Película exterior (fe)
		\draw[blue, thick, <->] (3,0) -- (3,6);
		\node[blue, rotate=90] at (3.4,3) {$f_e = 6 \frac{ {Btu}}{h \cdot ft^2 \cdot \degree F}$};
		\node[blue, rotate=90] at (4,3) {\textbf{Coeficiente de Película}};	
		
		% Coeficiente de Conductividad (k) para Poliuretano
		\node[red] at (0,6.3)  {$k = 0.17 \frac{ {Btu} \cdot in}{h \cdot ft^2 \cdot \degree F}$};
		
		% Coeficiente de Conductividad (k) para Pintro
		\node[red] at (0,-1) {$k = 30.04 \frac{ {Btu} \cdot in}{h \cdot ft^2 \cdot \degree F}$};
	\end{tikzpicture}
	\caption{Pared Compuesta propuesta para la cámara frigorífica} Fuente: Elaboración propia usando \LaTeX \; y \texttt{Tikz}
	\label{fig:paredes-peliculas2}
\end{figure}


\subsubsection{Carga térmica por paredes}
La carga térmica para cada muro de la cámara de refrigeración se calcula usando la fórmula:

\begin{equation} 
	Q = A \cdot U \cdot \Delta T
	\label{eq:carga-termica}
\end{equation}
Definimos las diferencias de temperatura
\begin{equation}
	\begin{aligned}		 
		 \Delta T_n & = T_{bs} - t_{almac} = 77 - 37.4 = 39.6 \,  \degree F \\  
		 \Delta T_s & = T_{bs} - t_{almac} + t_{almac} = 77 - 37.4 + 37.4 = 77 \,  \degree F \\  
		 \Delta T_o & = T_{bs} - t_{almac} + t_{almac} = 77 - 37.4 + 37.4 = 77 \,  \degree F \\  
		 \Delta T_p & = T_{bs} - t_{almac} + t_{almac} = 77 - 37.4 + 37.4 = 77 \,  \degree F \\  
		 \Delta T_{piso} & = T_{bs} + 9 + t_{almac} = 77 + 9 + 37.4 = 123.4 \, \degree F
	\end{aligned}
	\label{eq:dif-temp-paredes}
\end{equation}

 
Ahora reemplazando los resultados de \ref{eq:area-paredes}, \ref{eq:u}  y \ref{eq:dif-temp-paredes} en la ecuación \ref{eq:carga-termica} obtendremos la carga térmica para cada pared de nuestra cámara de refrigeración.
  
\begin{equation}
	\begin{aligned}
		\text{Norte:}&\\ 
		Q_n &= A_n \cdot U \cdot \Delta T_n = 3.299 \, {ft}^2 \cdot 0.0808 \, \frac{{{BTU}}}{{{ft}^2 \cdot \degree F}} \cdot (39.6 \degree F) \\
		& = 3.299 \times 0.0808 \times 39.6 \\
		& \approx 10.639\, {BTU/hr} \\ 
		\therefore Q_n &= 10.6 \, {BTU/hr} \\[1em]
		\text{Sur:}&\\
		Q_s &= A_s \cdot U \cdot \Delta T_s = 3.299 \, {ft}^2 \cdot 0.0808 \, \frac{{{BTU}}}{{{ft}^2 \cdot \degree F}} \cdot (77 - 37.4 + 37.4) \\
		& = 3.299 \times 0.0808 \times 77 \\
		& \approx 20.891\, {BTU/hr} \\ 
		\therefore Q_s &= 20.9 \, {BTU/hr} \\[1em]
		\text{Oriente:}&\\
		Q_o &= A_o \cdot U \cdot \Delta T_o = 3.867 \, {ft}^2 \cdot 0.0808 \, \frac{{{BTU}}}{{{ft}^2 \cdot \degree F}} \cdot (77 - 37.4 + 37.4) \\
		& = 3.867 \times 0.0808 \times 77 \\
		& \approx 23.144\, {BTU/hr} \\ 
		\therefore Q_o &= 23.1 \, {BTU/hr} \\[1em]
		\text{Poniente:}&\\
		Q_p &= A_p \cdot U \cdot \Delta T_p = 3.867 \, {ft}^2 \cdot 0.0808 \, \frac{{{BTU}}}{{{ft}^2 \cdot \degree F}} \cdot (77 - 37.4 + 37.4) \\
		& = 3.867 \times 0.0808 \times 77 \\
		& \approx 23.144\, {BTU/hr} \\ 
		\therefore Q_p &= 23.1 \, {BTU/hr} \\[1em]
		\text{Piso:}&\\
		Q_{piso} &= A_{pt} \cdot U \cdot \Delta T_{piso} = 3.867 \, {ft}^2 \cdot 0.0808 \, \frac{{{BTU}}}{{{ft}^2 \cdot \degree F}} \cdot (77 + 9 - 37.4) \\
		& = 3.867 \times 0.0808 \times 48.6 \\
		& \approx 15.160\, {BTU/hr} \\ 
		\therefore Q_{piso} &= 15.2 \, {BTU/hr}  =Q_{techo}
	\end{aligned}
\end{equation}


 \begin{equation}
Q_{T_{paredes}} = Q_n +Q_s + Q_o + Q_p + Q_{piso} +q_{techo} = 108.1  {BTU/hr}
 \end{equation}
 
 \subsubsection{Carga térmica del producto}
 Para calcular la carga térmica del producto, es fundamental considerar que no hay calor latente que remover en el proceso de almacenamiento. Sin embargo, existe un calor sensible que debe tenerse en cuenta. Almacenaremos la insulina a una temperatura específica, lo que nos permite calcular la carga térmica sensible considerando la tasa de transferencia de masa y el calor específico del producto.\\
 La tasa de flujo de masa se calcula de la siguiente manera: 
 \begin{equation}
 	\begin{aligned}
 		\dot{m} &= (F \cdot R) \cdot \left( \frac{2,200 \, \text{lb}}{1 \, \text{TM}} \right) \cdot \left( \frac{1 \, \text{día}}{24 \, \text{hrs}} \right) \\
 		\dot{m} &=1TM\left( \frac{2,200 \, \text{lb}}{1 \,TM  } \right) \cdot \left( \frac{1 \, \text{día}}{24 \, \text{hrs}} \right) \\
 		\dot{m} &= 91.6666 \, \frac{\text{lb}}{\text{hr}}
 	\end{aligned}
 \end{equation}
La carga térmica sensible de la insulina \( Q_{\text{insulina}} \) se obtiene con la fórmula:

\begin{equation}
	\begin{aligned}
		Q_{\text{insulina}} &= \dot{m} \cdot C_p \cdot (T_{\text{en}} - T_{\text{alm}}) \\
		C_p &= 0.35 \, \frac{ {BTU}}{ {lb} \cdot \degree F}
	\end{aligned}
\end{equation}

Sustituyendo los valores:

\begin{equation}
	\begin{aligned}
		Q_S &=  91.6666 \, \frac{ {lb}}{ {hr}}\times \left( 0.35 \, \frac{ {BTU}}{ {lb} \cdot \degree F} \right) \times ( 6 - 3)=192.5 {BTU/hr} \\
		\therefore Q_{insulina} &=192.5 {BTU/hr}
	\end{aligned}
\end{equation}

\subsubsection{Carga térmica por Infiltración}
En este apartado de los cálculos, es importante aclarar varios factores que influyen en la selección del sistema de refrigeración para la conservación de insulina. Uno de los primeros aspectos a considerar en un espacio refrigerado es cuantificar la cantidad de calor que puede infiltrarse en la cámara. Para realizar este análisis de manera precisa, se debe comenzar por calcular la capacidad volumétrica del área a refrigerar, lo cual implica determinar el volumen total del espacio. Este valor es fundamental para dimensionar adecuadamente el equipo de refrigeración necesario.

Además del volumen, se deben tener en cuenta otros factores importantes, como la ubicación geográfica, la carga térmica generada por las operaciones internas y las condiciones ambientales externas que puedan afectar el rendimiento del sistema. Estos datos son esenciales para diseñar un sistema eficiente que mantenga las condiciones óptimas de temperatura y humedad dentro de la cámara refrigerada. ASHRAE proporciona una tabulación de datos que indica la cantidad de cambios de aire por hora en función de la capacidad volumétrica del espacio, lo que permite calcular con precisión la carga térmica total del sistema. 
 
 \begin{equation}
 	\begin{aligned}
 		V&=l\times a\times h\\
 		&=0.6m \times 0.6m\times 0.51m\\
 		&=0.1836 m^3 \times \left(35.31 \frac{ft^3}{m^3}\right)\\
  \therefore V&= 6.483 ft^3 
 	\end{aligned}
 \end{equation}
 
 Usando la tabla \ref{tabla:4-promedio-aire24h}, para poder encontrar el factor que indica los cambios de aire por cada  24 horas, de ser necesario, interpolar entre los valores aproximados para una mayor  exactitud
 
 \begin{table}[H]
 	\centering
 	\caption{ Promedio de cambios de aire en 24 horas para cámaras de
 		almacenaje debido a la apertura de puertas e infiltración} Fuente: Extraído del manual de Fundamentos ASHRAE, (1981). \\
 		\label{tabla:4-promedio-aire24h}		
 	\begin{tabular}{cccccc}
 		\hline
 		& \multicolumn{2}{c}{Cambios de aire en 24 horas}                             &                                       & \multicolumn{2}{c}{Cambios de aire en 24 horas}                           \\ \cline{2-3} \cline{5-6} 
 		\multirow{-2}{*}{\begin{tabular}[c]{@{}c@{}}Volumen\\  $ft^3$\end{tabular}} & Arriba de 32°F                       & Abajo de 32°F                        & \multirow{-2}{*}{Volumen $ft^3$}         & Arriba de 32°F                      & Abajo de 32°F                       \\ \hline
 		{\color[HTML]{DA8292} \textbf{200}}                                      & {\color[HTML]{DA8292} \textbf{44.0}} & {\color[HTML]{DA8292} \textbf{33.5}} &  \textbf{6,000} &  \textbf{6.5} & \textbf{5.0} \\
 		300                                                                      & 34,5                                 & 26.2                                 & 8.000                                 & 5.5                                 & 4.3                                 \\
 		400                                                                      & 29,5                                 & 22.5                                 & 10,000                                & 4.9                                 & 3.8                                 \\
 		500                                                                      & 26.0                                 & 20.0                                 & 15.000                                & 3.9                                 & 3.0                                 \\
 		600                                                                      & 23,0                                 & 18.0                                 & 20,000                                & 3.5                                 & 2.6                                 \\
 		800                                                                      & 20.0                                 & 15.3                                 & 25,000                                & 3.0                                 & 2.3                                 \\
 		1 ,000                                                                   & 17.5                                 & 13.5                                 & 30,000                                & 2.7                                 & 2.1                                 \\
 		1 ,500                                                                   & 14.0                                 & 11.0                                 & 40,000                                & 2.3                                 & 1.8                                 \\
 		2,000                                                                    & 12.0                                 & 9.3                                  & 50,000                                & 2.0                                 & 1.6                                 \\
 		3,000                                                                    & 9.2                                  & 7.4                                  & 75,000                                & 1.6                                 & 1.3                                 \\
 		4,000                                                                    & 8.2                                  & 6.3                                  & 100,000                               & 1.4                                 & 1.1                                 \\
 		6,000                                                                    & 7.2                                  & 5.6                                  &                                       &                                     &                                     \\ \hline
 	\end{tabular}
 	\end{table}
 	
 	
 	Para poder obtener factor del calor removido en el aire es necesario tener como dato la
 	temperatura de bulbo seco y la temperatura de almacenaje, así apoyándose de la tabla \ref{tabla:4-calor-removido} encontrar el factor de calor removido.
 
 	\begin{table}[H]
 		\centering
 		\caption{Calor removido en aire de enfriamiento en las condiciones de almacenamiento (BTU por pie cúbico)} 
 		Fuente: Extraído del manual de Fundamentos ASHRAE 1981.
 		\label{tabla:4-calor-removido}
 		\begin{tabular}{ccccccccc}
 			\hline
 			& \multicolumn{8}{c}{Temperatura del aire exterior °F}                                                                    \\ \cline{2-9} 
 			& \multicolumn{2}{c}{85}                      & \multicolumn{2}{c}{90} & \multicolumn{2}{c}{95} & \multicolumn{2}{c}{100} \\ \cline{2-9} 
 			& \multicolumn{8}{c}{Porciento de Humedad Relativa}                                                                       \\ \cline{2-9} 
 			\multirow{-4}{*}{\begin{tabular}[c]{@{}c@{}}Temperatura de\\ la cámara de\\ Almacenamiento\\ °F\end{tabular}} & 60   & 60                                   & 70         & 80        & 50         & 60        & 50         & 60         \\ \hline
 			25                                                                                                            & 0.39 & 0.43                                 & 0.69       & 0.91      & 0.93       & 1.20      & 2.54       & 1.51       \\
 			20                                                                                                            & 0.62 & 0.56                                 & 0.89       & 1.12      & 1.14       & I.41      & 2.68       & 1.71       \\
 			15                                                                                                            & 0.65 & 0.69                                 & 1.08       & 1.31      & 1.33       & 1.60      & 2.80       & 1.91       \\
 			10                                                                                                            & 0.77 & 0.82                                 & 1.26       & 149       & 1.51       & 1.78      & 2.93       & 2.09       \\
 			{\color[HTML]{DA8292} \textbf{5}}                                                                             & 0.89 & {\color[HTML]{DA8292} \textbf{0.94}} & 1.43       & 1.66      & 1.68       & 1.94      & 3.05       & 2.25       \\
 			{\color[HTML]{DA8292} \textbf{0}}                                                                             & 1.01 & {\color[HTML]{DA8292} \textbf{1.05}} & 1.59       & 1.81      & 1.83       & 2.10      & 3.16       & 2.41       \\
 			-5                                                                                                            & 1.13 & 1.17                                 & 1.74       & 1.96      & 1.99       & 2.25      & 3.28       & 2.56       \\
 			-10                                                                                                           & 1.24 & 1.29                                 & 1.88       & 2.10      & 2.13       & 2.39      & 3.40       & 2.70       \\ \hline
 		\end{tabular}
 	\end{table}
 	
 	
 	
 	\begin{equation}
 		\begin{aligned}
 			Q_{\text{Infiltración}} &= (V) \cdot \left( \frac{\text{Cambios}}{24 \, \text{hrs}} \right) \cdot (\text{Calor removido}) \cdot f \\
 			Q_{\text{Infiltración}} &= (6.483 {ft}^3) \cdot \left( 33.5 \, \frac{\text{Cambios}}{24 \, \text{hrs}} \right) \cdot \left(0.90  \, \frac{{BTU}}{{ft}^3} \right) \cdot 0.6 \\
 			Q_{\text{Infiltración}} &=117.27747\, \frac{{BTU}}{{hr}}
 		\end{aligned}
 	\end{equation}
 	
 	
 	
 	\subsubsection{Carga del Motor eléctrico}
 	
 	Este cálculo se basa en la preselección del equipo adecuado. Al hablar de un dispositivo diseñado para refrigerar insulina, es fundamental considerar el uso del Thermo King y sus componentes. El evaporador, que se instala dentro de la cámara refrigerada, se toma en cuenta dentro de los cálculos de carga térmica.\\
 	Para realizar el cálculo de la carga térmica es necesario disponer de los datos del motor y la cantidad de equipo dentro de la cámara. Para determinar el calor emitido por estos dispositivos, es necesario consultar la tabla \ref{tabla:4-motores-calor}, que contiene la información requerida para su aplicación en el área refrigerada.
 	
 	
 	
 	\begin{table}[H]
 		\centering
 		\caption{alor disipado por los motores eléctricos}
 		Fuente: Extraído del manual de Fundamentos ASHRAE 1981
 		 \label{tabla:4-motores-calor}
 		\begin{tabular}{cccc}
 			\hline
 			\multicolumn{1}{c|}{}                                                                         & \multicolumn{3}{c}{\textit{BTU por (hp)/(hora)}}                                                                                                                                                                                       \\ \cline{2-4} 
 			\multicolumn{1}{c|}{\multirow{-2}{*}{\begin{tabular}[c]{@{}c@{}}HP\\ del motor\end{tabular}}} & \begin{tabular}[c]{@{}c@{}}Motor y ventilador\\ dentro del cuarto\end{tabular} & \begin{tabular}[c]{@{}c@{}}Motor fuera y\\ ventilador dentro\end{tabular} & \begin{tabular}[c]{@{}c@{}}Motor dentro y\\ ventilador fuera\end{tabular} \\ \hline
 			1/8 a 1/2                                                                                     & 4250                                                                           & 2.545                                                                     & 1,700                                                                     \\
 			{\color[HTML]{DA8292} \textbf{1/2 a 3}}                                                       & {\color[HTML]{DA8292} \textbf{3,700}}                                          & 2,545                                                                     & 1,150                                                                     \\
 			3 a 20                                                                                        & 2,950                                                                          & 2,545                                                                     & 400                                                                       \\ \hline
 		\end{tabular}
 	\end{table}
 	Tomando como apoyo la ecuación del calor en motores y sustituyendo las variables a considerar de la tabla \ref{tabla:4-motores-calor} se obtiene el calor generado por la cantidad de motores
 	\begin{equation}
 		\begin{aligned}
 			Q_{\text{Motores}} &= (\#\text{Motores}) (\text{HP}) (\text{Calor disipado por los motores}) \\
 			Q_{\text{Motores}} &= (1)(1.20 \,  {HP}) \left( 2545 \, \frac{ {BTU}}{ {hr} \cdot  {HP}} \right) \\
 			Q_{\text{Motores}} &= 3054 \, \frac{ {BTU}}{ {hr}} 
 		\end{aligned}
 	\end{equation}
 	\subsubsection{Carga por iluminación}
 	Se toma en consideración la carga térmica promedio generada en una situación crítica
 	donde se toma 1 watt por cada pie cuadrado, y el total del área superficial total del techo.
 	
 	\begin{equation}
 		\begin{aligned}
 			Q_{iluminacion} &= (\text{Largo}_{Ext})(\text{Ancho}_{Ext})(\text{Factor de conversión})(\text{Dato de Norma})(\text{FC}_{Norma}) \\
 			&= 0.6\, m \times 0.6\, m \times 10.76\, \frac{ft^2}{m^2} \times 1\, \frac{Watt}{ft^2} \times 3.41\, \frac{BTU}{hr \cdot Watt} \\
 		\therefore Q_{iluminacion} &=  13.208 \frac{BTU}{hr}
 		\end{aligned}
 	\end{equation}
 	 \subsection{Carga térmica total}
 	
 	 \begin{equation}
 	 	\begin{aligned}
 	 		Q_{subtotal} &=Q_{T_paredes}+Q_{insulina}+Q_{Infiltracion}+ Q_{motores}+ Q_{iluminacion} \\
 	 		&= (108.1+192.5+ 117.3 + 3054+13.208)BTU/hr\\
 	 		 \therefore Q_{subtotal} &=3,485.108 BTU/hr
 	 	\end{aligned}
 	 \end{equation}
 	Considerando un factor de seguridad por cuestiones de variaciones de temperatura originadas
 	por los cambios climáticos sufridos por el país, que no son consideradas dentro de los
 	parámetros calculados.
 	 \begin{equation}
 		\begin{aligned}
 			10\% \; F.S &=34.851 Btu/h\\
 			\therefore Q_{Total} &=3,519.951 BTU/hr
 		\end{aligned}
 	\end{equation}
 	
 \section{Selección de equipo}	

 
 \subsection{Unidad condensadora}
 
 En esta sección, es fundamental contar con una amplia variedad de catálogos de unidades condensadoras tipo Thermo King para realizar una selección eficiente del equipo adecuado. La selección de la unidad condensadora Thermo King incluye los componentes esenciales del proceso de refrigeración, como la unidad evaporadora, la unidad de expansión, el condensador o intercambiador de calor, y el compresor. Con estos elementos incluidos en el sistema Thermo King, solo queda seleccionar con base en los parámetros de temperatura de almacenamiento y la carga térmica con la que operará la unidad condensadora.
 
 De acuerdo con el Catálogo Thermo King, categoría Advancer A-500 (\hyperref[fig:axo-manual-thermo-king]{anexo 7}) ), la capacidad de la unidad condensadora a una temperatura de operación de $-20^\circ C$ (muy cercana a los $-18^\circ C$ calculados) es de 
 \[
 10,400\, W \left( 35,464\, \frac{BTU}{hr} \right)
 \]
 Comparando con los cálculos obtenidos de 
 \[
	3,485.108 \, \frac{BTU}{hr}
 \]
 hay una diferencia aproximada de 
 \[
 1,000\, \frac{BTU}{hr}
 \]
 lo que representa una variación del 2.78\%. Esta pequeña diferencia confirma que la selección del equipo es adecuada. Para más detalles, consulte el catálogo completo en el \hyperref[fig:axo-manual-thermo-king]{anexo 7}.
 
 \begin{figure}[H]
 	\centering
 	\includegraphics[width=0.8\linewidth]{figures/4-seleccion-condensador}
 	\caption{Tabla de datos recortado del catálogo de especificaciones Thermo King del anexo 7.}
 	Fuente: \hyperref[fig:axo-manual-thermo-king]{(Thermo King, 2024)}
 	\label{fig:4-seleccion-condensador}
 \end{figure}
 
 \section{Diseño del sistema eléctrico}
 Es fundamental implementar un sistema eléctrico seguro (ver figura \ref{fig:4-frontalcharolas}) para una cámara de refrigeración destinada al almacenamiento de insulina, dada la importancia de mantener condiciones óptimas para la conservación de productos farmacéuticos. Un sistema eléctrico confiable garantiza que la temperatura dentro de la cámara se mantenga constante y segura, evitando la degradación del fármaco y asegurando su efectividad. En la figura \ref{fig:4-diag-electrico} se puede apreciar un acercamiento al diagrama resumido que se propone, para más detalles vea el \hyperref[axo:diag-electrico]{anexo 3}. \\
 Prevención de Incendios: Un diseño eléctrico defectuoso o inadecuado puede provocar cortocircuitos o sobrecargas eléctricas, lo que incrementa significativamente el riesgo de incendios. Esto no solo puede causar daños al equipo, sino que también compromete la seguridad de las instalaciones y del personal.\\
  \textbf{Mantenimiento del Equipo}: Un sistema eléctrico bien diseñado contribuye al funcionamiento óptimo y prolongado de los equipos de refrigeración. Esto reduce la probabilidad de fallas y preserva los componentes en buen estado por más tiempo, lo cual es crucial en un entorno donde la integridad de los productos farmacéuticos es prioritaria.\\
\textbf{Cumplimiento Normativo}: En el ámbito farmacéutico, existen regulaciones y estándares específicos que deben cumplirse para el almacenamiento de productos sensibles como la insulina. Asegurarse de cumplir con estos requisitos es esencial para evitar multas y sanciones, además de garantizar la seguridad y eficacia de los productos almacenados.


\subsection{Esquema a bloques del funcionamiento del sistema eléctrico.}

El esquema de la figura \ref{fig:4-blockelectric} incluye las siguientes protecciones para mantener seguros y protegidos los componentes del sistema de refrigeración:
\begin{itemize}

	\item[$\odot$] Interruptor de Circuito: Permite cortar el suministro eléctrico en caso de sobrecarga o cortocircuito, protegiendo así los componentes eléctricos.

	\item[$\odot$] Protección Contra Sobrecorriente: Previene daños al sistema eléctrico en caso de corrientes excesivas.

\item[$\odot$] Protección Contra Sobretensión: Salvaguarda al sistema contra picos de voltaje que podrían dañar los componentes eléctricos.

\item[$\odot$] Controlador de Temperatura: Asegura que la temperatura dentro de la cámara de refrigeración se mantenga dentro de los límites seguros.

\item[$\odot$] Sensor de Temperatura y Humedad: Monitorea continuamente las condiciones de la cámara, garantizando el correcto funcionamiento del sistema de refrigeración.

\end{itemize}
 \begin{figure}[H]
	\centering 
	\begin{tikzpicture}[node distance=2cm and 1cm]
		
		% Bloques del sistema
		\node (input) [block] {Entrada de energía};
		\node (switch) [block, right=of input] {Interruptor de circuito};
		\node (protector) [block, right=of switch] {Protección contra sobrecorriente};
		
		% Filtro de secado más abajo
		\node (controltemp) [block, below=of protector] {Control de temperatura};
		\node (compresor) [block, left=of controltemp] {Compresor y sistema de refrigeración};
		\node (evaporator) [block, left=of evaporator] {Evaporador};
		
		% un salto más
		\node (sobretension) [block, below=of evaporator] {Protección contra sobretensión};
		\node (sensor) [block, right=of sobretension] {Sensor de temperatura y humedad};
		\node (output) [block, right=of sensor] {Salida de energía};
		
		% Flechas entre los bloques
		\draw [arrow] (input) -- (switch);
		\draw [arrow] (switch) -- (protector);
		
		% Flecha hacia abajo
		\draw [arrow] (protector) -- (controltemp);
		
		% Flecha de vuelta a la izquierda
		\draw [arrow] (controltemp) -- (compresor);
		\draw [arrow] (compresor) -- (evaporator);
		
		\draw [arrow] (evaporator) -- (sobretension);
			\draw [arrow] (sobretension) -- (sensor);
			\draw [arrow] (sensor) -- (output);
		
	\end{tikzpicture}
	\caption{Esquema a bloques del sistema eléctrico.}
	Fuente: Elaboración propia usando \texttt{Tikz}
	\label{fig:4-blockelectric}
\end{figure} 



 \subsection{Protección contra sobrecarga}
 La protección contra sobrecarga es un mecanismo diseñado para evitar daños en un circuito eléctrico debido a corrientes excesivas. Este sistema se implementa para garantizar la seguridad del sistema eléctrico y proteger sus componentes de posibles fallas o daños. Es esencial contar con un dispositivo de protección contra sobrecarga en el sistema de refrigeración para asegurar su seguridad, fiabilidad y cumplimiento normativo.
 
 En el contexto del sistema de la cámara de refrigeración para este proyecto, se utilizará un disyuntor (ver figura \ref{fig:4-disyuntor}). Este dispositivo abrirá el circuito en caso de sobrecarga, y tras corregir cualquier fallo que se presente en el sistema, se podrá reiniciar manualmente, restableciendo así su funcionalidad original.
 
 \begin{figure}[H]
 	\centering
 	\includegraphics[width=0.6\linewidth]{figures/4-disyuntor}
 	\caption{Disyuntor simbología}
 	Fuente: \cite{areatecnologia}
 	\label{fig:4-disyuntor}
 \end{figure}
 	 
 \subsection{Protección Contra Cortocircuito}
 
 Un cortocircuito ocurre cuando se forma una conexión eléctrica no deseada entre dos puntos de un circuito que normalmente no deberían estar conectados. Esto puede resultar en corrientes extremadamente altas que fluyen a través del circuito, provocando sobrecalentamiento, incendios e incluso explosiones. Para proteger el sistema contra cortocircuitos, se implementará un fusible (ver figura \ref{fig:4-fusible}) que se colocará en serie en la conexión principal. Al detectar una corriente muy alta, el fusible se fundirá, impidiendo el paso de electricidad y protegiendo así la integridad de los componentes. Una vez corregido el fallo, el fusible puede ser fácilmente reemplazado por otro de características similares, permitiendo que el sistema vuelva a funcionar con normalidad.
 
 \begin{figure}[H]
 	\centering
 	\begin{tikzpicture}[scale=0.8]
 		
 		% Draw the fuse body
 		\draw[thick] (0,0) rectangle (1,3); % Rectángulo del fusible
 		
 		% Draw the vertical line in the middle
 		\draw[thick] (0.5, -1) -- (0.5, 4); % Línea vertical del fusible
 		
 		% Draw the labels
 		\node at (-0.5, 1.5) {-F};
 		\node at (1.2, 3.2) {1}; % Punto 1 arriba
 		\node at (1.2, -0.2) {2}; % Punto 2 abajo
 		
 	\end{tikzpicture}
 	\caption{Simbología de un fusible}
 	Fuente: Elaboración propia usando \LaTeX y \texttt{Tikz}
 	\label{fig:4-fusible}
 \end{figure}
 
 
 \subsection{Diseño del Sistema de Seguridad Eléctrico}
 
 El diseño del sistema eléctrico es un aspecto crucial en la planificación y ejecución de cualquier instalación eléctrica. Garantizar la seguridad de las personas, proteger los productos farmacéuticos y prevenir incidentes son objetivos prioritarios en este proceso.
 
 \subsubsection{Metas de la segurida eléctrica}
 \begin{enumerate}
 	\item Salvaguardar la integridad física de las personas que interactúan con el sistema de refrigeración.
 	\item Proteger los equipos y componentes eléctricos de daños causados por condiciones anormales de operación.
 	\item Minimizar el riesgo de incendios y otros accidentes eléctricos.
 	\item Cumplir con las normativas y estándares de seguridad eléctrica aplicables.
 \end{enumerate}
 
 \subsubsection{Elementos del Sistema de Seguridad}
 \begin{itemize}
 	\item \textbf{Protección Contra Sobrecarga:} Se implementará un disyuntor que será el encargado de proteger el sistema (ver figura \ref{fig:4-disyuntor}).
 	\item \textbf{Protección Contra Cortocircuito:} Se instalará un fusible que impedirá el paso de corriente en caso de cortocircuito, protegiendo así los componentes del sistema.
 	\item \textbf{Sistemas de Puesta a Tierra:} Estos sistemas servirán para proteger a las personas de descargas eléctricas; el disyuntor también cumplirá esta función.
 	\item \textbf{Aislamiento Eléctrico:} Se utilizarán materiales y técnicas adecuadas para garantizar un aislamiento seguro de conductores y equipos, evitando descargas y arcos eléctricos.
 \end{itemize}
 
 Considerando todos estos puntos, se diseñará un circuito que sea seguro para el operador de la cámara y mantenga la continuidad operativa de las instalaciones eléctricas.  
 
 
 
 
 \subsection{Tablero de Control}
 
 El panel de control del equipo seleccionado cuenta con un Controlador SR-4 desarrollado por Thermo King, que incorpora los últimos avances tecnológicos. Este panel no solo supervisa y regula la temperatura, sino que también gestiona integralmente el funcionamiento de la unidad de manera eficiente. A través de una interfaz de máquina humana (HMI), el microprocesador en la caja de control proporciona una operación intuitiva y amigable para el operador, mostrando información de funcionamiento de manera rápida y precisa.
 
 La caja de control está estratégicamente ubicada en la puerta de servicio inferior para facilitar el acceso y el mantenimiento. Además, los puertos USB integrados permiten la recuperación sencilla de datos del sistema de registro, lo que proporciona una herramienta invaluable para el análisis y la optimización del rendimiento del equipo. Este diseño avanzado del panel de control no solo mejora la eficiencia operativa, sino que también garantiza un monitoreo detallado y una gestión efectiva de todas las funciones críticas del sistema de refrigeración.
 
 En la figura 3.13 se muestra el tablero de control del equipo seleccionado, junto con las funciones que permite acceder cada botón, destacando su amigabilidad para el operador.

\begin{figure}[H]
	\centering
 \begin{tikzpicture}
	% Draw the main panel
	\draw[rounded corners=10pt, fill=white] (0,0) rectangle (6,4);
	
	% Draw the screen
	\draw[rounded corners=5pt, fill=gray!10] (1,2.5) rectangle (5,3.5);
	
	% Draw the buttons with a more appealing design
	\foreach \i in {0,1,2} {
		\draw[rounded corners=5pt, fill=gray!20] (1+\i*1.5,1) rectangle (2+\i*1.5,1.5);
		% Add shadow effect
		\draw[rounded corners=5pt, fill=black!10] (1+\i*1.5,1) rectangle (2+\i*1.5,1.55);
	}
	
	% Draw the button labels
	\node[font=\large\bfseries] at (1.5,1.25) {ON};
	\node[font=\large\bfseries] at (3,1.25) {{\tiny CYCLE}};
	\node[font=\large\bfseries] at (4.5,1.25) {OFF};
	
	% Draw the indicator lights with distinct colors
	\foreach \i in {1,2} {
		\fill[red] (5.2,3.2-\i*0.4) circle (0.1);
	}
	
	% Draw the temperature display
	\node[font=\huge\bfseries] at (3,3) {35.8};
	
	
	% Draw panel title
	\node[font=\Large\bfseries] at (3,3.8) {THERMO KING};
	
	
	
\end{tikzpicture}
\caption{Pantalla del panel de control}
Fuente: Elaboración propia basado de \hyperref[fig:axo-manual-thermo-king]{(Thermo King, 2024)}
\end{figure}
 
 
 
 
 
   1. Tecla On de encendido (tecla fija);
   2. Tecla Off de apagado (tecla fija);
   3. Pantalla;
   4. Tecla Descarche (tecla fija);
   5. Tecla de modo continuo o CYCLE-SENTRY (tecla fija);
   6. Teclas de software;
 
 
\subsection{Esquema del sistema eléctrico}
\begin{figure}[H]
	\centering
	\includegraphics[width=0.7\linewidth]{figures/4-diag-electrico}
	\caption{Diagrama eléctrico}
	Fuente: Elaboración propia en Proteus 8.2
	\label{fig:4-diag-electrico}	
\end{figure}

 \newpage

\section{Conclusión}


La redacción de este capítulo ha sido fundamental en la propuesta de solución para el diseño y selección de componentes de una cámara de refrigeración destinada a la conservación de insulina, especializada en pacientes diabéticos de la alcaldía Azcapotzalco en la Ciudad de México. Conocer y estructurar correctamente las cargas térmicas ha sido esencial para garantizar que el sistema de refrigeración se ajuste de manera precisa a los requisitos de conservación de la insulina, tomando en cuenta las condiciones climáticas específicas de la zona. 

Este proceso ha permitido prever las variaciones de temperatura externas y cómo estas afectan la operación del sistema, asegurando que la insulina mantenga sus propiedades farmacológicas y su eficacia durante todo su periodo de almacenamiento. La correcta estimación y estructuración de las cargas térmicas es crucial para diseñar un sistema de refrigeración que no solo cumpla con los estándares de conservación, sino que también sea eficiente en términos energéticos, reduciendo costos operativos y minimizando el impacto ambiental.

Finalmente, la eficiencia en la conservación de insulina en los hospitales públicos de la Ciudad de México es de vital importancia para garantizar el bienestar de los pacientes diabéticos, especialmente en zonas de alta demanda como Azcapotzalco. El correcto funcionamiento de estos sistemas permitirá evitar el deterioro de la insulina, asegurando un suministro constante y seguro para los pacientes que dependen de ella.





	 
 %\setcounter{page}{45}
 \clearpage
 %\pagenumbering{arabic}
 \newpage
 % \addcontentsline{toc}{chapter}{\hfill 34}
 \addtocontents{toc}{\protect\contentsline{chapter}{CAPÍTULO IV. Propuesta de diseño   \hfill  110}{}{}}
 
 

 \begin{titlepage}
 	
 	
 	\centering
 	\begin{tikzpicture}%opacity=0.5
 		\node[inner sep=0pt, ] (image) at (0,0) {\includegraphics[width=\textwidth]{figures/FinalDesigner}};
 		\fill [white,path fading=south] (-5,-4) rectangle (5,4);
 		\node[black,font=\Huge\bfseries] at (0,3) {Capítulo V. Análisis de costos del proyecto};
 		\node[black,font=\Large\bfseries] at (0,1) {Estudio de costos del proyecto};
 		\node[black,font=\Large\bfseries] at (0,0) {Costos directos};
 			\node[black,font=\Large\bfseries] at (0,-1) {Costos indirectos};
 		\node[black,font=\Large\bfseries] at (0,-2) {Estimación del costo del proyecto};
 	\end{tikzpicture}
 \end{titlepage}
 
 \newpage 
 
 \section*{Introducción}
 
 
 

 \setcounter{chapter}{5}
 \setcounter{page}{111}   
 \setcounter{section}{0}
 \setcounter{figure}{0}
 \setcounter{table}{0}
  \addcontentsline{toc}{section}{{Introducción}} 
La distribución de la cámara de refrigeración se detalla en la Figura \ref{fig:4-propuestasol}. El contenedor del refrigerador está adaptado con láminas compuestas de poliuretano (película interna $f_i$ - Poliuretano - película externa $f_e$) en las cuatro paredes, con el objetivo de minimizar la pérdida de temperatura por transferencia térmica a través de las superficies. Esta aislación contribuye a evitar el uso de ventiladores, mejorando la eficiencia energética en las etapas iniciales del funcionamiento.

En la parte posterior de la cámara se integra la unidad de refrigeración, cuya función es proteger los componentes del sistema. Esta unidad alberga los elementos principales, como el compresor, el condensador y el dispositivo de expansión, que están conectados directamente al evaporador. El evaporador está situado en el interior de la cámara y conectado al serpentín, cuya función es asegurar una mejor distribución del refrigerante dentro de la cámara, lo que permite una disipación de calor más eficiente y uniforme.

Además, la cámara está equipada con diversas tapas y una cubierta de cristal, diseñadas para mantener el medicamento en condiciones óptimas de almacenamiento.

La cámara tiene dimensiones de 60 × 60 × 51.2 centímetros. Es fundamental considerar aspectos como el flujo de aire interno y la distribución térmica. La selección del serpentín garantizará una transferencia de calor eficiente y uniforme, minimizando las zonas frías o calientes que podrían afectar la integridad del producto. Además, en los cálculos de secciones posteriores se considera la capacidad del evaporador para manejar la carga térmica en función de la cantidad y tipo de insulina almacenada, así como las condiciones ambientales externas propias de la Alcaldía.

 \section{Costos Directos}
 Son aquellos costos que tienen una conexión directa con el proyecto y se reflejan de manera tangible; se manifiestan en el desarrollo del proyecto y en la conclusión de este. Son los costos responsables de manifestar de manera tangible aquello que en su principio fue una propuesta de solución, tanto en los componentes de la obra, como en los operadores responsables del desarrollo del proyecto.
 
 \subsection{Mano de obra}
 Para el desarrollo del proyecto se necesita de apoyo técnico y personal capacitado para la ejecución e instalación de componentes específicos. En este proyecto, es fundamental contar con técnicos especializados para la instalación de la unidad de refrigeración, cuyo sistema garantizará la correcta refrigeración del medicamento en cuestión, y debe estar adaptado a las necesidades específicas del entorno de almacenamiento.
 
 
 Según las tarifas estimadas por el Gobierno de México y la Secretaría de Economía del México, los técnicos especializados en instalación de sistemas de refrigeración tienen un salario aproximado de \$10,250.00 MXN por instalación. Este cálculo proviene de fuentes especializadas en la instalación y puesta en marcha de sistemas de refrigeración, y se estima que el tiempo necesario para la instalación de todos los componentes, incluidos los trabajos de aislamiento, es de aproximadamente 2,400 minutos, lo que equivale a un tiempo máximo de trabajo de 5 días.

 \subsection{Distribución y Características de la Cámara de Refrigeración}
 \subsection{Consideraciones del Producto}
Las dimensiones de la cámara son 60 × 60 × 51.2 centímetros, lo que permite una distribución eficiente del aire y una distribución térmica adecuada. En este sentido, la selección del serpentín será clave para garantizar una transferencia de calor eficiente y minimizar las zonas frías o calientes que podrían afectar la integridad del producto. Además, la capacidad del evaporador se seleccionará en función de la carga térmica generada por la cantidad y tipo de insulina almacenada, así como las condiciones ambientales externas propias de la Alcaldía de Azcapotzalco.


 \subsubsection{Unidad Condensadora}
 La selección de la unidad condensadora toma en cuenta los siguientes parámetros esenciales para el correcto funcionamiento del sistema de refrigeración en la cámara destinada al almacenamiento de insulina:
 
 \begin{itemize}
 	\item \textbf{Carga térmica:} 0.0882 T.R. = 1,058.6 BTU/h
 	\item \textbf{Temperatura de almacenamiento:} 3 °C (39.2 °F)
 	\item \textbf{Temperatura de evaporación del refrigerante:} -20 °C (-4 °F)
 	\item \textbf{Temperatura del exterior:} 25 °C (77 °F)
 	\item \textbf{Temperatura de condensación del refrigerante:} 33.5 °C (92.3 °F)
 \end{itemize}
 
 Además de estos parámetros, es necesario considerar el tipo de refrigerante que emplea la unidad, ya que es el fluido de trabajo con el que opera todo el sistema. Se opta por el uso del refrigerante R-134a debido a su adecuado desempeño para sistemas de refrigeración de baja capacidad y temperaturas moderadas, como las requeridas en tu diseño. Este refrigerante ofrece una excelente eficiencia energética, es químicamente estable, tiene un bajo potencial de agotamiento de la capa de ozono (ODP=0), y cumple con normativas internacionales para aplicaciones médicas y farmacéuticas. Además, su disponibilidad en el mercado mexicano y su compatibilidad con compresores de pequeña escala y sistemas compactos lo hacen una opción viable. Los criterios de selección incluyen la capacidad de transferencia térmica, compatibilidad con los materiales de construcción del sistema, seguridad (baja toxicidad e inflamabilidad), impacto ambiental (bajo GWP), y facilidad de mantenimiento. Estas características lo posicionan como una solución eficiente, segura y sostenible para tu cámara de refrigeración. Si tu diseño requiere temperaturas extremadamente bajas, también podría evaluarse el uso de R-404A, aunque este tiene un mayor impacto ambiental.

Además cabe señalar que se elige dicho refriqerante ya que es el que maneja la unidad evaporadora, lo que garantiza la compatibilidad y eficiencia en el sistema de refrigeración.
 
 Es importante destacar que la elección de la unidad condensadora y el refrigerante deben ser coherentes con las condiciones operativas de la cámara de refrigeración y la carga térmica que se manejará, para evitar fluctuaciones de temperatura que puedan comprometer la efectividad del almacenamiento del medicamento.
 

 La cámara de refrigeración se encuentra diseñada para mantener las condiciones óptimas para el almacenamiento de insulina. La distribución del espacio y los componentes de la cámara se detallan en la figura 4.1, destacando el uso de láminas compuestas de poliuretano para minimizar la pérdida de temperatura por transferencia térmica. Estas láminas están ubicadas en las cuatro paredes de la cámara, lo que contribuye a evitar el uso de ventiladores y mejora la eficiencia energética en las etapas iniciales del funcionamiento.
 
 
 En la parte posterior de la cámara, se integra la unidad de refrigeración, cuya función es proteger los componentes del sistema. Esta unidad alberga los elementos principales, como el compresor, el condensador y el dispositivo de expansión, los cuales están conectados directamente al evaporador. El evaporador está ubicado en el interior de la cámara y conectado al serpentín, que asegura una distribución adecuada del refrigerante dentro de la cámara, permitiendo una disipación de calor más eficiente y uniforme.
 

 
 \subsection{Comparación de Unidades Condensadoras}
 
 El análisis a continuación presenta las características y ventajas de las unidades condensadoras BOHN e Ice Shadow, comparando tres opciones disponibles para la refrigeración en la UMF40. La elección de la unidad adecuada dependerá de varios factores, como la carga térmica necesaria, el costo y la facilidad de mantenimiento.
 
% Please add the following required packages to your document preamble:
% \usepackage{booktabs}
% Please add the following required packages to your document preamble:
% \usepackage{booktabs}
% 
% Beamer presentation requires \usepackage{colortbl} instead of \usepackage[table,xcdraw]{xcolor}
\begin{table}[H]
	\caption{Comparación de unidades condensadoras de Temperatura BAJA}
	Fuente: Elaboración propia, basado de \cite{bohn}.
	\centering
	  \scalebox{0.7}{ 
\begin{tabular}{@{}ccllccl@{}}
	\cmidrule(r){1-6}
	\textbf{Opción}   & \textbf{Proveedor/Marca} & \multicolumn{1}{c}{\textbf{Ventajas}}                                                                                                                                                           & \multicolumn{1}{c}{\textbf{Desventajas}}                                                                                                                               & \textbf{Costo\footnote{Al la fecha 13/diciembre/2024 el dólar amerciano (USD) tiene un costo de \$20.16 MXN}} & \multicolumn{1}{l}{\textbf{Elección}} &   \\ \cmidrule(r){1-6}
	\begin{tabular}[c]{@{}c@{}}IMCON012\\ 1/8 HP\end{tabular} & ICE SHADOW               & \begin{tabular}[c]{@{}l@{}}- Cumple con la carga \\       térmica necesaria. \\ - Facilidad de mantenimiento. \\ - Supervisor de consumo \\       de combustible. \\ - Silencioso.\end{tabular} & \begin{tabular}[c]{@{}l@{}}- El deshielo eléctrico \\ aumenta la carga térmica.\end{tabular}                                                                           & \$90 USD       &\textbf{ \textcolor[HTML]{34ff34}{ \checkmark }} &  \\ \cmidrule(lr){3-4}
\begin{tabular}[c]{@{}c@{}}CH161L6B\\ 1/4 HP\end{tabular}         & BOHN                     & \begin{tabular}[c]{@{}l@{}}- Marca Líder en el mercad. \\ - Facilidad de mantenimiento.\\ - Carga térmica aproximada.\end{tabular}                                                              & \begin{tabular}[c]{@{}l@{}}- No cumple con la carga \\       térmica requerida. \\ - Costo elevado .\\ - Diseñado para equipos \\     un poco más grandes\end{tabular} & \$9,095.00  USD    & \textbf{\textcolor[HTML]{fd6864}{\text{x}}}    &   \\ \cmidrule(lr){3-4}
\begin{tabular}[c]{@{}c@{}}CH111L6 \\ 1/4HP\end{tabular}         & BOHN                     & \begin{tabular}[c]{@{}l@{}}- Marca Líder en el mercad. \\ - Facilidad de mantenimiento.\\ - Carga térmica cerca a la necesitada.\end{tabular}                                                   & \begin{tabular}[c]{@{}l@{}}- No cumple con la carga\\     térmica requerida. \\ - Costo elevado..\\ - Diseñado para equipos \\      un poco más grandes\end{tabular}   & \$7,750.00  USD    & \textbf{\textcolor[HTML]{fd6864}{\text{x}}}    &   \\ \cmidrule(r){1-6}
\end{tabular}
}
\end{table}
 
 \subsubsection{Análisis de las Opciones}
 Cada una de las opciones presenta características particulares que pueden ser relevantes dependiendo de las necesidades específicas del proyecto en la UMF40. Se debe considerar que el costo es un factor importante, pero también lo es la eficiencia en el mantenimiento y la capacidad de enfriamiento.
 
 \begin{itemize}
 	\item \textbf{IMCON012 1/8 HP} Es la opción que mejor cumple con la carga térmica necesaria para el almacenamiento de insulina en la UMF40. Destaca por su fácil mantenimiento y el control del consumo de combustible. Además de ser una marca comercial muy común dentro del país.
 	\item \textbf{CH161L6B 1/4HP} Aunque es silenciosa y de fácil uso, no cumple con los requerimientos de carga térmica. Su costo también es elevado en relación con su utilidad en este proyecto, añadiendo su funcionalidad sobrada para la cámara diseñada, lo que puede hacerla menos atractiva para este tipo de proyecto.
 	\item \textbf{CH111L6B 1/4HP} Ofrece un menor tiempo de desescarche y un precio más bajo, pero carece de la funcionalidad de seguimiento del consumo de combustible y de carga térmica, lo que podría ser un inconveniente en términos de eficiencia energética.
 \end{itemize}
 
 \subsubsection{Preferencia y Elección}
La elección final se ha basado en los principios de los requisitos específicos de la cámara de refrigeración, así como en el presupuesto público disponible para la unidad condensadora. Se evaluaron varios factores clave, incluyendo la carga térmica necesaria, la eficiencia en el mantenimiento, y el coste total de propiedad (TCO), considerando tanto los costos iniciales como los costos operativos a largo plazo.

Tras un análisis exhaustivo, se concluye que la opción \textbf{IMCON012 - 1/8 HP} sería la más adecuada para satisfacer los requisitos de la instalación, debido a su óptima eficiencia energética y sus características de mantenimiento mínimas, lo que resulta en menores costos operativos y una vida útil prolongada. Además, esta opción se alinea con las especificaciones de carga térmica y capacidad de enfriamiento necesarias para garantizar un funcionamiento eficiente de la cámara de refrigeración.

El coste total de la opción \textbf{IMCON012 - 1/8 HP} es razonable y dentro del presupuesto asignado, lo que la convierte en una elección sostenible tanto a corto como a largo plazo. A nivel técnico, cumple con la carga térmica mínima requerida y tiene la capacidad de operar con eficiencia en la gama de temperaturas previstas para la cámara.

En resumen, la opción \textbf{IMCON012 - 1/8 HP} ofrece el mejor balance entre rendimiento, fiabilidad, coste y facilidad de mantenimiento, lo que la convierte en la elección más adecuada para la unidad condensadora, garantizando el cumplimiento de los requisitos técnicos y financieros de la instalación de refrigeración.

\begin{figure}[H]
	\centering
	\includegraphics[width=0.45\linewidth]{figures/condensador}
	\caption{Condensador ICE SHADOW elegida para el proyecto}
	Fuente: Tomado de (\citeauthor{ml2024}, 2024).
	\label{fig:condensador}
\end{figure}

 
 
 
  \subsection{Comparación de Unidades Evaporadoras}
 
El análisis a continuación presenta las características y ventajas de los evaporadores de las marcas MABE y BOHN, comparando tres opciones disponibles para la refrigeración en la UMF40. La elección del evaporador adecuado dependerá de varios factores clave, como la capacidad de evaporación, la eficiencia energética, la facilidad de mantenimiento y la compatibilidad con la unidad condensadora seleccionada. Además, se evaluará el costo total de propiedad y la durabilidad de cada opción para asegurar un funcionamiento óptimo en condiciones de operación constantes. La selección final buscará equilibrar el rendimiento térmico y la eficiencia económica, asegurando una solución integral que cumpla con los requisitos técnicos y presupuestarios establecidos para la instalación  
 
 \begin{table}[H]
 	\caption{Comparación de unidades evaporadoras de Perfil BAJO}
 	Fuente: Elaboración propia, basado de \cite{bohn2024}.
 	\centering
 	\scalebox{0.7}{ 
 \begin{tabular}{@{}ccllcc@{}}
 	\toprule
 	\textbf{Opción} & \textbf{Proveedor/Marca} & \multicolumn{1}{c}{\textbf{Ventajas}}                                                                                       & \multicolumn{1}{c}{\textbf{Desventajas}}                                                                                                      & \textbf{Costo} & \multicolumn{1}{l}{\textbf{Elección}}       \\ \midrule
 	IEM MABE        & MABE                     & \begin{tabular}[c]{@{}l@{}}- Diseñado para \\   equipos pequeños\\ - Adecuada al equipo\\ - Proveedor Mexicano\end{tabular} & \begin{tabular}[c]{@{}l@{}}- Al ser de gama\\  baja su tiempo \\  de vida es menor.\end{tabular}                                              & \$40.00  USD       & \textcolor[HTML]{34ff34}{ \checkmark}       \\ \cmidrule(lr){3-4}
 	Reach IN (TA)   & BOHN                     & \begin{tabular}[c]{@{}l@{}}- Equipo completo \\ - Diseño compacto\\ - Vendido por EU\end{tabular}                           & \begin{tabular}[c]{@{}l@{}}- El equipo viene\\ armado y no es\\ conveniente comprarlo\\ porque no se usarán toda\\ su capacidad.\end{tabular} & \$1,750.00  USD    & \textbf{\textcolor[HTML]{fd6864}{\text{x}}} \\ \cmidrule(lr){3-4}
 	Reach IN (TL)   & BOHN                     & \begin{tabular}[c]{@{}l@{}}- Gabinete de aluminio\\ - Diseño compacto\\ - Vendido por EU\end{tabular}                       & \begin{tabular}[c]{@{}l@{}}- El equipo está diseñado\\ para cámaras de al menos\\ 2m de alto.\end{tabular}                                    & \$7,750.00  USD    & \textbf{\textcolor[HTML]{fd6864}{\text{x}}} \\ \bottomrule
 \end{tabular}
 	}
 \end{table}
 
 \subsubsection{Análisis de las Opciones}
 Cada una de las opciones presenta características particulares que pueden ser relevantes dependiendo de las necesidades específicas del proyecto en la UMF40. Es fundamental evaluar, no solo el costo, sino también la eficiencia operativa, la facilidad de mantenimiento y la capacidad de enfriamiento de cada unidad.
 
 \begin{itemize}
 	\item \textbf{IEM MABE} (1/8 HP): Esta opción se ajusta perfectamente a la carga térmica necesaria para la conservación de la insulina en la UMF40. Destaca por su bajo costo, facilidad de mantenimiento y su origen local, lo que facilita el acceso a repuestos y soporte. Sin embargo, su vida útil es más corta debido a ser una opción de gama baja, lo cual debe ser considerado en términos de costos operativos a largo plazo.
 	\item \textbf{Reach IN (TA)} (1/4 HP): A pesar de ser un equipo de alta calidad con diseño compacto y eficiencia energética, no cumple con los requerimientos de carga térmica para este proyecto. Además, su costo es elevado en comparación con el beneficio que aporta, lo que la hace menos atractiva para este tipo de aplicaciones específicas.
 	\item \textbf{Reach IN (TL)} (1/4 HP): Aunque su diseño es robusto y adecuado para espacios más grandes, no es compatible con los requisitos de tamaño de la cámara en la UMF40. El costo elevado y el diseño no adecuado para cámaras más pequeñas hacen que esta opción sea menos conveniente para este proyecto.
 \end{itemize}
 
 \subsubsection{Preferencia y Elección}
 La selección final se ha basado en un análisis exhaustivo de las opciones disponibles, considerando los requerimientos específicos del proyecto, el presupuesto disponible, y los beneficios a largo plazo. Tras evaluar la carga térmica necesaria, la eficiencia de los equipos, y los costos operativos, se determinó que la opción \textbf{IEM MABE} (1/8 HP) es la más adecuada.
 
 Esta opción no solo cumple con la carga térmica mínima requerida, sino que también ofrece un excelente balance entre costo y eficiencia operativa. Aunque su vida útil es más corta, su bajo costo inicial y su fácil mantenimiento la convierten en una opción sostenible para la cámara de refrigeración en la UMF40.
 
 El costo total de la opción \textbf{IEM MABE} es adecuado dentro del presupuesto asignado, lo que hace de esta elección una solución económica y efectiva para este proyecto. A nivel técnico, cumple con todos los requisitos de capacidad de enfriamiento y estabilidad térmica necesaria para el almacenamiento de insulina.
 
 En resumen, la opción \textbf{IEM MABE} (1/8 HP) ofrece el mejor equilibrio entre rendimiento, costo y facilidad de mantenimiento, lo que la convierte en la elección más adecuada para este proyecto de refrigeración médica.
 
 
 \begin{figure}[H]
 	\centering
 	\includegraphics[width=0.6\linewidth]{figures/evaporador}
	\caption{Evaporador IEM MABE elegida para el proyecto}
	Fuente: Tomado de \cite{mabe2024}
 	\label{fig:evaporador}
 \end{figure}
 
  
 \subsection{Comparación de Unidades de motores eléctricos}
 
El análisis a continuación presenta las características y ventajas de los motores eléctricos para el sistema eléctrico del equipo, comparando tres opciones disponibles para su integración en la unidad de refrigeración de la UMF40 de Azcapotzalco.


 \begin{table}[H]
	\caption{Comparación de unidades motores eléctricos de potencia baja}
	Fuente: Elaboración propia, basado de \cite{ml2024}.
	\centering
	\scalebox{0.7}{ 
 \begin{tabular}{@{}ccllcc@{}}
 	\toprule
 	\textbf{Opción}                                                           & \textbf{Proveedor/Marca} & \multicolumn{1}{c}{\textbf{Ventajas}}                                                                                                                                                                           & \multicolumn{1}{c}{\textbf{Desventajas}}                                                                                                         & \textbf{Costo} & \multicolumn{1}{l}{\textbf{Elección}}          \\ \midrule
 	\begin{tabular}[c]{@{}c@{}}AP APPLI PARTS\\ APFM-51E \\ 120V\end{tabular} & Ap Appli Parts           & \begin{tabular}[c]{@{}l@{}}- La etapa de potencia\\ del equipo es similar \\ a la calculada\\ - Cuenta con protección \\ a fallas eléctricas.\\ - Es compatible con el \\ condensador y evaporador\end{tabular} & \begin{tabular}[c]{@{}l@{}}- Es genérico y debe \\ ser reemplazado en \\ el mantenimiento,\\ según las observaciones\\ del técnico.\end{tabular} & \$40.00 USD    & \textbf{\textcolor[HTML]{34ff34}{ \checkmark}} \\ \cmidrule(lr){3-4}
 	K50P-1125-001                                                             & DANFOS                   & \begin{tabular}[c]{@{}l@{}}- Precio menor\\ - Sistema más compacto\\ - Para equipos pequeños\end{tabular}                                                                                                       & \begin{tabular}[c]{@{}l@{}}- El sistema es sencillo\\ y no soportará cortos\\ circuitos\end{tabular}                                             & \$20.00 USD    & \textbf{\textcolor[HTML]{fd6864}{\text{x}}}    \\ \cmidrule(lr){3-4}
 	GenèricoRefri                                                             & Universal                & \begin{tabular}[c]{@{}l@{}}- Diseñado para sistemas pequeños\\ - Tamaño compacto\end{tabular}                                                                                                                   & \begin{tabular}[c]{@{}l@{}}- Motor genérico no \\ apto para el tiempo de \\ funcionamiento del equipo\\ médico.\end{tabular}                     & \$15.00 USD        & \textbf{\textcolor[HTML]{fd6864}{\text{x}}}    \\ \bottomrule
 \end{tabular}
	}
\end{table}


  
\subsubsection{Análisis de las Opciones}
Cada una de las opciones presenta ventajas y desventajas particulares que pueden ser relevantes dependiendo de las necesidades específicas de la cámara de refrigeración en la UMF40. A continuación, se justifica la elección del motor **AP APPLI PARTS APFM-51E 120V**:

\begin{itemize}
	\item \textbf{AP APPLI PARTS APFM-51E 120V}: Esta opción ha sido seleccionada debido a su adecuada compatibilidad con los componentes del sistema de refrigeración (condensador y evaporador). Además, su etapa de potencia es similar a la carga calculada para el proyecto, lo que garantiza un funcionamiento eficiente. Su protección a fallas eléctricas es una ventaja importante para mantener la seguridad del sistema. Aunque el motor es genérico y requiere reemplazo durante el mantenimiento, su costo accesible y su fiabilidad lo hacen ideal para este tipo de aplicación, donde la duración del equipo y los costos operativos son factores clave. 
	\item \textbf{K50P-1125-001 (DANFOS)}: Aunque tiene un costo relativamente bajo y es más compacto, esta opción presenta limitaciones en cuanto a su capacidad para soportar cortocircuitos. Además, la incompatibilidad con los requerimientos técnicos de la carga térmica hace que esta opción no sea adecuada para este proyecto.
	\item \textbf{GenèricoRefri (Universal)}: Aunque diseñado para equipos pequeños, este motor no es adecuado para los tiempos de funcionamiento de un equipo médico. Su vida útil limitada y las deficiencias en el rendimiento térmico lo hacen inapropiado para garantizar la eficiencia y seguridad en la conservación de insulina.
\end{itemize}

\subsubsection{Preferencia y Elección}
La opción seleccionada es el motor **AP APPLI PARTS APFM-51E 120V** debido a su fiabilidad, su compatibilidad con los componentes de refrigeración y su costo razonable. A pesar de ser un motor genérico, su protección contra fallas eléctricas y su adecuado rendimiento térmico lo convierten en la mejor opción para satisfacer los requisitos del sistema de refrigeración en la UMF40. Su bajo costo de adquisición y los beneficios a largo plazo en términos de mantenimiento hacen que esta opción sea la más adecuada para este proyecto.
  
 
  
  \begin{figure}[H]
  	\centering
  	\includegraphics[width=0.5\linewidth]{figures/motorelectrico}
 	\caption{Motor eléctrico seleccionado para el proyecto}
Fuente: Tomado de (\citeauthor{ml2024}, 2024).
  	\label{fig:motorelectrico}
  \end{figure}
  
  
  \subsection{Selección del Aislante Térmico: Espuma de Poliuretano}
  
  El aislamiento térmico adecuado es esencial para cualquier sistema de refrigeración, especialmente en una cámara de refrigeración destinada a la conservación de productos perecederos como la insulina. Para reducir la transferencia de calor entre el interior y el exterior de la cámara, se seleccionó la espuma de poliuretano como el material aislante principal. Este material fue elegido debido a su excelente capacidad de aislamiento térmico, que permite alcanzar el espesor necesario para mantener condiciones de temperatura óptimas dentro de la cámara.
  
  El poliuretano es un material que ofrece una conductividad térmica muy baja, lo que lo convierte en un excelente aislante. Además, su ligereza es una característica clave, ya que permite mantener el peso total del sistema de refrigeración dentro de límites razonables, lo que optimiza el consumo de energía y facilita el transporte de la unidad refrigerada sin exceder la carga útil. La resistencia a la humedad y a la compresión del poliuretano garantiza que el material conserve sus propiedades a lo largo del tiempo, lo que prolonga la vida útil del sistema y asegura un rendimiento constante bajo diversas condiciones ambientales.
  
  El volumen total que se desea llenar con espuma de poliuretano es aproximadamente $2.5 ft^3$. Este volumen se calcula para cubrir las necesidades térmicas de aislamiento de la cámara, asegurando una eficiencia máxima en la conservación de la temperatura. De acuerdo con las especificaciones del producto seleccionado, el costo de un kit de espuma expansiva de poliuretano de 750 ml es de aproximadamente  {\$250.00 MXN }. El costo total estimado para el volumen requerido de poliuretano es de {\$1,250.00 MXN} \cite{poliuretanohomedepot}.
  
  El uso de poliuretano como material aislante no solo asegura el cumplimiento de los estándares térmicos necesarios para el transporte seguro de productos sensibles como la insulina, sino que también contribuye a la eficiencia energética y la durabilidad del sistema de refrigeración, lo que lo convierte en una opción ideal para este tipo de aplicaciones.
  
 
\subsection{Resumen costos indirectos}
A continuación, se ofrece una tabla resumen (\ref{tabla:costosdirectos}	) de los costos directos del proyecto.
\begin{table}[H]
	\caption{Tabla resumen de los costos directos del proyecto. (Elaboración propia)}
	\label{tabla:costosdirectos}
		\scalebox{0.8}{ 
	\begin{tabular}{@{}clclcc@{}}
		\toprule
		\textbf{\begin{tabular}[c]{@{}c@{}}Num \\ Unidades\end{tabular}} & \multicolumn{1}{c}{\textbf{Unidad}}       & \textbf{Descripción}                                                                   & \multicolumn{1}{c}{\textbf{Proveedor}}                                                   & \textbf{\begin{tabular}[c]{@{}c@{}}Precio \\ Unitario (MXN)\end{tabular}} & \textbf{\begin{tabular}[c]{@{}c@{}}Precio\\ Total (MXN)\end{tabular}} \\ \midrule
		1                                                                & \multicolumn{1}{c}{Condensador}           & \begin{tabular}[c]{@{}c@{}}Condensador de BAJA,\\ marca ICE SHADOW 1/8 HP\end{tabular} & \multicolumn{1}{c}{\begin{tabular}[c]{@{}c@{}}Mercado Libre\\ o ICE SHADOW\end{tabular}} & \textdollar{1,812.03}                                                           & \textdollar{1,812.03}                                                             \\
		1                                                                & \multicolumn{1}{c}{Evaporador}            & \begin{tabular}[c]{@{}c@{}}Evaporador de BAJA,\\ marca: Iem MABE\end{tabular}          & \multicolumn{1}{c}{MABE}                                                                 & \textdollar{805.35}                                                             & \textdollar{805.35}                                                               \\
		1                                                                & Motor eléctrico                           & \begin{tabular}[c]{@{}c@{}}Motor APFM-51E 120V,\\ Marca: APPLI PARTS\end{tabular}      & APPLI PARTS                                                                              & \textdollar{805.35}                                                             & \textdollar{805.35}                                                               \\
		3                                                                & Poliuretano                               & \begin{tabular}[c]{@{}c@{}}Poliuretano expandido\\  universal\end{tabular}             & HOME DEPOT                                                                               & \textdollar{254.00}                                                             & \textdollar{762.00}                                                               \\
		1                                                                & \multicolumn{1}{c}{Instalación de equipo} & \begin{tabular}[c]{@{}c@{}}Instalación de equipo de\\ refrigeración\end{tabular}       & \multicolumn{1}{c}{Contrato}                                                             & \textdollar{10,250.00}                                                           & \textdollar{10,250.00}                                                            \\ \cmidrule(r){1-4}
		\multicolumn{1}{l}{}                                             &                                           & \multicolumn{1}{l}{}                                                                   &                                                                                          & \multicolumn{1}{l}{IVA 16\%}                                                   & \textdollar{2,309.56}                                                             \\
		\multicolumn{1}{l}{}                                             &                                           & \multicolumn{1}{l}{}                                                                   &                                                                                          & \multicolumn{1}{l}{Importe Final}                                             & \textdollar{12,744.29}                                                            \\  \bottomrule
	\end{tabular}
}
\end{table}
  
  
\section{Costos Indirectos}

Los costos indirectos son aquellos gastos que no están directamente asociados con la producción de un bien específico o la prestación de un servicio en particular, pero que son necesarios para el funcionamiento general de un proyecto. En el caso de la instalación de una cámara de refrigeración para la conservación de insulina, estos costos incluyen el mantenimiento de los equipos, gastos administrativos, alquiler de instalaciones, seguros, servicios públicos y la depreciación de los activos involucrados.

En proyectos de infraestructura como la instalación de sistemas de refrigeración, los costos indirectos pueden representar hasta un 30\% del presupuesto total del proyecto. La correcta estimación de estos costos es esencial por varias razones:

\begin{itemize}
	\item Son fundamentales para calcular el presupuesto total y garantizar que todos los aspectos operativos estén cubiertos.
	\item Permiten tomar decisiones informadas sobre la asignación de recursos, la planificación financiera y las estrategias de mitigación de riesgos.
	\item Facilitan una evaluación precisa de la rentabilidad del proyecto, ayudando a determinar su viabilidad.
\end{itemize}

\subsubsection{Costos de Ingeniería}

El costo de ingeniería es un componente clave al evaluar proyectos de instalación de sistemas complejos, como las cámaras de refrigeración para la conservación de insulina, donde la precisión y eficiencia son cruciales. Este costo involucra los recursos necesarios para el diseño, planificación e implementación del proyecto, incluyendo mano de obra, materiales y tiempo invertido.

De acuerdo con datos de \textit{Data México} y la figura \ref{fig:mecanicos}, el salario promedio mensual de los ingenieros mecánicos en México, durante el segundo trimestre de 2024, es de \$8,830 MXN. Este monto puede variar dependiendo de la región y el sector, ya que los mejores salarios para ingenieros mecánicos se encuentran en estados como Durango (\$42,300 MXN) y Tlaxcala (\$32,000 MXN) \cite{salarioingeniero}. 

Para calcular el costo de ingeniería de un proyecto como el de instalación de una cámara de refrigeración, se toma en cuenta el salario promedio de un ingeniero mecánico y el tiempo estimado para llevar a cabo todas las etapas del proyecto. Por ejemplo, si un ingeniero recién egresado gana alrededor de \$8,830 MXN al mes y dedica aproximadamente 3 meses a la planificación y ejecución de la instalación del sistema de refrigeración, el costo de ingeniería total podría ascender a más de \$26,000 MXN, dependiendo de la complejidad y duración del proyecto.

La correcta estimación de estos costos es crucial para asegurar que el proyecto se ejecute dentro de los límites financieros establecidos, maximizando la eficiencia y evitando desviaciones presupuestarias. La planificación adecuada y la asignación de recursos también garantizarán que el sistema de refrigeración para la conservación de insulina cumpla con los estándares de calidad y seguridad necesarios.
 
 
  
  \begin{figure}[H]
  	\centering
  	\caption{Costos por el Ingenieroo o Técnico}
  	\includegraphics[width=0.6\linewidth]{figures/mecanicos}
  	\caption{Insights y KPI del perfil de Ingenieros Mecánicos en México}
  	Fuente: \cite{salarioingeniero}
  	\label{fig:mecanicos}
  \end{figure}
  
  En la figura \ref{fig:mecanicos} se puede observar el ingreso mensual promedio de un ingeniero mecánico recién egresado en 2024. Para poder obtener el sueldo por hora se hacen las operaciones
  necesarias.
  
  Salario promedio mensual = \$8,300.00 MXN\\
  Horas semanales trabajadas = 42.3 h\\
  Semanas que tienes un mes= 4\\
  Salario promedio por hora =  \$49.06 MXN\\
  
  Tomando en cuenta estos datos se calcula el tiempo total que se le dedicó al proyecto y el costo del mismo tiempo (Tabla \ref{tabla:costopersonal})
  
  
 \begin{table}[H]
 	\centering
 	\caption{ Costos del personal}
 	\label{tabla:costopersonal}
\begin{tabular}{@{}cccccl@{}}
	\toprule
	\multicolumn{6}{c}{Tiempo invertido en el proyecto}                                \\ \midrule
	Horas/día & Horas/semana & Horas/mes & Meses & Horas totales & Costo total         \\ \midrule
	2         & 10           & 40        & 7     & 280           & \$13,736.8 MXN. \\ \bottomrule
\end{tabular}
 \end{table}
  
  
  La tabla \ref{tabla:costopersonal} muestra el cálculo de las horas trabajadas por el autor de este trabajo durante el proyecto.Este costo considera el número de integrantes en el desarollo de todo el proyecto para cálculos posteriores, en este caso solo es una persona y asciende a un total de \$3,456,178.88 MXN.
  
  \subsubsection{Costos por adquisición}
  Los costos de adquisición representan una inversión esencial para garantizar la ejecución exitosa del proyecto, incluyendo todos los elementos necesarios para su desarrollo. Entre estos, se consideran materiales técnicos como componentes para la cámara de refrigeración (condensador, evaporador y motor), servicios contratados para la instalación y puesta en marcha, así como recursos administrativos como papelería y permisos. Además, se incluyen los costos asociados a herramientas tecnológicas, como equipo de cómputo y licencias de software especializadas, como ANSYS, utilizadas para las simulaciones térmicas y el diseño estructural. La Tabla \ref{tabla:costos-adquisicion} resume detalladamente estos costos, permitiendo visualizar su distribución y su impacto en el presupuesto global del proyecto.
  
  % Please add the following required packages to your document preamble:
  % \usepackage{booktabs}
  \begin{table}[H]
  	\centering
  	\caption{Costos de adquisición. (Elaboración propia)}
  	\label{tabla:costos-adquisicion}
  	\begin{tabular}{@{}llll@{}}
  		\toprule
  		\multicolumn{1}{c}{Concepto}                                                                             & \multicolumn{1}{c}{Cantidad} & \multicolumn{1}{c}{Precio por unidad} & \multicolumn{1}{c}{Costo total} \\ \midrule
  		Bolígrafos                                                                                               & 3 unidades                   & \$ 7.00 MXN                            & \$21.00 MXN                     \\
\  		Hojas                                                                                                    & 1 paquete                    & \$129.00MXN                            & \$129.00 MXN                    \\
  		Folders                                                                                                  & 5 unidades                   & \$5.00 MXN                             & \$25.00 MXN                     \\
  		Impresiones                                                                                              & 150 hojas                    & \$1.00 MXN                             & \$150.00 MXN                    \\
  		\begin{tabular}[c]{@{}l@{}}Libro Buenas prácticas\\ de refrigeración y aire\\ acondicionado\end{tabular} & 1 unidad                     & \$250.00 MXN                           & \$250.00 MXN                    \\
  		Libro Refrigeración                                                                                      & 1 unidad                     & \$700.00 MXN                           & \$700.00 MXN                    \\
  		Internet                                                                                                 & 7 meses                      & \$599.00 MXN                           & \$4193.00 MXN                   \\
  		Software(SolidWorks)                                                                                     & Licencia 1 año               & \$84,602.00 MXN                        & \$84,602.00 MXN                 \\
  		Laptops                                                                                                  & 1 equipos                    & \$22,000.00 MXN                        & \$22,000.00 MXN                 \\
  		Pasajes                                                                                                  & 20 vueltas                   & \$20.00                                & \$400.00 MXN                    \\
  		&                              & \multicolumn{1}{r}{\textbf{TOTAL}}             & \$112,470.00 MXN                \\ \bottomrule
  	\end{tabular}
  \end{table}
  
  El total de los costos de adquisición que se obtiene a lo largo del proyecto es de  \$112,470.00 MXN  los cuales forman parte de los gastos indirectos del proyecto
  
  \subsubsection{Costo de desarrollo del proyecto}
  
  Los costos asociados al desarrollo del proyecto comprenden los gastos necesarios para garantizar su planificación, ejecución y conclusión exitosa. Este apartado incluye elementos como la mano de obra técnica, los materiales específicos para la construcción y montaje, la adquisición de tecnología, servicios de consultoría y capacitación especializada. En el caso particular de este proyecto, se realizaron consultorías técnicas enfocadas en la selección y dimensionamiento de equipos de refrigeración, así como en la integración de los accesorios del sistema eléctrico. Estas asesorías fueron fundamentales para asegurar que los componentes seleccionados cumplieran con las especificaciones técnicas requeridas, optimizando tanto el rendimiento del sistema como su costo. La Tabla \ref{tabla:costos-desarrollo} detalla la relación entre las horas invertidas en asesoramiento y el costo correspondiente, tomando como referencia el salario promedio de un ingeniero mecánico especializado en refrigeración.
  
  
  
  
  % Please add the following required packages to your document preamble:
  % \usepackage{booktabs}
  \begin{table}[]
  	\centering
  	\caption{Tabla de costos del desarrollo del proyecto}
  	\label{tabla:costos-desarrollo}
  	\begin{tabular}{@{}ccccc@{}}
  		\toprule
  		Horas/mes & Num. meses & Total de horas & Salario/hora & Total         \\ \midrule
  		2         & 7          & 14             & \$49.06 MXN  & \$ 686.84 MXN \\ \bottomrule
  	\end{tabular}
  \end{table}
  
  \subsection{Costo de la Protección Económica del Proyecto}
  La protección económica de un proyecto es un elemento clave para salvaguardar sus finanzas y garantizar su viabilidad a largo plazo. Desde la concepción hasta la ejecución, los proyectos están expuestos a diversos factores internos y externos que pueden afectar su estabilidad financiera. Por ello, es fundamental considerar un porcentaje de "protección financiera" que permita mitigar posibles riesgos. Este porcentaje suele oscilar entre el 25\% y el 42\% del costo subtotal, dependiendo de la naturaleza y los requerimientos específicos del proyecto.
  
  Para este proyecto en particular, se ha determinado un porcentaje de protección económica del 30\%, el cual se calcula sumando los costos directos e indirectos asociados. El costo directo del proyecto asciende a \$12,744.2 MXN, mientras que el costo indirecto es de \$126,893.64 MXN. Al sumar ambos conceptos, se obtiene un subtotal de \$139,637.84 MXN, sobre el cual se aplica el porcentaje de protección económica.
  
  El monto destinado a la protección económica, equivalente al 30\% del subtotal, asciende a \$41,891.35 MXN. Este fondo adicional asegura que el proyecto disponga de los recursos necesarios para enfrentar contingencias financieras, garantizando su correcta ejecución y sostenibilidad a lo largo del tiempo (Tabla \ref{tabla:5costoprote}).
  
  \begin{table}[H]
  	\centering
  	\caption{Costo de la protección económica del proyecto.}
  	\label{tabla:5costoprote}
  	\begin{tabular}{@{}cc@{}}
  		\toprule
  		Concepto                                       & Costo (MXN)                       \\ \midrule
  		Subcosto                                       & \$ 139,637.84                     \\
  		\multicolumn{1}{l}{Protección económica (30\%)} & \multicolumn{1}{l}{\$ 181,529.19} \\ \bottomrule
  	\end{tabular}
  \end{table}
  
  
  
  \subsection{Costo Total del Proyecto}
  El costo total del proyecto se calcula sumando los costos directos e indirectos, obtenidos previamente como sub-costo, y el monto destinado a la protección económica. Este cálculo asegura que se incluyan todos los factores necesarios para la correcta ejecución y sostenibilidad del proyecto.
  
  \begin{table}[H]
  	\caption{Costo total del proyecto}
  	{Fuente: Elaboración propia (2024)}\\
  	\centering
  	\label{tabla:costo-total}
  	\begin{tabular}{lr}
\toprule
  		 {Concepto}             &  {Cantidad (MXN)} \\ \midrule
  		Sub-costo                     & \$139,637.84            \\  
  		Protección económica (30\%)   & \$41,891.35             \\  
  		 {Costo total}          &  {\$181,529.19}   \\ \bottomrule
  	\end{tabular}
  \end{table}
  
  \subsection{Costo de Venta del Proyecto}
  El costo de venta del proyecto se determina considerando varios factores, entre ellos el costo de producción, el margen de beneficio, el análisis de mercado, y las posibles contingencias. Para este proyecto, el margen de utilidad se ha definido en un 34\% sobre el costo total calculado. Este porcentaje asegura que el proyecto sea viable financieramente y permita un retorno de inversión adecuado.
  
  \begin{table}[H]
  	\caption{Costo de venta del proyecto}
  	{Fuente: Elaboración propia (2024)}\\
  	\centering
  	\label{tabla:costo-venta}
  	\begin{tabular}{lr}
  		\toprule
  		{Concepto}             & {Cantidad (MXN)} \\ \midrule
  		Costo total                   & \$181,529.19            \\  
  		Porcentaje de venta (34\%)    & \$61,719.93             \\  
  		 {Costo de venta}       &  {\$243,249.12}   \\  \bottomrule
  	\end{tabular}
  	
  \end{table}
  
  Como resultado, el costo de venta del proyecto asciende a un total de \$243,249.12 MXN. Este monto considera todos los factores relevantes, asegurando la viabilidad del proyecto sin pérdidas y garantizando un margen de beneficio sostenible para su implementación y comercialización.
  
  
  
  \section{Conclusión}
  
  
  El desarrollo de este proyecto representa un avance significativo en la mejora de las condiciones de conservación de medicamentos críticos, como la insulina, en un entorno urbano complejo como la Ciudad de México. A lo largo del trabajo, se abordaron múltiples aspectos esenciales para garantizar la viabilidad técnica y económica de la cámara de refrigeración diseñada específicamente para la Unidad Médica Familiar (UMF) 40.
  
  El análisis comenzó con la identificación de las necesidades particulares del entorno, como las fluctuaciones de temperatura y la altitud de 2,240 m sobre el nivel del mar, las cuales imponen restricciones únicas sobre el diseño y operación de los sistemas de refrigeración. A partir de estas condiciones, se realizó un cálculo detallado de la carga térmica, el cual consideró factores como la temperatura de almacenamiento requerida de 2°C a 8°C y las propiedades de los materiales seleccionados para el aislamiento térmico, como el poliuretano expandido. Estos cálculos fueron la base para la selección precisa de componentes clave, como el condensador, evaporador y motor eléctrico, asegurando que cada uno cumpla con los requisitos de eficiencia energética y capacidad operativa.
  
  La selección del equipo no solo estuvo orientada a cumplir con las especificaciones técnicas, sino también a garantizar la sostenibilidad financiera del proyecto. Se evaluaron cuidadosamente los costos directos, indirectos y el porcentaje de protección económica, asegurando una gestión financiera robusta. El costo total del proyecto se calculó de manera integral, lo que permite no solo su implementación inmediata, sino también su viabilidad a largo plazo, con un enfoque en la reducción de costos operativos y el cumplimiento de normativas médicas internacionales.
  
  Desde un punto de vista técnico, la cámara de refrigeración integra principios modernos de ingeniería térmica y ciencia de materiales. La compatibilidad entre los componentes seleccionados, como el evaporador de bajo perfil de MABE y el motor eléctrico AP APPLI PARTS, optimiza el rendimiento del sistema al mantener temperaturas constantes de manera eficiente y confiable. 
  
  La implementación de esta solución no solo asegura la conservación de insulina en condiciones óptimas, sino que también establece un precedente para futuros proyectos en el ámbito médico. La tecnología aplicada en este proyecto puede ser adaptada a otras necesidades críticas de almacenamiento, contribuyendo al fortalecimiento del sistema de salud mediante la reducción de pérdidas de medicamentos y la mejora en la calidad de atención a los pacientes.
  
 Finalmente, este proyecto logra un balance integral entre viabilidad técnica, sostenibilidad financiera y cumplimiento de los estándares médicos más exigentes. La cámara de refrigeración diseñada para la UMF 40 no solo resuelve un problema específico, sino que también ofrece una solución replicable y escalable, posicionándose como un ejemplo de cómo la ingeniería y la tecnología pueden contribuir significativamente al bienestar social.
  
  
  \section{Recomendaciones}
  Para garantizar el correcto funcionamiento, durabilidad y eficiencia de la cámara de refrigeración diseñada para la conservación de insulina, se proponen las siguientes recomendaciones:
  
  \begin{enumerate}
  	\item \textbf{Mantenimiento regular:}
  	\begin{itemize}
  		\item Realizar inspecciones periódicas para verificar el estado de todos los componentes, incluyendo el sistema de refrigeración, el aislamiento y las juntas de las puertas.
  		\item Limpiar los componentes regularmente para evitar la acumulación de polvo y suciedad, que pueden reducir la eficiencia del enfriamiento.
  		\item Programar revisiones profesionales con técnicos especializados para detectar y corregir cualquier problema potencial antes de que se convierta en un fallo mayor.
  	\end{itemize}
  	
  	\item \textbf{Control de temperatura:}
  	\begin{itemize}
  		\item Monitorear constantemente la temperatura interna de la cámara para asegurarse de que se mantenga dentro del rango deseado.
  		\item Verificar regularmente el funcionamiento de las alarmas de temperatura para asegurar una detección inmediata de anomalías.
  	\end{itemize}
  	
  	\item \textbf{Manejo adecuado:}
  	\begin{itemize}
  		\item No sobrecargar la cámara; asegúrese de no exceder la carga diseñada y permita suficiente espacio para la circulación del aire frío.
  		\item Distribuir los elementos de manera uniforme para evitar puntos calientes y garantizar una refrigeración homogénea.
  	\end{itemize}
  	
  	\item \textbf{Cuidados del sistema:}
  	\begin{itemize}
  		\item Inspeccionar las conexiones eléctricas y los cables para detectar desgaste o daños, asegurándose de que las conexiones estén seguras y libres de corrosión.
  	\end{itemize}
  	
  	\item \textbf{Sellos y puertas:}
  	\begin{itemize}
  		\item Revisar regularmente las juntas de las puertas para asegurarse de que no haya fugas de aire. Reemplazar cualquier junta dañada.
  		\item Asegurarse de cerrar correctamente las puertas y evitar abrirlas innecesariamente para mantener la temperatura interna.
  	\end{itemize}
  	
  	\item \textbf{Limpieza y desinfección:}
  	\begin{itemize}
  		\item Limpiar y desinfectar regularmente el interior de la cámara para evitar la acumulación de bacterias y moho. Usar productos de limpieza adecuados según las instrucciones del fabricante.
  	\end{itemize}
  	
  	\item \textbf{Ventilación:}
  	\begin{itemize}
  		\item Asegurarse de que el área alrededor de la cámara esté bien ventilada para evitar el sobrecalentamiento del motor y otros componentes.
  		\item Evitar bloquear las ventilas del sistema de refrigeración para mantener la eficiencia operativa.
  	\end{itemize}
  	
  	\item \textbf{Movilidad y transporte:}
  	\begin{itemize}
  		\item Mover la cámara con cuidado para evitar golpes y daños estructurales.
  	\end{itemize}
  	
  	\item \textbf{Almacenamiento adecuado:}
  	\begin{itemize}
  		\item Cuando no esté en uso, almacenar la cámara en un lugar seco, bien ventilado y protegido de la exposición directa al sol y temperaturas extremas.
  		\item Antes de almacenarla por períodos prolongados, desconectar, limpiar y secar completamente la cámara para evitar la formación de moho y malos olores.
  	\end{itemize}
  	
  	\item \textbf{Capacitación del personal:}
  	\begin{itemize}
  		\item Asegurarse de que todo el personal que maneja la cámara esté capacitado adecuadamente en su uso y mantenimiento.
  		\item Proporcionar acceso al manual de usuario y asegurar que se sigan todas las recomendaciones del fabricante.
  	\end{itemize}
  \end{enumerate}
  
  
  
	
	\chapter*{Glosario}
\addcontentsline{toc}{chapter}{Glosario}  
%\setcounter{chapter}{4} 
\textbf{A}
\begin{enumerate}[label={ },leftmargin=*]
	\item \textbf{Absorción:} Es la extracción de uno o más componentes de una mezcla de gases cuando los gases y los líquidos entran en contacto. El proceso se caracteriza por un cambio en el estado físico o químico de los componentes.
\end{enumerate}

\textbf{B}
\begin{enumerate}[label={ },leftmargin=*]
	\item \textbf{Barrido:} Práctica en refrigeración que consta en la limpieza de las tuberías que forman un circuito frigorífico mediante la impulsión (por medio de un gas a alta presión) de un fluido de limpieza que barre el interior de las tuberías.
\end{enumerate}

\textbf{C}
\begin{enumerate}[label={ },leftmargin=*]
	\item \textbf{Caída de presión:} La diferencia de presión entre dos puntos.
	\item \textbf{Calor latente:} Calor que provoca el cambio de estado de una sustancia sin cambio en la temperatura o presión.
	\item \textbf{Calor sensible:} Calor que cambia la temperatura de una sustancia. Puede ser medida con un termómetro.
	\item \textbf{Compresor:} Es el componente de una instalación frigorífica encargado de aspirar el refrigerante en estado gaseoso, para luego comprimirlo, y descargarlo hacia el condensador como refrigerante en estado gaseoso a alta temperatura y presión.
	\item \textbf{Conducción:} La transferencia de calor por contacto directo entre dos objetos a diferentes temperaturas. Esta toma lugar en sólidos y también entre sólidos que están en contacto directo con otro.
	\item \textbf{Convección:} El proceso mediante el cual gases y líquidos se mueven debido a cambios en la temperatura y presión.
\end{enumerate}

\textbf{E}
\begin{enumerate}[label={ },leftmargin=*]
	\item \textbf{Entalpía:} La cantidad total de energía térmica (calor) contenida en una sustancia. Esto depende de la naturaleza de la sustancia, presión y temperatura.
\end{enumerate}

\textbf{G}
\begin{enumerate}[label={ },leftmargin=*]
	\item \textbf{Gas no condensable:} Un gas que no cambia a estado líquido bajo condiciones normales de operación. Los gases no condensables en un sistema generalmente son la humedad o el aire.
\end{enumerate}

\textbf{R}
\begin{enumerate}[label={ },leftmargin=*]
	\item \textbf{Refrigerante:} Fluido en un sistema frigorífico que adquiere calor mediante su evaporación a baja temperatura y presión y entrega este calor mediante su condensación a alta presión y temperatura.
\end{enumerate}

\textbf{S}
\begin{enumerate}[label={ },leftmargin=*]
	\item \textbf{Sistema en cascada:} Es el arreglo en el cual dos o más sistemas frigoríficos operan en serie; el evaporador de una máquina enfría el condensador de la otra máquina.
\end{enumerate}

\textbf{T}
\begin{enumerate}[label={ },leftmargin=*]
	\item \textbf{Termostato:} Elemento de una instalación frigorífica que controla la temperatura de un recinto o ambiente. Mediante la apertura o cierre de un contacto, establece el corte o puesta en marcha de la instalación frigorífica.
	\item \textbf{Tonelada de refrigeración:} Cantidad de frío producido mediante el derretimiento de 1 tonelada de hielo en 24 horas.
	\item \textbf{Torre de enfriamiento:} Es un accesorio del condensador usado para enfriar agua.
\end{enumerate}

\textbf{V}
\begin{enumerate}[label={},leftmargin=*] %\textbf{\arabic*.}
	\item \textbf{Visor de líquido:} Tal como su nombre lo describe, la utilización de este elemento nos permite observar el pasaje del refrigerante. Se instala antes del dispositivo de expansión, y en algunos modelos, lleva indicador de humedad.
\end{enumerate}

	
	\bibliographystyle{apacite}
	\bibliography{referencias.bib}
	
	\newpage
\setcounter{chapter}{6}
\setcounter{section}{0}
\rsp \rsp \chapter*{Anexos: Planos de Ingeniería}\rsp\rsp
\addcontentsline{toc}{chapter}{{Anexos}} 
\addcontentsline{toc}{section}{{Planos de Ingeniería}} 

\section{Anexo 1. Vista explosionada de la cámara de refrigeración} 

\begin{minipage}{\textwidth} 
	\begin{figure}[H]
	\centering 
	\caption*{\textit{Continuación de anexo 4.}}\rsp\rsp\rsp\rsp
	\vspace{6cm}
	\includepdf[scale=0.75, angle=90]{pdfs/vista-explosionada.pdf}
	\label{axo:vista-explosionada}
\end{figure} 
\end{minipage}
\newpage
\section{Anexo 2. Cuatro vistas de la cámara de refrigeración} 
\begin{minipage}{\textwidth} 
	\begin{figure}[H]
		\centering 
		\includepdf[scale=0.75, angle=90]{pdfs/vistas-camara.pdf}
		\label{axo:vista-camara}
	\end{figure} 
\end{minipage}

\newpage
\section{Anexo 3. Diagrama eléctrico completo} 
\begin{minipage}{\textwidth} 
	\begin{figure}[H]
		\centering 
		\includepdf[scale=0.75, angle=90]{pdfs/plano-electrico.pdf}
		\label{axo:diag-electrico}
	\end{figure} 
\end{minipage}

 




\newpage
\begin{minipage}{\textwidth}
	\section*{Tablas}
	\addcontentsline{toc}{section}{{Tablas}}\rsp \rsp	 
		\begin{table}[H]
 \centering
 \caption*{Anexo 4. Requerimientos y propiedades de almacenamiento para productos perecederos}\rsp\rsp\rsp
\includepdf[pages=1,scale=0.8]{bohn-perecederos.pdf}
 \label{anexo:bohn-perecederos}
		\end{table}		
	\end{minipage}
	
	
	 \newpage
	\begin{table}[H]
		\centering
		\caption*{\textit{Continuación de anexo 4.}}\rsp\rsp\rsp\rsp
		\includepdf[pages=2,scale=0.9]{bohn-perecederos.pdf}
		\label{anexo:bohn-perecederos2}
	\end{table}
	
	\newpage
	\section*{Imágenes}
		\addcontentsline{toc}{section}{{Imágenes}} 
	
	\begin{figure}[H]
		\centering
		%\includegraphics[width=0.6\linewidth]{figures/axo-design-div}
		\includegraphics[width=0.415\linewidth]{figures/axo-parrilasycharolas}	\includegraphics[width=0.415\linewidth]{figures/axo-parrilasycharolas2}
		\caption*{Anexo 5. Idea general de las divisiones al interior de la cámara.}
		\label{axo-design-div}
	\end{figure}
	
	 	\begin{figure}[H]
	 	\centering
	 	\includegraphics[width=0.7\linewidth]{figures/planos.pdf}
	 	\caption*{Anexo 6. Vistas de la cámara (diseño preliminar - capítulo 3).}
	 	\label{axo-planos}
	 \end{figure}
	 
 \begin{figure}[H]
 	\centering
 	\includegraphics[width=0.8\linewidth]{figures/condensador-bohn}
	\caption*{Anexo 7. Manual Bohn para unidades condensadoras.}
 	\label{fig:axo-manual-thermo-king}
 \end{figure}
 
	 
	 
	
	
\end{document}

