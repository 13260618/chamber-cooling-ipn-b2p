
\usepackage[utf8]{inputenc}
\usepackage[spanish,es-noquoting]{babel} %es-noquo... para poner . en equations
\decimalpoint

\usepackage{amssymb}
\usepackage{amsmath}

\usepackage{tabularx}
\usepackage{amsfonts}
\usepackage{array}
\usepackage{graphicx}

\usepackage{multicol}
\usepackage{multirow}

\usepackage{makeidx} % para las tablas de contenidos índices etc


\usepackage{lscape}
\usepackage{float}
\usepackage{array}

\usepackage{lmodern} 
\usepackage{fancyhdr}

\fancypagestyle{plain}{%
	\fancyhf{} % Limpia todos los encabezados y pies de página anteriores
	\renewcommand{\headrulewidth}{0pt} % Sin línea de encabezado
	\renewcommand{\footrulewidth}{0pt} % Sin línea de pie de página
	\fancyfoot[c]{\thepage} % Números de página en la posición central
}



%  usando arial en el trabajo, lo quito para el paper
%\usepackage[T1]{fontenc}
%\usepackage{helvet}

\renewcommand*\familydefault{\sfdefault}
\usepackage{setspace} % interlineados
\usepackage{parskip}  % sangrias 
\setlength{\parindent}{0pt} %sangria párrafos
\onehalfspacing % interlineado total

\usepackage{sectsty}
\sectionfont{\fontsize{12}{15}\selectfont} % Tamaño 12 y espaciado 15
\subsectionfont{\fontsize{12}{15}\selectfont} % Tamaño 12 y espaciado 15

\usepackage{longtable}
\usepackage{booktabs}
\usepackage{ragged2e} 
\usepackage{pdfpages}

%\usepackage{hyperref}

\usepackage[hidelinks]{hyperref} % hidelinks para quitar los bordes de los enlaces


%\usepackage{xcolor}
 \usepackage{listings}
\usepackage{enumitem}
\usepackage{tikz}
\usetikzlibrary{decorations.markings}

\usetikzlibrary{shapes,arrows}
\usetikzlibrary{shapes.geometric, arrows}

\usetikzlibrary{mindmap, trees}
\usetikzlibrary{fadings}
\usetikzlibrary{patterns} %pared compuesta 

\usetikzlibrary{shapes.geometric, arrows.meta, positioning}
\tikzstyle{block} = [rectangle, draw, fill=blue!20, text width=2.5cm, text centered, minimum height=1.5cm, rounded corners]
\tikzstyle{arrow} = [draw, -{Stealth[scale=1.5]}, thick]
\usepackage{pgfplots} %graphs // temperature graph chap5

\usepackage{fontawesome} % Para íconos
%\usepackage{fontawesome5}
%\usetikzlibrary{positioning}



%\usepackage{natbib}

\usepackage{titlesec}

\titleformat{\chapter}[display]
{\normalfont\huge\bfseries}{\chaptertitlename\ \thechapter}{11}{\Huge}
% Establecer el espaciado antes y después de los capítulos
\titlespacing*{\chapter}{0pt}{-40pt}{15pt} % Ajusta el último valor para cambiar el espacio después del título

\usepackage{caption} 


\usepackage{tabularx}
\usepackage{array}
\usepackage{multirow}



\usepackage{etoolbox} % Para modificar la numeración de página
\usepackage{tocbibind}  %numeracion dif


%tabla productos
\usepackage[normalem]{ulem}



 % Configura el punto como separador decimal
\usepackage{siunitx}
\sisetup{output-decimal-marker = {.}}


\usepackage{multirow}
%\usepackage[table,xcdraw]{xcolor}


\usepackage{tocloft}% personalizar indices

% Configurar un índice de ecuaciones con un nombre personalizado (por ejemplo, "eqn")
\newlistof{eqn}{section}{Índice de Ecuaciones}  % Cambiar 'equation' a 'eqn'


