\chapter*{Glosario}
\addcontentsline{toc}{chapter}{Glosario}  
%\setcounter{chapter}{4} 
\textbf{A}
\begin{enumerate}[label={ },leftmargin=*]
	\item \textbf{Absorción:} Es la extracción de uno o más componentes de una mezcla de gases cuando los gases y los líquidos entran en contacto. El proceso se caracteriza por un cambio en el estado físico o químico de los componentes.
\end{enumerate}

\textbf{B}
\begin{enumerate}[label={ },leftmargin=*]
	\item \textbf{Barrido:} Práctica en refrigeración que consta en la limpieza de las tuberías que forman un circuito frigorífico mediante la impulsión (por medio de un gas a alta presión) de un fluido de limpieza que barre el interior de las tuberías.
\end{enumerate}

\textbf{C}
\begin{enumerate}[label={ },leftmargin=*]
	\item \textbf{Caída de presión:} La diferencia de presión entre dos puntos.
	\item \textbf{Calor latente:} Calor que provoca el cambio de estado de una sustancia sin cambio en la temperatura o presión.
	\item \textbf{Calor sensible:} Calor que cambia la temperatura de una sustancia. Puede ser medida con un termómetro.
	\item \textbf{Compresor:} Es el componente de una instalación frigorífica encargado de aspirar el refrigerante en estado gaseoso, para luego comprimirlo, y descargarlo hacia el condensador como refrigerante en estado gaseoso a alta temperatura y presión.
	\item \textbf{Conducción:} La transferencia de calor por contacto directo entre dos objetos a diferentes temperaturas. Esta toma lugar en sólidos y también entre sólidos que están en contacto directo con otro.
	\item \textbf{Convección:} El proceso mediante el cual gases y líquidos se mueven debido a cambios en la temperatura y presión.
\end{enumerate}

\textbf{E}
\begin{enumerate}[label={ },leftmargin=*]
	\item \textbf{Entalpía:} La cantidad total de energía térmica (calor) contenida en una sustancia. Esto depende de la naturaleza de la sustancia, presión y temperatura.
\end{enumerate}

\textbf{G}
\begin{enumerate}[label={ },leftmargin=*]
	\item \textbf{Gas no condensable:} Un gas que no cambia a estado líquido bajo condiciones normales de operación. Los gases no condensables en un sistema generalmente son la humedad o el aire.
\end{enumerate}

\textbf{R}
\begin{enumerate}[label={ },leftmargin=*]
	\item \textbf{Refrigerante:} Fluido en un sistema frigorífico que adquiere calor mediante su evaporación a baja temperatura y presión y entrega este calor mediante su condensación a alta presión y temperatura.
\end{enumerate}

\textbf{S}
\begin{enumerate}[label={ },leftmargin=*]
	\item \textbf{Sistema en cascada:} Es el arreglo en el cual dos o más sistemas frigoríficos operan en serie; el evaporador de una máquina enfría el condensador de la otra máquina.
\end{enumerate}

\textbf{T}
\begin{enumerate}[label={ },leftmargin=*]
	\item \textbf{Termostato:} Elemento de una instalación frigorífica que controla la temperatura de un recinto o ambiente. Mediante la apertura o cierre de un contacto, establece el corte o puesta en marcha de la instalación frigorífica.
	\item \textbf{Tonelada de refrigeración:} Cantidad de frío producido mediante el derretimiento de 1 tonelada de hielo en 24 horas.
	\item \textbf{Torre de enfriamiento:} Es un accesorio del condensador usado para enfriar agua.
\end{enumerate}

\textbf{V}
\begin{enumerate}[label={},leftmargin=*] %\textbf{\arabic*.}
	\item \textbf{Visor de líquido:} Tal como su nombre lo describe, la utilización de este elemento nos permite observar el pasaje del refrigerante. Se instala antes del dispositivo de expansión, y en algunos modelos, lleva indicador de humedad.
\end{enumerate}
